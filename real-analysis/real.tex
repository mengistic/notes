% ==> preamble

\usepackage{tikz}
\usetikzlibrary{arrows}
\usepackage{enumitem}
\usepackage{mathtools}
\usepackage{etoolbox}

\usepackage{amssymb}
\usepackage{amsmath}
\usepackage{framed}
\usepackage{pstricks}
\usepackage[framed]{ntheorem}


\mathtoolsset{centercolon} 
\DeclarePairedDelimiter\abs{\lvert}{\rvert}
\DeclarePairedDelimiter\ang{\langle}{\rangle}

%% my theorem styles
\newtheoremstyle{meng-margin}%
{\item[\hskip\labelsep {\bfseries ##1\ ##2}]}%
{\item[\llap{\itshape(##3)\quad\hskip\labelsep}  {\bfseries ##1\ ##2}]}%

\newtheoremstyle{meng-ex}%
{\item[\hskip\labelsep {\normalfont\bfseries ##1\ ##2}]}%
{\item[\hskip\labelsep {\normalfont\bfseries ##1\ ##2}\ {\itshape (##3)} ]}%

\newtheoremstyle{meng-thm}%
{\item[\hskip\labelsep {\normalfont\bfseries ##1\ ##2}]}%
{\item[\hskip\labelsep {\normalfont\bfseries ##1\ ##2}\ {\itshape (##3)} ]}%



%% exercise, example, definition
\theoremstyle{meng-ex}
\theorembodyfont{\normalfont\upshape}
\theoreminframepreskip{0pt}
\theoreminframepostskip{0pt}
\newtheorem{exercise}{Exercise}
\newtheorem{example}{Example}
\newframedtheorem{definition}{Definition}

\theoremstyle{meng-thm}
\theorembodyfont{\normalfont\itshape}
\theoremprework{\bigskip\hrule\leavevmode}
\theorempostwork{\hrule\leavevmode}
\newtheorem{theorem}{Theorem}
\newtheorem{lemma}{Lemma}
\newtheorem{corollary}{Corollary}



\DeclareMathOperator{\Aut}{Aut}
\DeclareMathOperator{\rot}{rot}
\def\IM{\text{Im}}
\def\inv{^{-1}}
\def\SL{\text{SL}_2(\mathbb R)}
\def\H{\mathbb{H}}
\def\D{\mathbb{D}}
\def\R{\mathbb{R}}
\def\C{\mathbb{C}}
\renewcommand{\vec}[1]{\mathbf{#1}}
\DeclarePairedDelimiterX\norm[1]\lVert\rVert{
	\ifblank{#1}{\:\cdot\:}{#1}
}



\usepackage{enumitem}
\def\labelenumi{\textbf{(\alph*)}}

\title{Notes on Real Analysis}
% <==

\begin{document}
\maketitle
% ==> Real Numbers
\chapter{Real Numbers}
\section{Properties of Real Numbers}
% ==> properties
For now, let me just assume that the set $\N,\Z$ and $\Q$ 
are already exist. Just note that they're an 
\emph{``Order Field''}. The thing is, there must be some 
number system that is quite \emph{larger} than $\Q$, because
for example, there is no $q\in\Q$ such that $q^2=2$.

To extend from $\Q$, we're going to make up a new number 
system (kinda cheat a little bit, dun you think?) denoted 
by $\R$, which has the addition operation
$(+)$ and multiplication $(\cdot)$ such that for all $a,b\in\R$
\[a+b\in\R\quad\text{and}\quad ab:=a\cdot b\in\R.\]
Since we extended from $\Q$, this new set $\R$ is going to inherit all
the properties from $\Q$ listed below

\begin{enumerate}[label={\bfseries[A\arabic*]}]
  \item for any $a,b\in\R$, then $a+b=b+a$ and $ab=ba$.
  \item for any $a,b,c\in\R$, then $a+(b+c)=(a+b)+c$ and 
    $a(bc)=(ab)c$
  \item there is a number (\emph{indentity}) $\theta\in\R$ 
    and $\gamma\in\R$ such that 
    $a+\theta=a\gamma=a$  for all $a\in\R$.
  \item for any $a\in\R$, 
    \begin{itemize}
      \item there exist an \emph{additive inverse} $\alpha$ 
        such that $a+\alpha=\theta$.
      \item if $a\neq \theta$, there exist a 
        \emph{multiplicative inverse} $\beta$ such that 
        $a\beta=\gamma$.
    \end{itemize}
  \item for any $a,b,c\in\R$, then $a(b+c)=ab+ac$.
\end{enumerate}

Well because $\Q\subset\R$ (subspace), the indentity $\theta$ 
and $\gamma$ are the same of those in $\Q$, which we already
know they're simply the numbers $0,1$. 
Note also that, we must assume that $0\neq 1$.
\begin{theorem}
  For any $a,b,c\in\R$, the following holds
  \begin{itemize}
    \item the \emph{additive} and the \emph{multiplicative} 
      indentity are unique. 
      (later denoted them by $0$ and $1$ resp.)
    \item the addition and multiplicative inverses are
      unique. (later denoted them by $-a$ and $a\inv$ resp.)
    \item if $a+c=b+c\iff a=b$.
    \item $a\cdot 0=0,~-a=(-1)a$ and $-(-a)=a$.
    \item if $ab=0$ then $a=0$ or $b=0$.
  \end{itemize}
\end{theorem}
The above theorem is not that hard to prove. However, I promise 
to come back to this point to provide a full proof about it.

What now? The set $\R$ here behaves the same as $\Q$. 
How is it possible that $\R$ is bigger that $\Q$?
Let's find the condition that $\Q$ is lack of.

A long the way, we're introduced \emph{bounded sets.}
% <==
% ==> bounded
\begin{definition}[Maximum and Minimum]
  Let $S\subset\R$.
  \begin{itemize}
    \item if there exists $M\in S$ such that 
      $M\geq s$ for all $s\in S$, then $M$ is said to be the
      maximum of $S$ and is denoted by $\max S$.
    \item if there exists $m\in S$ such that 
      $m\leq s$ for all $s\in S$, then $m$ is said to be the
      minimum of $S$ and is denoted by $\min S$.
  \end{itemize}
\end{definition}
\begin{definition}[Bounded Sets]
  The set $S\subseteq\R$ is called
  \begin{itemize}
    \item \emph{bounded above} if $\exists M\in\R$ such that 
      $s\leq M$ for all $s\in S$.
    \item \emph{bounded below} if $\exists m\in\R$  such that
      $m\leq s$ for all $s\in S$.
    \item \emph{bounded} if it is both bounded below and above, 
      that is $\exists M>0$ such that $\abs{s}<M$ for all $s\in S$.
  \end{itemize}
\end{definition}
\begin{definition}[Upper and Lower bounds]
  Let $S\subseteq\R$ be a bounded set. We denote
  \begin{itemize}
    \item the set of all upper bounds of $S$ by 
      $\mathcal{U}(S)=\set{M\in\R}{M\geq s \text{ for all }s\in S}$
    \item the set of all lower bounds of $S$ by
      $\mathcal{L}(S)=\set{m\in\R}{m\leq s\text{ for all }s\in S}$
  \end{itemize}
\end{definition}
\begin{definition}[Infimum and Supremum]
  Let $S$ be a bounded set. If
  \begin{itemize}
    \item the set $\mathcal{U}(S)$ has a minimum $\alpha$, then the set
      $S$ is said to have a \emph{supremum}, that is
      $\sup S:=\alpha$.
    \item the set $\mathcal{L}(S)$ has a maximum $\beta$,
      then the set $S$ is said to have an \emph{infimum}, that is
      $\inf S:=\beta$.
  \end{itemize}
\end{definition}
% <== 
% ==> AoC
\begin{axiom}[Axiom of Completeness]
  Every non-empty subset of $\R$ that is bounded above 
  has a supremum.
\end{axiom}
% <==


% <==




\end{document}
