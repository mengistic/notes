% ==> preamble

\documentclass[10pt]{memoir}
% ==> math
\usepackage{amsthm}
\usepackage{amsfonts}
\usepackage{amssymb}
\usepackage{amsmath}
\usepackage{mathtools}

\mathtoolsset{centercolon} % not work when using |mathpazo|
\DeclarePairedDelimiter\abs{\lvert}{\rvert}
\DeclarePairedDelimiterX\norm[1]\lVert\rVert{
	\ifblank{#1}{\:\cdot\:}{#1}
}
\def\set#1#2{\left\{#1 ~:~ #2\right\}}
\def\permil{\text{\hskip 0.3pt\englishfont\textperthousand}}


\DeclareMathOperator{\arccot}{cot}
\DeclareMathOperator{\arcsec}{arcsec}
\DeclareMathOperator{\arccsc}{arccsc}
\DeclareMathOperator{\lcm}{lcm}
\DeclareMathOperator{\ord}{ord}
\DeclareMathOperator{\sym}{sym}


\def\N{\mathbb{N}}
\def\Z{\mathbb{Z}}
\def\Q{\mathbb{Q}}
\def\R{\mathbb{R}}
\def\C{\mathbb{C}}
\def\labelitemi{$\circ$}
\def\inv{^{-1}}
\def\ang#1{\left\langle#1\right\rangle}
\def\tran{^\mathrm{T}}
\renewcommand{\vec}[1]{\mathbf{#1}}

\theoremstyle{definition}
\newtheorem{definition}{Definition}[chapter]
\newtheorem{axiom}[definition]{Axiom}
\newtheorem{exercise}{Exercise}[section]
\newtheorem{example}[exercise]{Example}

\theoremstyle{plain}
\newtheorem{theorem}{Theorem}
\newtheorem{proposition}{Proposition}

\theoremstyle{remark}
\newtheorem{corollary}{Corollary}
\newtheorem{claim}{Claim}


% <== end: math
% ==> language
\usepackage[no-math]{fontspec}
\usepackage{mathpazo}
\setmainfont{TeX Gyre Pagella}
% <== 
% ==> setup
%\usepackage{geometry}
%\geometry{a4paper,
%  left=2.5cm, right=2.5cm,
%  top=3cm, bottom=3cm
%}
% <== 
% ==> enumerate
\usepackage{enumitem}
\usepackage{multicol}
% <==
% ==> title
\author{Sivmeng}
% <==
% ==> listings
\usepackage{xcolor}
\usepackage{listings}
% ==> basic
\lstset{%
	basicstyle=\small\ttfamily,
	keywordstyle=\color{black},
	commentstyle=\color{gray},
	keywordstyle=[1]{\color{blue!90!black}},
	keywordstyle=[2]{\color{magenta!90!black}},
	keywordstyle=[3]{\color{red!60!orange}},
	keywordstyle=[4]{\color{teal}},
	commentstyle=\color{gray},
	stringstyle=\color{green!60!black},
	tabsize=2,
	%
	numbers=left,
	numberstyle=\tiny\color{blue!70!gray},
	stepnumber=1,
	%
	frame=Lt,
	breaklines=true,
	xleftmargin=0cm,
	rulecolor=\color{gray!50!black},
	aboveskip=0.5cm,
	belowskip=0.5cm
}
% <==
% ==> code c
\lstdefinelanguage{cmeng}{
  morekeywords={
    auto,break,case,char,const,continue,default,do,double,%
    else,enum,extern,float,for,goto,if,int,long,register,return,%
    short,signed,sizeof,static,struct,switch,typedef,union,unsigned,%
    void,volatile,while},%
  morekeywords=[2]{
    printf, scanf,  include
  },
  sensitive,%
	morecomment=[l]{//},
	morecomment=[s]{/*}{*/},
	morestring=[b]',
	morestring=[b]",
}
% <==
% ==> code python
\lstdefinelanguage{py}{
	morekeywords={
		access,and,as,break,class,continue,def,del,elif,else,
		except,exec,finally,for,from,global,if,import,in,is,lambda,
		not,or,pass,print,raise,return,try,while},
	% Built-ins
	morekeywords=[2]{
		abs,all,any,basestring,bin,bool,bytearray,
		callable,chr,classmethod,cmp,compile,complex,delattr,dict,dir,
		divmod,enumerate,eval,execfile,file,filter,float,format,
		frozenset,getattr,globals,hasattr,hash,help,hex,id,input,int,
		isinstance,issubclass,iter,len,list,locals,long,map,max,
		memoryview,min,next,object,oct,open,ord,pow,property,range,
		raw_input,reduce,reload,repr,reversed,round,set,setattr,slice,
		sorted,staticmethod,str,sum,super,tuple,type,unichr,unicode,
		vars,xrange,zip,apply,buffer,coerce,intern,True,False},
	%
	morecomment=[l]\#,%
	morestring=[b]',%
	morestring=[b]",%
	morecomment=[s]{'''}{'''},% used for documentation text
	%                         % (mulitiline strings)
	morecomment=[s]{"""}{"""},% added by Philipp Matthias Hahn
	morestring=[s]{r'}{'},% `raw' strings
	morestring=[s]{r"}{"},%
	morestring=[s]{r'''}{'''},%
	morestring=[s]{r"""}{"""},%
	morestring=[s]{u'}{'},% unicode strings
	morestring=[s]{u"}{"},%
	morestring=[s]{u'''}{'''},%
	morestring=[s]{u"""}{"""},%
	%
	sensitive=true,%
}
% <==
% ==> code asy
\lstdefinelanguage{asy}{ %% Added by Sivmeng HUN
	morekeywords=[1]{
		import, for, if, else,new, do,and, access,
		from, while, break, continue, unravel, 
		operator, include, return},
	morekeywords=[2]{
		struct,typedef,static,public,readable,private,explicit,
		void,bool,int,real,string,var,picture,
		pair, path, pair3, path3, triple, transform, guide, pen, frame
	},
	morekeywords=[3]{
		true,false,and,cycle,controls,tension,atleast,
		curl,null,nullframe,nullpath,
		currentpicture,currentpen,currentprojection,
		inch,inches,cm,mm,pt,bp,up,down,right,left,
		E,N,S,W,NE,NW,SE,SW,
		solid,dashed,dashdotted,longdashed,longdashdotted,
		squarecap,roundcap,extendcap,miterjoin,roundjoin,
		beveljoin,zerowinding,evenodd,invisible
	},
	morekeywords=[4]{
		size,unitsize,draw,dot,label,
		sqrt,sin,cos,tan,cot,Sin,Cos,Tan,Cot,
		graph,
	},
	%
	morecomment=[l]{//},
	morecomment=[s]{/*}{*/},
	morestring=[b]',
	morestring=[b]",
	%
}
% <==
% <==



\title{Notes on Real Analysis}
% <==

\begin{document}
\maketitle
% ==> Real Numbers
\chapter{Real Numbers}
\section{Properties of Real Numbers}
% ==> properties
For now, let me just assume that the set $\N,\Z$ and $\Q$ 
are already exist. Just note that they're an 
\emph{``Order Field''}. The thing is, there must be some 
number system that is quite \emph{larger} than $\Q$, because
for example, there is no $q\in\Q$ such that $q^2=2$.

To extend from $\Q$, we're going to make up a new number 
system (kinda cheat a little bit, dun you think?) denoted 
by $\R$, which has the addition operation
$(+)$ and multiplication $(\cdot)$ such that for all $a,b\in\R$
\[a+b\in\R\quad\text{and}\quad ab:=a\cdot b\in\R.\]
Since we extended from $\Q$, this new set $\R$ is going to inherit all
the properties from $\Q$ listed below

\begin{enumerate}[label={\bfseries[A\arabic*]}]
  \item for any $a,b\in\R$, then $a+b=b+a$ and $ab=ba$.
  \item for any $a,b,c\in\R$, then $a+(b+c)=(a+b)+c$ and 
    $a(bc)=(ab)c$
  \item there is a number (\emph{indentity}) $\theta\in\R$ 
    and $\gamma\in\R$ such that 
    $a+\theta=a\gamma=a$  for all $a\in\R$.
  \item for any $a\in\R$, 
    \begin{itemize}
      \item there exist an \emph{additive inverse} $\alpha$ 
        such that $a+\alpha=\theta$.
      \item if $a\neq \theta$, there exist a 
        \emph{multiplicative inverse} $\beta$ such that 
        $a\beta=\gamma$.
    \end{itemize}
  \item for any $a,b,c\in\R$, then $a(b+c)=ab+ac$.
\end{enumerate}

Well because $\Q\subset\R$ (subspace), the indentity $\theta$ 
and $\gamma$ are the same of those in $\Q$, which we already
know they're simply the numbers $0,1$. 
Note also that, we must assume that $0\neq 1$.
\begin{theorem}
  For any $a,b,c\in\R$, the following holds
  \begin{itemize}
    \item the \emph{additive} and the \emph{multiplicative} 
      indentity are unique. 
      (later denoted them by $0$ and $1$ resp.)
    \item the addition and multiplicative inverses are
      unique. (later denoted them by $-a$ and $a\inv$ resp.)
    \item if $a+c=b+c\iff a=b$.
    \item $a\cdot 0=0,~-a=(-1)a$ and $-(-a)=a$.
    \item if $ab=0$ then $a=0$ or $b=0$.
  \end{itemize}
\end{theorem}
The above theorem is not that hard to prove. However, I promise 
to come back to this point to provide a full proof about it.

What now? The set $\R$ here behaves the same as $\Q$. 
How is it possible that $\R$ is bigger that $\Q$?
Let's find the condition that $\Q$ is lack of.

A long the way, we're introduced \emph{bounded sets.}
% <==
% ==> bounded
\begin{definition}[Maximum and Minimum]
  Let $S\subset\R$.
  \begin{itemize}
    \item if there exists $M\in S$ such that 
      $M\geq s$ for all $s\in S$, then $M$ is said to be the
      maximum of $S$ and is denoted by $\max S$.
    \item if there exists $m\in S$ such that 
      $m\leq s$ for all $s\in S$, then $m$ is said to be the
      minimum of $S$ and is denoted by $\min S$.
  \end{itemize}
\end{definition}
\begin{definition}[Bounded Sets]
  The set $S\subseteq\R$ is called
  \begin{itemize}
    \item \emph{bounded above} if $\exists M\in\R$ such that 
      $s\leq M$ for all $s\in S$.
    \item \emph{bounded below} if $\exists m\in\R$  such that
      $m\leq s$ for all $s\in S$.
    \item \emph{bounded} if it is both bounded below and above, 
      that is $\exists M>0$ such that $\abs{s}<M$ for all $s\in S$.
  \end{itemize}
\end{definition}
\begin{definition}[Upper and Lower bounds]
  Let $S\subseteq\R$ be a bounded set. We denote
  \begin{itemize}
    \item the set of all upper bounds of $S$ by 
      $\mathcal{U}(S)=\set{M\in\R}{M\geq s \text{ for all }s\in S}$
    \item the set of all lower bounds of $S$ by
      $\mathcal{L}(S)=\set{m\in\R}{m\leq s\text{ for all }s\in S}$
  \end{itemize}
\end{definition}
\begin{definition}[Infimum and Supremum]
  Let $S$ be a bounded set. If
  \begin{itemize}
    \item the set $\mathcal{U}(S)$ has a minimum $\alpha$, then the set
      $S$ is said to have a \emph{supremum}, that is
      $\sup S:=\alpha$.
    \item the set $\mathcal{L}(S)$ has a maximum $\beta$,
      then the set $S$ is said to have an \emph{infimum}, that is
      $\inf S:=\beta$.
  \end{itemize}
\end{definition}
% <== 
% ==> AoC
\begin{axiom}[Axiom of Completeness]
  Every non-empty subset of $\R$ that is bounded above 
  has a supremum.
\end{axiom}
% <==


% <==




\end{document}
