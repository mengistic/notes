

\usepackage{tikz}
\usetikzlibrary{arrows}
\usepackage{enumitem}
\usepackage{mathtools}
\usepackage{etoolbox}

\usepackage{amssymb}
\usepackage{amsmath}
\usepackage{framed}
\usepackage{pstricks}
\usepackage[framed]{ntheorem}


\mathtoolsset{centercolon} 
\DeclarePairedDelimiter\abs{\lvert}{\rvert}
\DeclarePairedDelimiter\ang{\langle}{\rangle}

%% my theorem styles
\newtheoremstyle{meng-margin}%
{\item[\hskip\labelsep {\bfseries ##1\ ##2}]}%
{\item[\llap{\itshape(##3)\quad\hskip\labelsep}  {\bfseries ##1\ ##2}]}%

\newtheoremstyle{meng-ex}%
{\item[\hskip\labelsep {\normalfont\bfseries ##1\ ##2}]}%
{\item[\hskip\labelsep {\normalfont\bfseries ##1\ ##2}\ {\itshape (##3)} ]}%

\newtheoremstyle{meng-thm}%
{\item[\hskip\labelsep {\normalfont\bfseries ##1\ ##2}]}%
{\item[\hskip\labelsep {\normalfont\bfseries ##1\ ##2}\ {\itshape (##3)} ]}%



%% exercise, example, definition
\theoremstyle{meng-ex}
\theorembodyfont{\normalfont\upshape}
\theoreminframepreskip{0pt}
\theoreminframepostskip{0pt}
\newtheorem{exercise}{Exercise}
\newtheorem{example}{Example}
\newframedtheorem{definition}{Definition}

\theoremstyle{meng-thm}
\theorembodyfont{\normalfont\itshape}
\theoremprework{\bigskip\hrule\leavevmode}
\theorempostwork{\hrule\leavevmode}
\newtheorem{theorem}{Theorem}
\newtheorem{lemma}{Lemma}
\newtheorem{corollary}{Corollary}



\DeclareMathOperator{\Aut}{Aut}
\DeclareMathOperator{\rot}{rot}
\def\IM{\text{Im}}
\def\inv{^{-1}}
\def\SL{\text{SL}_2(\mathbb R)}
\def\H{\mathbb{H}}
\def\D{\mathbb{D}}
\def\R{\mathbb{R}}
\def\C{\mathbb{C}}
\renewcommand{\vec}[1]{\mathbf{#1}}
\DeclarePairedDelimiterX\norm[1]\lVert\rVert{
	\ifblank{#1}{\:\cdot\:}{#1}
}



\usepackage{enumitem}
\def\labelenumi{\textbf{(\alph*)}}
\title{Problems in Real Analysis}
\author{Sivmeng HUN}
\date{\today}



\begin{document}
\maketitle
\chapter{Real Numbers}

\begin{example}
  Let $A$ be nonempty and bounded below.
  Prove that $\inf(A)$ exists.
\end{example}
\begin{proof}[Proof 1]
  Define $\L(A)$ to be the set of all lower bounds of $A$.
	Notice that $\L(A)$ is bounded above. Of course, let
  fix $a\in A$, therefore $b\leq a$ for all $b\in\L(A)$.
  This implies that $a$ is an upper bound of $\L(A)$, and thus
  bounded above. By AoC, it has supremum. Let's denote it's
  supremum by $m$. Thus
  \[m\leq a,\quad\forall a\in A.\]
  If $m_0$ arbitrary lower bound of $A$, then $m_0\in\L(A)$, then
  we must have $m_0\leq \sup(\L(A))=m$. This implies that
  $m=\sup(\L(A))$ is the greatest lower bound of $A$.
  Moreover, $\inf(A)=\sup(\L(A))$.
\end{proof}
\begin{proof}[Proof 2]
  Let $B:=\set{-a}{a\in A}$. Let $\ell$ be arbitrary lower bound
  of $A$. Then $\ell\leq a\iff -\ell\geq -a$ for all $a\in A$.
  This implies that $B$ is bounded above. By AoC, let
  $-s:=\sup B$. We conclude that
  \[
    \begin{cases}
    	-s \geq -a,\quad \forall a\in A\\
      \text{if }\forall a\in A,~ -s_0\geq -a\implies -s_0\geq s.
    \end{cases}
  \]
  Muliply both eqations by $-1$, we obtain that $s$ is the greatest
  lower bound of $A$. Moreover $\sup(-A)=-\inf(A)$.
\end{proof}




\end{document}