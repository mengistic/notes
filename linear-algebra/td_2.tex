
% ==> preamble

\documentclass[10pt]{memoir}
% ==> math
\usepackage{amsthm}
\usepackage{amsfonts}
\usepackage{amssymb}
\usepackage{amsmath}
\usepackage{mathtools}

\mathtoolsset{centercolon} % not work when using |mathpazo|
\DeclarePairedDelimiter\abs{\lvert}{\rvert}
\DeclarePairedDelimiterX\norm[1]\lVert\rVert{
	\ifblank{#1}{\:\cdot\:}{#1}
}
\def\set#1#2{\left\{#1 ~:~ #2\right\}}
\def\permil{\text{\hskip 0.3pt\englishfont\textperthousand}}


\DeclareMathOperator{\arccot}{cot}
\DeclareMathOperator{\arcsec}{arcsec}
\DeclareMathOperator{\arccsc}{arccsc}
\DeclareMathOperator{\lcm}{lcm}
\DeclareMathOperator{\ord}{ord}
\DeclareMathOperator{\sym}{sym}


\def\N{\mathbb{N}}
\def\Z{\mathbb{Z}}
\def\Q{\mathbb{Q}}
\def\R{\mathbb{R}}
\def\C{\mathbb{C}}
\def\labelitemi{$\circ$}
\def\inv{^{-1}}
\def\ang#1{\left\langle#1\right\rangle}
\def\tran{^\mathrm{T}}
\renewcommand{\vec}[1]{\mathbf{#1}}

\theoremstyle{definition}
\newtheorem{definition}{Definition}[chapter]
\newtheorem{axiom}[definition]{Axiom}
\newtheorem{exercise}{Exercise}[section]
\newtheorem{example}[exercise]{Example}

\theoremstyle{plain}
\newtheorem{theorem}{Theorem}
\newtheorem{proposition}{Proposition}

\theoremstyle{remark}
\newtheorem{corollary}{Corollary}
\newtheorem{claim}{Claim}


% <== end: math
% ==> language
\usepackage[no-math]{fontspec}
\usepackage{mathpazo}
\setmainfont{TeX Gyre Pagella}
% <== 
% ==> setup
%\usepackage{geometry}
%\geometry{a4paper,
%  left=2.5cm, right=2.5cm,
%  top=3cm, bottom=3cm
%}
% <== 
% ==> enumerate
\usepackage{enumitem}
\usepackage{multicol}
% <==
% ==> title
\author{Sivmeng}
% <==
% ==> listings
\usepackage{xcolor}
\usepackage{listings}
% ==> basic
\lstset{%
	basicstyle=\small\ttfamily,
	keywordstyle=\color{black},
	commentstyle=\color{gray},
	keywordstyle=[1]{\color{blue!90!black}},
	keywordstyle=[2]{\color{magenta!90!black}},
	keywordstyle=[3]{\color{red!60!orange}},
	keywordstyle=[4]{\color{teal}},
	commentstyle=\color{gray},
	stringstyle=\color{green!60!black},
	tabsize=2,
	%
	numbers=left,
	numberstyle=\tiny\color{blue!70!gray},
	stepnumber=1,
	%
	frame=Lt,
	breaklines=true,
	xleftmargin=0cm,
	rulecolor=\color{gray!50!black},
	aboveskip=0.5cm,
	belowskip=0.5cm
}
% <==
% ==> code c
\lstdefinelanguage{cmeng}{
  morekeywords={
    auto,break,case,char,const,continue,default,do,double,%
    else,enum,extern,float,for,goto,if,int,long,register,return,%
    short,signed,sizeof,static,struct,switch,typedef,union,unsigned,%
    void,volatile,while},%
  morekeywords=[2]{
    printf, scanf,  include
  },
  sensitive,%
	morecomment=[l]{//},
	morecomment=[s]{/*}{*/},
	morestring=[b]',
	morestring=[b]",
}
% <==
% ==> code python
\lstdefinelanguage{py}{
	morekeywords={
		access,and,as,break,class,continue,def,del,elif,else,
		except,exec,finally,for,from,global,if,import,in,is,lambda,
		not,or,pass,print,raise,return,try,while},
	% Built-ins
	morekeywords=[2]{
		abs,all,any,basestring,bin,bool,bytearray,
		callable,chr,classmethod,cmp,compile,complex,delattr,dict,dir,
		divmod,enumerate,eval,execfile,file,filter,float,format,
		frozenset,getattr,globals,hasattr,hash,help,hex,id,input,int,
		isinstance,issubclass,iter,len,list,locals,long,map,max,
		memoryview,min,next,object,oct,open,ord,pow,property,range,
		raw_input,reduce,reload,repr,reversed,round,set,setattr,slice,
		sorted,staticmethod,str,sum,super,tuple,type,unichr,unicode,
		vars,xrange,zip,apply,buffer,coerce,intern,True,False},
	%
	morecomment=[l]\#,%
	morestring=[b]',%
	morestring=[b]",%
	morecomment=[s]{'''}{'''},% used for documentation text
	%                         % (mulitiline strings)
	morecomment=[s]{"""}{"""},% added by Philipp Matthias Hahn
	morestring=[s]{r'}{'},% `raw' strings
	morestring=[s]{r"}{"},%
	morestring=[s]{r'''}{'''},%
	morestring=[s]{r"""}{"""},%
	morestring=[s]{u'}{'},% unicode strings
	morestring=[s]{u"}{"},%
	morestring=[s]{u'''}{'''},%
	morestring=[s]{u"""}{"""},%
	%
	sensitive=true,%
}
% <==
% ==> code asy
\lstdefinelanguage{asy}{ %% Added by Sivmeng HUN
	morekeywords=[1]{
		import, for, if, else,new, do,and, access,
		from, while, break, continue, unravel, 
		operator, include, return},
	morekeywords=[2]{
		struct,typedef,static,public,readable,private,explicit,
		void,bool,int,real,string,var,picture,
		pair, path, pair3, path3, triple, transform, guide, pen, frame
	},
	morekeywords=[3]{
		true,false,and,cycle,controls,tension,atleast,
		curl,null,nullframe,nullpath,
		currentpicture,currentpen,currentprojection,
		inch,inches,cm,mm,pt,bp,up,down,right,left,
		E,N,S,W,NE,NW,SE,SW,
		solid,dashed,dashdotted,longdashed,longdashdotted,
		squarecap,roundcap,extendcap,miterjoin,roundjoin,
		beveljoin,zerowinding,evenodd,invisible
	},
	morekeywords=[4]{
		size,unitsize,draw,dot,label,
		sqrt,sin,cos,tan,cot,Sin,Cos,Tan,Cot,
		graph,
	},
	%
	morecomment=[l]{//},
	morecomment=[s]{/*}{*/},
	morestring=[b]',
	morestring=[b]",
	%
}
% <==
% <==


\renewcommand{\thesection}{\arabic{section}.}
\renewcommand{\theexercise}{\arabic{section}.\arabic{exercise}}
\def\by{\times}
\def\theenumi{\alph{enumi})}
%\renewcommand{\thehomework}{\arabic{section}.\arabic{exercise}}
% <==

\setcounter{chapter}{1}

\begin{document}

\chapter{Systems of linear equations}

% ==> intro
\section{Different faces of linear transformation}
% <==
% ==> Echelon form
\section{Solution of a linear system. Echelon forms}

% ==> ex1
\begin{exercise}
  Write the systems of equations below in matrix form.
\end{exercise}
% <==
% ==> ex2
\begin{exercise}
  Find all solutions of the vector equation
  \[x_1\vec{v_1}+x_2\vec{v_2}+x_3\vec{v_3}=\vec{0}\]
  where $\vec{v_1}=(1,1,0)\tran,~\vec{x_2}=(0,1,1)\tran$ and 
  $\vec{v_3}=(1,0,1)\tran.$ What conclusion can you make about linear
  independence (dependence) of the system of vectors 
  $\vec{v_1},\vec{v_2},\vec{v_3}$.
\end{exercise}
\begin{proof}
  The echelon form of the system is
  \[
    \begin{pmatrix}
      1 &0 &1\\
      1 &1 &0\\
      0 &1 &1
    \end{pmatrix}
    \sim
    \begin{pmatrix}
      1 &0 &1\\
      0 &1 &-1\\
      0 &0 &2
    \end{pmatrix}
  \]
  so, clearly the solution to this equation is
  $x_1=x_2=x_3=0$.

  \textbf{Conclusion.} If the vectors $\vec{v_1},\vec{v_2},\vec{v_3}$
  are linearly independence, then the above equation has unique 
  solution, namely $x_1=x_2=x_3=0$.

  If the vectors $\vec{v_1},\vec{v_2},\vec{v_3}$ are linearly 
  dependence, then there exists $\alpha, \beta,\gamma$ 
  (some of them are non-zero) such that
  \[\alpha\vec{v_1}+\beta\vec{v_2}+\gamma\vec{v_3}=\vec{0}\]
  Therefore, the solution to the above equation is
  \[
    \begin{cases}
      x_1=\alpha t\\
      x_2=\beta t\\
      x_3=\gamma t
    \end{cases}
  \]
  for some $t\in\R$.
\end{proof}
% <==

% <==
% ==> Analyzing pivots
\section{Analyzing pivots}


\setcounter{exercise}{5}
% ==> ex6
\begin{exercise}
  Prove or disprove. If the columns of a square ($n\by n$)
  matrix are linearly independence, so are the columns of $A^2$.
\end{exercise}
\begin{proof}
  Since $A$ is a square matrix, and columns vectors are linearly 
  independence, so
  \begin{align*}
    &\text{has pivot in every column}\\\implies\quad
    &\text{has pivot in evry row}\\\implies\quad
    &A\text{ is invertable}\\\implies\quad
    &A^2\text{ also invertable}\\\implies\quad
    &A^2\text{ has pivot every column and row}
  \end{align*}
  Therefore, the column vectors of $A^2$ also independence.
\end{proof}
% <==
% ==> ex7
\begin{exercise}
  Prove or disprove. If the columns of a square ($n\by n$)
  matrix are linearly independence, so are the columns of $A^3$.
\end{exercise}
% <==
% ==> ex8
\begin{exercise}
  Show that if the equation $A\vec{x}=\vec{0}$ has unique solution,
  then $A$ is left invertable.
\end{exercise}
\begin{proof}
  Because $A$ has unique solution, then it has pivot in every
  column. Therfore, \#col $\leq$ \#row, so we let $m\by n$ be the size
  of $A$ where $n\leq m$
  Let $R$ be the reduced 
  echelon form of $A$, hence there exists $E=E_k\cdots E_2E_1$
  such that $R=EA$.
  Observe that, $R$ would look like
  \[
    R=
    \begin{pmatrix}
      \fbox{1} &0  &\cdots &0\\
      0 &\fbox{1}  &\cdots &0\\
      \vdots&\vdots&\ddots & \\
      0 &0  &\cdots &\fbox{1}\\
      0 &0  &\cdots &0\\
      0 &0  &\cdots &0\\
    \end{pmatrix}
  \]
  Using matrix muliplication gives us $R\tran R=I_n$. We obtain that
  \[
    R\tran EA=R\tran T=I_n
  \]
  Therfore, 
  \fbox{$A$ is left invertable, and $R\tran E$ is its left inverse}.



\end{proof}
% <==
% <==
% ==> find A-inverse
\section{Find $A\inv$ by row reduction}
% <==
% ==> dimension
\section{Dimension}

% ==> ex1
\begin{exercise}
  True or false.
  \begin{enumerate}
    \item Every vector space that is generated by a finite set has a basis;
    \item Every vector space has a (finite) basis;
    \item A vector space cannot have more that one basis;
    \item A vector space has a finite basis, then the number of vectors
      in every basis is the same;
    \item The dimension of $\P_n$ is $n$;
    \item The dimension of $M_{m\by n}$ is $m+n$;
    \item If vectors $\vec{v_1},\vec{v_2},\dots,\vec{v_n}$ generate (span)
      the vector space $V$, then every vector in $V$ can be written as
      a linear combination of vectors $\vec{v_1},\vec{v_2},\dots,\vec{v_n}$
      in only one way;
    \item Every subspace of a finite-dimensional space is finite-dimensional.
    \item If $V$ is a vector space having dimension $n$, then $V$ has exactly
      one subspace of dimension $0$, and exactly one subspace of dimension
      $n$.
  \end{enumerate}
\end{exercise}
\begin{proof}
  \begin{enumerate}
    \item \textbf{True.} That finite set which generated a vector space
      is the spanning set itself. Since it's finite, it contains a
      basis.
    \item \textbf{False.} Take $\R[x]$ for example.
    \item \textbf{False.} In $\R^2$, one can choose 
      \[
        \left\{\binom{1}{0}, \binom{0}{1}\right\}
        \quad\text{or}\quad
        \left\{\binom{1}{1}, \binom{0}{1}\right\}
      \]
      as a basis.
    \item \textbf{True.} As proved in the above theorems,
      \begin{align*}
        &\text{\# independence vectors}\leq \dim V\\
        &\text{\# generating vectors}\geq \dim V
      \end{align*}
      hence the number of any basis in $V$ must be exactly $\dim V$ vectors.
    \item \textbf{False.} In $\P_n$, the standard basis is
      \[
        1,~t,~t^2,\dots,t^n
      \]
      which has $n+1$ vectors. Hence $\boxed{\dim\P_n=n+1}$.
    \item \textbf{False.} The standard basis in $M_{m\by n}$ is
      \[
        \{\vec{e}_{11},\vec{e}_{12},\dots,\vec{e}_{mn}\}
      \]
      has $m\times n$ vectors. Hence $\boxed{\dim M_{m\by n}=mn}$.
    \item \textbf{False.} span doesn't guarantee uniqueness.
    \item \textbf{True.} Let $W$ be a subspace of $V$. Because $\dim V$
      finite, we can find 
      $$\mathcal{A}=\{\vec{v_1},\vec{v_2},\dots,\vec{v_n}\}\subset V$$
      that spans $V$. WLOG, we assume that none of vectors in $\mathcal{A}$
      belongs to $W$ (the unluckiest case.) We can choose 
      $\vec{w_1}\in W$ such that $\vec{w_1}\neq\vec{0}$. Then
      \[
        \vec{w_1}=\alpha_1\vec{v_1}+\cdots+\alpha_n\vec{v_n}
      \]
      Because $\vec{w_1}\neq 0$, we're sure some of the $\alpha_i$'s are
      non-zero, say $\alpha_1$. Then the new system
      \[
        \mathcal{A}_1=\{\vec{w_1},\vec{v_2},\dots,\vec{v_n}\}
      \]
      still spans the space $V$. Now, if $\mathcal{B}_1:=\{\vec{w_1}\}$
      doesn't span $W$, we can repeat the above procedure and find 
      $\vec{w_2}$. We can do this at most $n$ times, because once
      we reach the $n$th step, we have the new system 
      $\mathcal{A}_n\subset W$ that spans the whole space $V$.

      Therfore, after some finite $k\le n$ step, we have
      \[
        \mathcal{B}_k=\{\vec{w_1},\vec{w_2},\dots,\vec{w_k}\}\subset W
      \]
      spans $W$. Hence, $W$ is finite dimensional.
    \item \textbf{Not sure.}
  \end{enumerate}
\end{proof}
% <==
% ==> ex2
\begin{exercise}
  Prove that if $V$ is a vector space having dimension $n$, then
  a system of vectors $\vec{v_1},\vec{v_2},\dots,\vec{v_n}$ in $V$
  is linearly independent iff it spans $V$.
\end{exercise}
% <==
% <==
% ==> change of basis
\section{Change of basis}
\setcounter{exercise}{2}

% ==> ex3
\begin{exercise}
  Find the change of coordinates matrix that changes the 
  coordinates in basis $\{1,1+t\}$ in $\P_1$ to the 
  coordinates in the basis $\{1-t, 2t\}$.
\end{exercise}
\begin{proof}
  Let's denote $\mathcal{A}=\{1,1+t\}$ and 
  $\mathcal{B}=\{1-t, 2t\}$. Let $\mathcal{S}$
  be the standard basis in $\P_1$. Therefore, 
  the matrix that transforms from vector in basis
  $\mathcal{A}$ to basis $\mathcal{B}$ is 
  $[\mathcal{BA}]= [\mathcal{BS}][\mathcal{SA}]$.
  We have
  \[ [\mathcal{SA}] = \begin{pmatrix} 1 &1 \\ 0 &1 \end{pmatrix} \]
  and 
  \[
    [\mathcal{BS}]=[\mathcal{SB}]\inv=
    \begin{pmatrix} 1  & 0\\ -1 & 2 \end{pmatrix}\inv=\frac{1}{2}
    \begin{pmatrix} 2 &0\\ 1 &1 \end{pmatrix}
  \]
  Therefore, 
  \[
    [\mathcal{BA}]= \frac{1}{2}
    \begin{pmatrix} 2 &0\\ 1 &1 \end{pmatrix}
    \begin{pmatrix} 1 &1 \\ 0 &1 \end{pmatrix}.
  \]
\end{proof}
% <==
% ==> ex4
\begin{exercise}
  Let $T$ be the ...
\end{exercise}
\begin{proof}
  In standard basis, $T$ looks like
  \[ [T]= \begin{pmatrix} 3 &1\\ 1 &-2 \end{pmatrix}.\]
  Let $\mathcal{B}=\{(1,1)\tran, (1,2)\tran\}$. In basis $\mathcal{B}$,
  the transformation would look like
  \[
    [T]_{\mathcal{BB}}= [\mathcal{BS}] [T] [\mathcal{SB}]
  \]
  And we have
  \[ [\mathcal{SB}]= \begin{pmatrix} 1 &1\\ 1 &2 \end{pmatrix} \]
  so
  \[
    [\mathcal{BS}]= [\mathcal{SB}]\inv=
    \begin{pmatrix} 1 &1\\ 1 &2 \end{pmatrix}\inv=
    \begin{pmatrix} 2 &-1\\ -1&1 \end{pmatrix}
  \]
  Therefore, the transformation $[T]_{\mathcal{BB}}$ in basis
  $\mathcal{B}$ is
  \[
    [T]_{\mathcal{BB}}=
    \begin{pmatrix} 2 &-1\\ -1&1 \end{pmatrix}
    \begin{pmatrix} 3 &1\\ 1 &-2 \end{pmatrix}
    \begin{pmatrix} 1 &1\\ 1 &2 \end{pmatrix} 
  \]
\end{proof}
% <==
% ==> ex5
\begin{exercise}
  Prove that if $A$ and $B$ are similar matrices, then
  $\tr A=\tr B$.
\end{exercise}
\begin{proof}
  Because $A$ and $B$ are similar, then there exists 
  an invertable matrix $Q$ such that $A=Q\inv BQ$. Observe that
  \[
    A=Q\inv\cdot BQ
    \quad\text{and}\quad
    B=BQ\cdot Q\inv
  \]
  This implies that $\tr A=\tr B$.
\end{proof}
% <==

% <==



\end{document}
