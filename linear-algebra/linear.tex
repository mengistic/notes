
\usepackage{tikz}
\usetikzlibrary{arrows}
\usepackage{enumitem}
\usepackage{mathtools}
\usepackage{etoolbox}

\usepackage{amssymb}
\usepackage{amsmath}
\usepackage{framed}
\usepackage{pstricks}
\usepackage[framed]{ntheorem}


\mathtoolsset{centercolon} 
\DeclarePairedDelimiter\abs{\lvert}{\rvert}
\DeclarePairedDelimiter\ang{\langle}{\rangle}

%% my theorem styles
\newtheoremstyle{meng-margin}%
{\item[\hskip\labelsep {\bfseries ##1\ ##2}]}%
{\item[\llap{\itshape(##3)\quad\hskip\labelsep}  {\bfseries ##1\ ##2}]}%

\newtheoremstyle{meng-ex}%
{\item[\hskip\labelsep {\normalfont\bfseries ##1\ ##2}]}%
{\item[\hskip\labelsep {\normalfont\bfseries ##1\ ##2}\ {\itshape (##3)} ]}%

\newtheoremstyle{meng-thm}%
{\item[\hskip\labelsep {\normalfont\bfseries ##1\ ##2}]}%
{\item[\hskip\labelsep {\normalfont\bfseries ##1\ ##2}\ {\itshape (##3)} ]}%



%% exercise, example, definition
\theoremstyle{meng-ex}
\theorembodyfont{\normalfont\upshape}
\theoreminframepreskip{0pt}
\theoreminframepostskip{0pt}
\newtheorem{exercise}{Exercise}
\newtheorem{example}{Example}
\newframedtheorem{definition}{Definition}

\theoremstyle{meng-thm}
\theorembodyfont{\normalfont\itshape}
\theoremprework{\bigskip\hrule\leavevmode}
\theorempostwork{\hrule\leavevmode}
\newtheorem{theorem}{Theorem}
\newtheorem{lemma}{Lemma}
\newtheorem{corollary}{Corollary}



\DeclareMathOperator{\Aut}{Aut}
\DeclareMathOperator{\rot}{rot}
\def\IM{\text{Im}}
\def\inv{^{-1}}
\def\SL{\text{SL}_2(\mathbb R)}
\def\H{\mathbb{H}}
\def\D{\mathbb{D}}
\def\R{\mathbb{R}}
\def\C{\mathbb{C}}
\renewcommand{\vec}[1]{\mathbf{#1}}
\DeclarePairedDelimiterX\norm[1]\lVert\rVert{
	\ifblank{#1}{\:\cdot\:}{#1}
}



\usepackage{enumitem}
\def\labelenumi{\textbf{(\alph*)}}


\title{Notes on Linear Algebra}
\begin{document}
\maketitle

\chapter{Preliminaries}
\section{Vector Spaces}



%the set $E (\R^3)$
%\[\circ : E\times E\to E\]
%called binary operation.

%We use scalor multiplication to strech the vector, 
%called \emph{inner product.}
%\[(\R/\C)\times E\to E\]
%for example: $\alpha \cdot(a,b):=(\alpha a, \alpha b)$



\newpage
% ==> bases
\subsection{Bases}
% <==
% ==>
\subsection{Generating (span)}
% <==
% ==> independent and dependent
\subsection{Independent System}
\begin{definition}[Independent]
  The system $\left\{\vec{v_1},\vec{v_2},\vec{v_3},\dots,
  \vec{v_n}\right\}$ are 
  called ``Independent'' iff
  \[
    \lambda_1\vec{v_1}+\lambda2\vec{v_2}+\cdots+
    \lambda_n\vec{v_n}=\vec{0}
    \iff \lambda_i=0
  \]
\end{definition}
\begin{definition}[Dependent]
  The system $\left\{\vec{v_1},\vec{v_2},
  \vec{v_3},\dots,\vec{v_n}\right\}$ are 
  called ``Dependent'' iff there exits some 
  $\lambda_j\neq 0$ such that $j<n$ and 
  \[\sum_{i=1}^{n}\lambda_i\vec{v_i}=\vec{0}\]
\end{definition}
\begin{proposition}
  The system $\left\{v_1,v_2,v_3,\dots,v_n\right\}$ are 
  dependent iff there exits $j$ such that 
  $\vec{v_j}$ is the linear combination of the other vectors.
\end{proposition}
\begin{proof}
  There exits $j<n$ such that $\lambda_j\neq 0$ and 
  \begin{align*}
    &\sum_{i=1}^{n}\lambda_i\vec{v_i}=\vec{0}\\
    \implies\quad &\lambda_j\vec{v_j}+
    \sum_{i\neq j}\lambda_i\vec{v_i}=\vec{0}\\
    \implies\quad &
    \vec{v_j}=\sum_{i\neq j}\frac{-\lambda_i}{\lambda_j}\vec{v_i}
  \end{align*}
  And the converse is simple to prove.
\end{proof}
\begin{proposition}
  If the system $\left\{v_1,v_2,v_3,\dots,v_n\right\}$ are
  both 
\end{proposition}
\begin{proposition}
  Bases $\iff$ Generating + Independent
\end{proposition}
% <==








\end{document}
