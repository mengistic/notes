

\usepackage{tikz}
\usetikzlibrary{arrows}
\usepackage{enumitem}
\usepackage{mathtools}
\usepackage{etoolbox}

\usepackage{amssymb}
\usepackage{amsmath}
\usepackage{framed}
\usepackage{pstricks}
\usepackage[framed]{ntheorem}


\mathtoolsset{centercolon} 
\DeclarePairedDelimiter\abs{\lvert}{\rvert}
\DeclarePairedDelimiter\ang{\langle}{\rangle}

%% my theorem styles
\newtheoremstyle{meng-margin}%
{\item[\hskip\labelsep {\bfseries ##1\ ##2}]}%
{\item[\llap{\itshape(##3)\quad\hskip\labelsep}  {\bfseries ##1\ ##2}]}%

\newtheoremstyle{meng-ex}%
{\item[\hskip\labelsep {\normalfont\bfseries ##1\ ##2}]}%
{\item[\hskip\labelsep {\normalfont\bfseries ##1\ ##2}\ {\itshape (##3)} ]}%

\newtheoremstyle{meng-thm}%
{\item[\hskip\labelsep {\normalfont\bfseries ##1\ ##2}]}%
{\item[\hskip\labelsep {\normalfont\bfseries ##1\ ##2}\ {\itshape (##3)} ]}%



%% exercise, example, definition
\theoremstyle{meng-ex}
\theorembodyfont{\normalfont\upshape}
\theoreminframepreskip{0pt}
\theoreminframepostskip{0pt}
\newtheorem{exercise}{Exercise}
\newtheorem{example}{Example}
\newframedtheorem{definition}{Definition}

\theoremstyle{meng-thm}
\theorembodyfont{\normalfont\itshape}
\theoremprework{\bigskip\hrule\leavevmode}
\theorempostwork{\hrule\leavevmode}
\newtheorem{theorem}{Theorem}
\newtheorem{lemma}{Lemma}
\newtheorem{corollary}{Corollary}



\DeclareMathOperator{\Aut}{Aut}
\DeclareMathOperator{\rot}{rot}
\def\IM{\text{Im}}
\def\inv{^{-1}}
\def\SL{\text{SL}_2(\mathbb R)}
\def\H{\mathbb{H}}
\def\D{\mathbb{D}}
\def\R{\mathbb{R}}
\def\C{\mathbb{C}}
\renewcommand{\vec}[1]{\mathbf{#1}}
\DeclarePairedDelimiterX\norm[1]\lVert\rVert{
	\ifblank{#1}{\:\cdot\:}{#1}
}



\usepackage{enumitem}
\def\labelenumi{\textbf{(\alph*)}}

\renewcommand{\thesection}{\arabic{section}.}
\renewcommand{\theexercise}{\arabic{section}.\arabic{exercise}}

\begin{document}

\setcounter{chapter}{2}
\chapter{Determinant}

\setcounter{section}{2}
\section{Constructing the determinant}
% ==> ex4
\setcounter{exercise}{3}
\begin{exercise}
  A square $(n\times n)$ matrix is called skew-symmetric 
  (or antisymmetric) if $A\tran = -A$. Prove that if $A$ is 
  skew-symmetric and $n$ is odd, then $\det A=0$. Is it
  true for even $n$?
\end{exercise}
\begin{proof}
  When $n$ is odd, we have
  \begin{align*}
    \det A=\det A\tran=\det (-A)=(-1)^n\det A=-\det A
  \end{align*}
  This implies that $\det A=0$ whenever $n$ is odd. For even $n$,
  the determinant $\det A$ doestn't neccessarily zero, for instance
  \[ B=\begin{pmatrix} 0&-1\\1&0 \end{pmatrix} \]
  is a skew-symmetric matrix, yet $\det B=1$.
\end{proof}
% <==
% ==> ex5
\begin{exercise}
  A square matrix is called \emph{nilpoten} if $A^k=0$
  for some $k\in\N$. Show that if $A$ is nilpoten, then
  $\det A=0$.
\end{exercise}
\begin{proof}
  We use the property of determinant,
  \[ \det 0=\det A^k=(\det A)^k \]
  This shows that $\det A=0$.
\end{proof}
% <==
% ==> ex6
\begin{exercise}
  Prove that if the matrices $A$ and $B$ are similar,
  then $\det A=\det B$.
\end{exercise}
\begin{proof}
  Since $A\sim B$, hence there's invertable $Q$ such that
  $A=QBQ\inv$. Therefore
  \begin{align*}
    \det A=\det QBQ\inv
    &=(\det Q)(\det B)(\det Q\inv)\\
    &=\det(Q)(\det Q\inv)(\det B)=\det(QQ\inv)(\det B)\\
    &=\det B.
  \end{align*}
\end{proof}
% <==
% ==> ex7
\begin{exercise}
  A real square matrix $Q$ is called \emph{othogonal} if 
  $Q\tran Q=I$. Prove that if $Q$ is a n orthogonal matrix
  then $\det Q=\pm 1$.
\end{exercise}
\begin{proof}
  We have
  \[ 1=\det Q\tran Q=(\det Q\tran)(\det Q)=(\det Q)^2 \]
  this implies that $\det Q=\pm 1.$
\end{proof}
% <==

\section{Formal definition}
% ==> ex2
\setcounter{exercise}{1}
\begin{exercise}
  Let $P$ be a permutation matrix.
  \begin{itemize}
    \item Can you describe the correspoding linear
      transformation?
    \item Show that $P$ is invertable. Can you describe $P\inv$.
    \item Show that for some $N>0$, $P^N=I$.
  \end{itemize}
\end{exercise}
\begin{proof}
  Suppose that $P$ is an $n\times n$ matrix.
  \begin{itemize}
    \item The linear transformation looks like it's swapping 
      the axis in $\R^n$.
    \item If we interchange columns of $P$, we'll get the indentity
      matrix hence $\det P=\pm 1$. The direct computation shows that
      $P\inv =P\tran$.
    \item Because there are finitely many permutations, the sequence
      $\{P^n\}$ will eventually have repetitions. Hence we're sure 
      there is $i,j$ with $i>j$ such that $P^i=P^j$. Since $P$ is 
      invertable, we can multiply both sides by $P^{-j}$. Therefore
      \[P^{i-j}=I,\]
      now choose $N:=i-j$ and this completes the proof.
  \end{itemize}
\end{proof}
% <==
% ==> ex3
\begin{exercise}
  Why is there an even number of permutations of $(1,2,\dots,9)$
  and why are exactly half of them odd permutations?
\end{exercise}
\begin{proof}
  There are $9!$ permutations of $(1,2,\dots,9)$, which is an even 
  number. To prove that there are exactly half of them odd 
  (and other half even) we consider a $9\times 9$ matrix $A$ with all 
  the entries are $1$. The determinant is $\det A=0$. However,
  \[
    \det A=\sum_{\sigma\in\perm(9)}
    a_{\sigma(1),1}\cdots a_{\sigma(n),n}\sign(\sigma)
  \]
  Since all the $a_{jk}=1$, we get 
  \[\sum_{\sigma}\sign\sigma=\det A=0\]
  Because there are even number of permutations, and
  $\sign\sigma=\pm 1$ we must have half of them has $\sign=1$ and
  the other half has $\sign=-1$.
\end{proof}
% <==

\section{Cofactor}
\section{Minor and rank}

\setcounter{exercise}{1}
% ==> ex2
\begin{exercise}
  Let $A$ be an $n\times n$ matrix. How are $\det(3A),~\det(-A)$
  and $\det(A^2)$ related to $\det A$ ?
\end{exercise}
\begin{proof}
  It follows immediately that 
  \begin{align*}
    &\det(-A)=(-1)^n\det A,\\
    &\det(3A)=3^n\det A,\\
    &\det(A^2)=(\det A)^2.
  \end{align*}
\end{proof}
% <==
% ==> ex3
\begin{exercise}
  If the entries of both $A$ and $A\inv$ are integers,
  is it possible that $\det A=3$ ?
\end{exercise}
\begin{proof}
  It's impossible. If it was such a case, we'll get 
  $\det A\inv=1/\det A=1/3$. Since all the entries of $A\inv$ are
  integers, so
  \[
    \det A\inv=\sum_{\sigma}a_{\sigma(1),1}\cdots 
    a_{\sigma(n),n}\sign\sigma\in\Z
  \]
  which is a contradiction that $\det A\inv=1/3$.
\end{proof}
% <==
% ==> ex4
\begin{exercise}
  Let $\vec{v_1},\vec{v_2}$ be vectors in $\R^2$ and let $A$ be the
  $2\times 2$ matrix with columns $\vec{v_1},\vec{v_2}$. Prove that
  $\abs{\det A}$ is the area of parallelogram with two sides given by
  vectors $\vec{v_1}$ and $\vec{v_2}$.
\end{exercise}
\begin{proof}
  Let $\alpha$ be angle between $\vec{v_1}$ and the $x$-axis, and
  let $R$ be the matrix of rotation by $-\alpha$ angle. Denote
  \[
    \vec{\tilde{v}_1}:=R\vec{v_1}=(a,0)\quad\text{and}\quad
    \vec{\tilde{v}_2}:=R\vec{v_2}=(b,c).
  \]
  for some reals $a,b,c$. 
  Note that after the transformation, the area stays the same, i.e.
  $\mathrm{area}(\vec{v_1},\vec{v_2})
  =\mathrm{area}(\vec{\tilde{v}_1},\vec{\tilde{v}_2})$. It's easy 
  to see that the area of the new parallelogram is 
  \[\abs{ac}=\abs{\det (\vec{\tilde{v}_1},\vec{\tilde{v}_2})}\]
  (it has  base $\abs{a}$, and height $\abs{c}$). But
  \begin{align*}
    \det(\vec{\tilde{v}_1},\vec{\tilde{v}_2})
    &=\det(R\vec{v_1}, R\vec{v_2})=\det(RA)=\det A
  \end{align*}
  because $\det R=1$ (rotation matrix). This implies that the area of
  the parallelogram is $\abs{\det A}$.
\end{proof}
% <==
% ==> ex5
\begin{exercise}
  Let $\vec{v_1},\vec{v_2}$ be vectors in $\R^2$. Show that 
  $\det(\vec{v_1},\vec{v_2})>0$ if and only if there's a rotation
  $T_{\alpha}$ such that $T_{\alpha}\vec{v_1}$ parallel to $\vec{e_1}$
  and $T_{\alpha}\vec{v_2}$ is in the upper half-plane.
\end{exercise}
\begin{proof}
  \text{}
  \begin{itemize}
    \item[$(\Rightarrow)$] For this direction, the proof is almost 
      identical to the previous exercise. We claim that 
      $T_{\alpha}=R$ ($R$ defined in the previous exercise).
      We now wanna show that $\vec{\tilde{v}_2}=(b,c)$ is in the 
      upper-half plane, meaning $c>0$. But this immediately true 
      from the fact that $a>0$ and
      \[
        0<\det(\vec{v_1},\vec{v_2})=
        \det(\vec{\tilde{v}_1},\vec{\tilde{v}_2})=ac
      \]
    \item[$(\Leftarrow)$]
  \end{itemize}
\end{proof}
% <==





\end{document}
