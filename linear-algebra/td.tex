
% ==> preamble

\documentclass[10pt]{memoir}
% ==> math
\usepackage{amsthm}
\usepackage{amsfonts}
\usepackage{amssymb}
\usepackage{amsmath}
\usepackage{mathtools}

\mathtoolsset{centercolon} % not work when using |mathpazo|
\DeclarePairedDelimiter\abs{\lvert}{\rvert}
\DeclarePairedDelimiterX\norm[1]\lVert\rVert{
	\ifblank{#1}{\:\cdot\:}{#1}
}
\def\set#1#2{\left\{#1 ~:~ #2\right\}}
\def\permil{\text{\hskip 0.3pt\englishfont\textperthousand}}


\DeclareMathOperator{\arccot}{cot}
\DeclareMathOperator{\arcsec}{arcsec}
\DeclareMathOperator{\arccsc}{arccsc}
\DeclareMathOperator{\lcm}{lcm}
\DeclareMathOperator{\ord}{ord}
\DeclareMathOperator{\sym}{sym}


\def\N{\mathbb{N}}
\def\Z{\mathbb{Z}}
\def\Q{\mathbb{Q}}
\def\R{\mathbb{R}}
\def\C{\mathbb{C}}
\def\labelitemi{$\circ$}
\def\inv{^{-1}}
\def\ang#1{\left\langle#1\right\rangle}
\def\tran{^\mathrm{T}}
\renewcommand{\vec}[1]{\mathbf{#1}}

\theoremstyle{definition}
\newtheorem{definition}{Definition}[chapter]
\newtheorem{axiom}[definition]{Axiom}
\newtheorem{exercise}{Exercise}[section]
\newtheorem{example}[exercise]{Example}

\theoremstyle{plain}
\newtheorem{theorem}{Theorem}
\newtheorem{proposition}{Proposition}

\theoremstyle{remark}
\newtheorem{corollary}{Corollary}
\newtheorem{claim}{Claim}


% <== end: math
% ==> language
\usepackage[no-math]{fontspec}
\usepackage{mathpazo}
\setmainfont{TeX Gyre Pagella}
% <== 
% ==> setup
%\usepackage{geometry}
%\geometry{a4paper,
%  left=2.5cm, right=2.5cm,
%  top=3cm, bottom=3cm
%}
% <== 
% ==> enumerate
\usepackage{enumitem}
\usepackage{multicol}
% <==
% ==> title
\author{Sivmeng}
% <==
% ==> listings
\usepackage{xcolor}
\usepackage{listings}
% ==> basic
\lstset{%
	basicstyle=\small\ttfamily,
	keywordstyle=\color{black},
	commentstyle=\color{gray},
	keywordstyle=[1]{\color{blue!90!black}},
	keywordstyle=[2]{\color{magenta!90!black}},
	keywordstyle=[3]{\color{red!60!orange}},
	keywordstyle=[4]{\color{teal}},
	commentstyle=\color{gray},
	stringstyle=\color{green!60!black},
	tabsize=2,
	%
	numbers=left,
	numberstyle=\tiny\color{blue!70!gray},
	stepnumber=1,
	%
	frame=Lt,
	breaklines=true,
	xleftmargin=0cm,
	rulecolor=\color{gray!50!black},
	aboveskip=0.5cm,
	belowskip=0.5cm
}
% <==
% ==> code c
\lstdefinelanguage{cmeng}{
  morekeywords={
    auto,break,case,char,const,continue,default,do,double,%
    else,enum,extern,float,for,goto,if,int,long,register,return,%
    short,signed,sizeof,static,struct,switch,typedef,union,unsigned,%
    void,volatile,while},%
  morekeywords=[2]{
    printf, scanf,  include
  },
  sensitive,%
	morecomment=[l]{//},
	morecomment=[s]{/*}{*/},
	morestring=[b]',
	morestring=[b]",
}
% <==
% ==> code python
\lstdefinelanguage{py}{
	morekeywords={
		access,and,as,break,class,continue,def,del,elif,else,
		except,exec,finally,for,from,global,if,import,in,is,lambda,
		not,or,pass,print,raise,return,try,while},
	% Built-ins
	morekeywords=[2]{
		abs,all,any,basestring,bin,bool,bytearray,
		callable,chr,classmethod,cmp,compile,complex,delattr,dict,dir,
		divmod,enumerate,eval,execfile,file,filter,float,format,
		frozenset,getattr,globals,hasattr,hash,help,hex,id,input,int,
		isinstance,issubclass,iter,len,list,locals,long,map,max,
		memoryview,min,next,object,oct,open,ord,pow,property,range,
		raw_input,reduce,reload,repr,reversed,round,set,setattr,slice,
		sorted,staticmethod,str,sum,super,tuple,type,unichr,unicode,
		vars,xrange,zip,apply,buffer,coerce,intern,True,False},
	%
	morecomment=[l]\#,%
	morestring=[b]',%
	morestring=[b]",%
	morecomment=[s]{'''}{'''},% used for documentation text
	%                         % (mulitiline strings)
	morecomment=[s]{"""}{"""},% added by Philipp Matthias Hahn
	morestring=[s]{r'}{'},% `raw' strings
	morestring=[s]{r"}{"},%
	morestring=[s]{r'''}{'''},%
	morestring=[s]{r"""}{"""},%
	morestring=[s]{u'}{'},% unicode strings
	morestring=[s]{u"}{"},%
	morestring=[s]{u'''}{'''},%
	morestring=[s]{u"""}{"""},%
	%
	sensitive=true,%
}
% <==
% ==> code asy
\lstdefinelanguage{asy}{ %% Added by Sivmeng HUN
	morekeywords=[1]{
		import, for, if, else,new, do,and, access,
		from, while, break, continue, unravel, 
		operator, include, return},
	morekeywords=[2]{
		struct,typedef,static,public,readable,private,explicit,
		void,bool,int,real,string,var,picture,
		pair, path, pair3, path3, triple, transform, guide, pen, frame
	},
	morekeywords=[3]{
		true,false,and,cycle,controls,tension,atleast,
		curl,null,nullframe,nullpath,
		currentpicture,currentpen,currentprojection,
		inch,inches,cm,mm,pt,bp,up,down,right,left,
		E,N,S,W,NE,NW,SE,SW,
		solid,dashed,dashdotted,longdashed,longdashdotted,
		squarecap,roundcap,extendcap,miterjoin,roundjoin,
		beveljoin,zerowinding,evenodd,invisible
	},
	morekeywords=[4]{
		size,unitsize,draw,dot,label,
		sqrt,sin,cos,tan,cot,Sin,Cos,Tan,Cot,
		graph,
	},
	%
	morecomment=[l]{//},
	morecomment=[s]{/*}{*/},
	morestring=[b]',
	morestring=[b]",
	%
}
% <==
% <==



\renewcommand{\thesection}{\arabic{section}.}
\renewcommand{\theexercise}{\arabic{section}.\arabic{exercise}}
%\renewcommand{\thehomework}{\arabic{section}.\arabic{exercise}}
\def\by{\times}
% <==

\setlrmarginsandblock{3.5cm}{2.5cm}{*}
\setulmarginsandblock{2.5cm}{*}{1}
\checkandfixthelayout
\def\theenumi{\alph{enumi})}


\begin{document}
\chapter{Basic Notions}
% ==> Vector Spaces
\section{Vector Spaces}
% ==> ex1
\begin{exercise}
  Let $\vec{x}=(1,2,3)\tran, ~\vec{y}=(y_1,y_2,y_3)\tran$ and 
  $\vec{z}=(4,2,1)\tran$. Compute 
  $2\vec{x},~3\vec{y},~\vec{x}+2\vec{y}-3\vec{z}$.
\end{exercise}
\begin{proof}
  Little calculation reveals that
  \[
    2\vec{x}=
    \begin{pmatrix}
      2\\4\\6
    \end{pmatrix},\quad
    3\vec{y}=
    \begin{pmatrix}
      3y_1\\3y_2\\3y_3
    \end{pmatrix},\quad
    \vec{x}+2\vec{y}-3\vec{z}=
    \begin{pmatrix}
      2y_1 -11\\
      2y_2-4\\
      2y_3
    \end{pmatrix}
  \]
\end{proof}
% <==
% ==> ex2
\begin{exercise}
  Which of the following sets (with natural addition and multiplication
  by a scalar) are vector spaces? Justify your answers.
  \begin{enumerate}
    \item The set of all continuous functions on the interval $[0,1]$;
    \item The set of all non-negative functions on the interval $[0,1]$;
    \item The set of all polynomials of degree \emph{exactly} $n$;
    \item The set of all symmetric $n\times n$ matrices, i.e. 
      the set of matrices $A=\{a_{j,k}\}_{j,k=1}^{n}$ such 
      that $A\tran=A$.
  \end{enumerate}
\end{exercise}
\begin{proof}
  \text{}
  \begin{enumerate}
    \item Let $\mathcal{C}[0,1]$ be the set of all continuous functions
      on $[0,1]$. For any $f,g\in\mathcal{C}[0,1]$ and $\alpha\in\R$, we define
      \[
        (f+g)(x):=f(x)+g(x)
        \quad\text{and}\quad
        (\alpha f)(x):=\alpha\cdot f(x)
      \]
      for each $x\in[0,1]$. Therefore, $(\mathcal{C}[0,1], +,\cdot)$
      is closed under addition and scalar multiplication.
      It's immediate to see that all the eight properties of vector space
      are all satisfied, that is
      \begin{multicols}{2}
        \begin{itemize}
          \item $f+g=g+f$
          \item $f+(g+h)=(f+g)+h$
          %\item the function $0\in\mathcal{C}[0,1]$ such that 
          \item $f+0=f$
          \item $f+(-f)=0$
          %
          \item $1f=f$
          \item $\alpha(\beta f)=(\alpha\beta)f$
          \item $(\alpha+\beta)f=\alpha f+\beta f$
          \item $\alpha(f+g)=\alpha f+\beta g$
        \end{itemize}
      \end{multicols}
      Note that the function $0\in\mathcal{C}[0,1]$ such that
      $0(x)=0$ for each $x\in[0,1]$.
    \item Let $\mathcal{B}$ is the set of all non-negative functions on $[0,1]$.
      Then $(\mathcal{B},+\cdot)$ is not a vector space because it's not closed 
      under scalar multiplication, i.e. if $f\in\mathcal{B}$, hence $f>0$ yet
      \[-f=(-1)\cdot f<0.\]
      (Even if we restrict the scalar be to positive real numbers, this set 
      still won't be a vector space, because it fails to have an inverse.)
    \item Let $\mathcal{P}$ be the set of all polynomials of degree exactly $n$,
      then $(\mathcal{P}, +, \cdot)$ is \emph{not} a vector space, 
      because the addtive indentity is the polynomial 
      $0$. However, $0\notin\mathcal{P}$.
    \item Let $\sym(n)$ be the set of all symmetric $n\times n$ matrices. 
      The addition and scalar multiplication are defined as an
      \emph{entrywise} operations. Hence, $\sym(n)$ is closed under
      $(+)$ and $(\cdot)$. The additive indentity is the matrix 
      \[ \vec{0}=
        \begin{pmatrix}
          0&0&\cdots&0\\
          0&0&\cdots&0\\
          \vdots&&\ddots&\\
          0&0&\cdots&0
        \end{pmatrix}\in\sym(n).
      \]
      We could easily see that all the eight properties of vector spaces 
      are all satisfied.
  \end{enumerate}
\end{proof}
% <==
% ==> ex3
\begin{exercise}
  True or false:
  \begin{enumerate}
    \item Every vector space contains a zero vector;
      (\textbf{True.})
    \item A vector space can have more than one zero vector;
      (\textbf{False.} The zero vector is unique.)
    \item An $m\times n$ matrix has $m$ rows and $n$ columns;
      (\textbf{True.})
    \item If $f$ and $g$ are polynomials of degree $n$, then $f+g$
      is also a polynomial of degree $n$.
      (\textbf{False.} consider $t^n$ and $t-t^n$.)
    \item If $f$ and $g$ are polynomials of degree atmost $n$, the $f+g$
      is also a polynomial of degree atmost $n$.
      (\textbf{True.})
  \end{enumerate}
\end{exercise}
% <==
% ==> ex4
\begin{exercise}
  Prove that a zero vector $\vec{0}$ of a vector space $V$ is
  unique.
\end{exercise}
\begin{proof}
  Suppose that $\vec{a}$ and $\vec{b}$ are the zero vectors of $V$.
  From the \emph{Axioms of Vector Space}, we obtain that
  \begin{align*}
    \vec{a}
      &=\vec{a}+\vec{b} && \text{($\vec{b}$ is the zero vector)}\\
      &=\vec{b}+\vec{a} && \text{(commutitativity)}\\
      &=\vec{b}         && \text{($\vec{a}$ is the zero vector)}
  \end{align*}
  Hence, a zero vector of a vector space is unique,
  and we usually denote it by $\vec{0}$.
\end{proof}
% <==
% ==> ex5
\begin{exercise}
  What is the zero matrix of the space $M_{2\times 3}$?
\end{exercise}
\begin{proof}[Answer]
  In the space $M_{2\times 3}$, the zero matrix is 
  \[
    \vec{0}=
    \begin{pmatrix}
      0&0&0\\
      0&0&0
    \end{pmatrix}.
  \]
\end{proof}
% <==
% ==> ex6
\begin{exercise}
  Prove that the additive inverse of a vector space is unique.
\end{exercise}
\begin{proof}
  Let $\vec{a}$ be an arbitrary vector. Assume the $\vec{a}$ has 
  two inverses, namely $\vec{x}$ and $\vec{y}$. Hence
  \begin{align*}
    \vec{x}
      &=\vec{x}+\vec{0}   && \\
      &=\vec{x}+(\vec{a}+\vec{y}) && \text{($\vec{y}$ is an inverse)}\\
      &=(\vec{x}+\vec{a})+\vec{y} && \text{(associativity)}\\
      &=\vec{0}+\vec{y}           && \text{($\vec{x}$ is an inverse)}\\
      &=\vec{y}.
  \end{align*}
  Therefore, the inverse of any vector $\vec{a}\in V$ is unique, and 
  is usually denoted by $-\vec{a}$.
\end{proof}
% <==
% ==> ex7
\begin{exercise}
  Prove that $0\vec{v}=\vec{0}$ for any vector $\vec{v}\in V$.
\end{exercise}
\begin{proof}
  Let $\vec{v}\in V$ and  $\vec{b}$ is an inverse of $0\vec{v}$. Therefore, 
  \begin{align*}
    0
      &=0\vec{v}+b        \\
      &=(0+0)\vec{v}+b    \\
      &=(0\vec{v}+0\vec{v})+b && \text{(distributivity)}\\
      &=0\vec{v}+(0\vec{v}+b) && \text{(associativity)}\\
      &=0\vec{v}+\vec{0}      && \text{($\vec{b}$ is an inverse of $0\vec{v}$)}\\
      &=0\vec{v}
  \end{align*}
  for any $\vec{v}\in V$.
\end{proof}
% <==
% ==> ex8
\begin{exercise}
  Prove that for any vector $\vec{v}$ its additive inverse $-\vec{v}$
  is given by $(-1)\vec{v}$.
\end{exercise}
\begin{proof}
  As proved in the above exercise for any $\vec{v}\in V$, 
  \[
    \vec{0}=0\vec{v}=(1-1)\vec{v}=\vec{v}+(-1)\vec{v}
  \]
  where the last equallity derives from the distributive property.
  Because $-\vec{v}$ is the inverse of $\vec{v}$, then
  \begin{align*}
    -\vec{v}
      &=-\vec{v}+\vec{0}\\
      &=-\vec{v}+\big[\vec{v}+(-1)\vec{v}\big]\\
      &=(\underbrace{-\vec{v}+\vec{v}}_{\vec{0}})+(-1)\vec{v}\\
      &=(-1)\vec{v}
  \end{align*}
  as desired.
\end{proof}
% <==
% <==
% ==> Linear Combination
\section{Linear Combination, bases}
% ==> ex1
\begin{exercise}
  Find the basis in the space of $3\by 2$ matrices
  $M_{3\by 2}$.
\end{exercise}
\begin{proof}[Answer]
  Consider the vectors:
  \begin{align*}
    \vec{e_1}=\begin{bmatrix} 1&0\\0&0\\0&0 \end{bmatrix},\quad
    \vec{e_2}=\begin{bmatrix} 0&1\\0&0\\0&0 \end{bmatrix},\quad
    \vec{e_3}=\begin{bmatrix} 0&0\\1&0\\0&0 \end{bmatrix},\\[0.3cm]
    \vec{e_4}=\begin{bmatrix} 0&0\\0&1\\0&0 \end{bmatrix},\quad
    \vec{e_5}=\begin{bmatrix} 0&0\\0&0\\1&0 \end{bmatrix},\quad
    \vec{e_6}=\begin{bmatrix} 0&0\\0&0\\0&1 \end{bmatrix}
  \end{align*}
  and we're going to prove that the system of thses vectors are a basis.
  Any  matrix 
  \[
    \vec{v}=
    \begin{bmatrix}
      a&b\\ c&d\\ e&f
    \end{bmatrix}\in M_{3\by 2}
  \]
  can be represented as the combination
  $\vec{v}=a\vec{e_1}+b\vec{e_2}+c\vec{e_3}+d\vec{e_4}+e\vec{e_5}+f\vec{e_6}$
  thus this system is generating. Next we're going to prove the uniqueness.

  Supppose that there are $\hat{a},\hat{b},\dots,\hat{f}$ with
  \begin{align*}
    \vec{v}&=\hat{a}\vec{e_1}+\hat{b}\vec{e_2}+\hat{c}\vec{e_3}
              +\hat{d}\vec{e_4}+\hat{e}\vec{e_5}+\hat{f}\vec{e_6}\\
    \implies\quad 
    \begin{bmatrix} 
      a&b\\c&d\\e&f 
    \end{bmatrix}
    &=
    \begin{bmatrix} 
      \hat{a} & \hat{b}\\ \hat{c}&\hat{d} \\ \hat{e}&\hat{f}
    \end{bmatrix}
  \end{align*}
  This implies that each corresponding entry is equals. Hence the representation
  is unique. Therefore this system is a basis.




\end{proof}
% <==
% ==> ex2
\begin{exercise}
  True or false:
  \begin{enumerate}
    \item Any set containing a zero vector is linearly dependent;
    \item A basis must contain $\vec{0}$;
    \item subsets of linearly dependent sets are linearly dependent;
    \item subsets of linearly independent sets are linearly independent;
    \item if $\alpha_1\vec{v}_1+\alpha_2\vec{v_2}+\cdots+\alpha_n\vec{v_n}=0$
      then all scalars $\alpha_k$ are zero.
  \end{enumerate}
\end{exercise}
\begin{proof}[Answer]
  \text{}
  \begin{enumerate}
    \item \textbf{True.} because $\vec{0}$ can be represented as a linear 
      combination of the other vectors (simply put all the scalars to $0$).
    \item \textbf{No.} if so, they must be linearly dependent, which is not a base.
    \item \textbf{No.} Take for example the system of linearly dependent
      $\{\vec{e_1},\vec{e_2},\vec{e_3}\}$ where 
      $\vec{e_1}=(1,0),~\vec{e_2}=(0,1)$ and $\vec{e_3}=(1,1)$.
      The subset $\{\vec{e_1},\vec{e_2}\}$ is a basis, which is 
      clearly not linearly dependent.
    \item \textbf{True.} Supppose that the system 
      $\{\vec{v_1},\dots,\vec{v_p}\}$
      is a subset of the linearly independent system 
      $\{\vec{v_1},\dots,\vec{v_p},\dots,\vec{v_n}\}$. Let $\alpha_k$ the 
      real numbers such that
      $\alpha_{1}\vec{v_1}+\cdots+\alpha_{p}\vec{v_p}=\vec{0}$
      hence
      \[
        \alpha_{1}\vec{v_1}+\cdots+\alpha_{p}\vec{v_p}+
        0\vec{v_{p+1}}+\cdots+0\vec{v_n}=\vec{0}.
      \]
      Because the system $\{\vec{v_1},\dots,\vec{v_p},\dots,\vec{v_n}\}$
      is linearly independent, therefore all the scalars $\alpha_k=0$. 
      Thus, the system $\{\vec{v_1},\dots,\vec{v_p}\}$ is also linearly
      independent.
    \item \textbf{No.} Take, $\vec{e_1}=(2,2)$ and $\vec{e_2}=(1,1)$ for instance.
      We have $\vec{e_1}-2\vec{e_2}=\vec{0}$ yet the scalars are non-zero.
  \end{enumerate}
\end{proof}
% <==
% ==> ex3
\begin{exercise}
  Recall, that a matrix is called \emph{symmetric} if 
  $A\tran=A$. Write down a basis in the space of \emph{symmetric}
  $2\by 2$ matrices (there are many possible answers). How many
  elements are there in the basis.
\end{exercise}
\begin{proof}[Answer]
  We are going to prove that the system $\{\vec{d_1},\vec{d_2},\vec{e_1}\}$ where
  \[
    \vec{d_1}= \begin{bmatrix} 1&0\\0&0 \end{bmatrix},\quad
    \vec{d_2}= \begin{bmatrix} 0&0\\0&1 \end{bmatrix},\quad
    \vec{e_1}= \begin{bmatrix} 0&1\\1&0 \end{bmatrix},\quad
  \]
  is a basis. Observe that any symmetric matrix 
  \[
    \vec{v}=
    \begin{bmatrix}
      d_1 & e_1\\
      e_1 & d_2
    \end{bmatrix}
  \]
  can be represented as $\vec{v}=d_1\vec{d_1}+d_2\vec{d_2}+e_1\vec{e_1}$,
  hence it's generating. Note that the equation 
  \begin{align*}
    &d_1\vec{d_1}+d_2\vec{d_2}+e_1\vec{e_1}=\vec{0}\\
    &\begin{bmatrix}d_1 & e_1\\  e_1 &d_2\end{bmatrix}=\begin{bmatrix}0&0\\0&0\end{bmatrix}
  \end{align*}
  holds only when all the scalars are all zero. Hence the system is linearly 
  independent. Thus, it's a basis.

\end{proof}
% <==
% ==> ex4
\begin{exercise}
  Write down a basis for the space of
  \begin{enumerate}
    \item $3\by 3$ symmetric matrices;
    \item $n\by n$ symmetric matrices;
    \item $n\by n$ antisymmetric matrices.
  \end{enumerate}
\end{exercise}
\begin{proof}[Answer]
  \text{}
  \begin{enumerate}
    \item we are going to prove that the system of vectors
      \begin{align*}
        \vec{d_1}= \begin{bmatrix} 1&0&0 \\ 0&0&0 \\ 0&0&0 \end{bmatrix},\quad
        \vec{d_2}= \begin{bmatrix} 0&0&0 \\ 0&1&0 \\ 0&0&0 \end{bmatrix},\quad
        \vec{d_3}= \begin{bmatrix} 0&0&0 \\ 0&0&0 \\ 0&0&1 \end{bmatrix},\\[0.4cm]
        \vec{e_1}= \begin{bmatrix} 0&1&0 \\ 1&0&0 \\ 0&0&0 \end{bmatrix},\quad
        \vec{e_2}= \begin{bmatrix} 0&0&1 \\ 0&0&0 \\ 1&0&0 \end{bmatrix},\quad
        \vec{e_3}= \begin{bmatrix} 0&0&0 \\ 0&0&1 \\ 0&1&0 \end{bmatrix}.
      \end{align*}
      is the basis. First of, any symmetric matrix 
      \begin{align*}
        \vec{v}
        &=\begin{bmatrix}
          d_1 & e_1 & e_2 \\
          e_1 & d_2 & e_3 \\
          e_2 & e_3 & d_3
        \end{bmatrix}
      \intertext{can be represented as}
        \vec{v}&=d_1\vec{d_1}+d_2\vec{d_2}+d_3\vec{d_3}
          +e_1\vec{e_1}+e_2\vec{e_2}+e_3\vec{e_3}
      \end{align*}
      yeilds that the system is generating. Similar to the previous problem, 
      if the linear combination of these vectors equals $\vec{0}$, then all 
      the scalars must equals zero. Thus it's linearly independent. 
      Therefore it's a basis.
    \item Working on it.
    \item Working on it.
  \end{enumerate}
\end{proof}
% <==
% ==> ex5
\begin{exercise}
  Let a system of vectors $\vec{v}_1,\vec{v}_2,\dots, \vec{v}_r$
  be linearly independent but not generating. Show that it is possible 
  to find a vector $\vec{v}_{r+1}$ such that the system 
  $\vec{v}_1,\vec{v}_2, \dots, \vec{v}_r,\vec{v}_{r+1}$ is linearly 
  independent.
\end{exercise}
\begin{proof}
  Because the system $\vec{v}_1,\vec{v}_2,\dots, \vec{v}_r$ is not generating, 
  therefore there exists a vector $\vec{v}_{r+1}$ such that $\vec{v}_{r+1}$ 
  cannot be represented as a linear combination of $\vec{v}_1,\vec{v}_2,\dots, \vec{v}_r$.
  Let $\alpha_i$ be the scalars such that 
  \begin{equation}
    \label{eq:2:r_plus_1}
    \alpha_1\vec{v}_1+\alpha_2\vec{v}_2+\cdots+\alpha_r\vec{v}_r+\alpha_{r+1}\vec{v}_{r+1}=\vec{0}
  \end{equation}
  Now we have to prove that all the scalars are all zero.
  If $\alpha_{r+1}\neq 0$ then 
  \[
    \vec{v}_{r+1}=-\sum_{i=1}^{r}\frac{\alpha_i}{\alpha_{r+1}}\cdot\vec{v}_{i},
  \]
  meaning $\vec{v}_{r+1}$ is the linear combination of the other vectors, 
  a contradiction. Hence $\alpha_{r+1}$ must equals to zero. So
  the $r+1$ term in the equation \eqref{eq:2:r_plus_1} vanishes. And 
  because the system $\vec{v}_1,\vec{v}_2,\dots, \vec{v}_r$ is linearly independent, 
  all the scalars $\alpha_i=0$ for all $i=0,1,\dots,r$. Thus, the system
  \[
    \vec{v}_1,\vec{v}_2,\dots, \vec{v}_r, \vec{v}_{r+1}
  \]
  is also \emph{linearly independent}.
\end{proof}
% <==
% ==> ex6
\begin{exercise}
  Is it possible that vectors $\vec{v_1}, \vec{v_2}, \vec{v_3}$
  are linearly dependent, but the vectors $\vec{w_1}=\vec{v_1}+\vec{v_2}$,
  $\vec{w_2}=\vec{v_2}+\vec{v_3}$ and $\vec{w_3}=\vec{v_3}+\vec{v_1}$
  are linearly \emph{independent}.
\end{exercise}
\begin{proof}
  It's not possible, and we're going to prove this assertion via contradiction.
  Assume that there are such vectors $\vec{v}_1,\vec{v}_2,\vec{v}_3$
  satisfying the above conditions. Then there are numbers $x,y,z\in\R$
  such that
  \[
    \abs{x}+\abs{y}+\abs{z}>0
    \quad\text{and}\quad
    x\vec{v}_1+y\vec{v}_2+z\vec{v}_3=\vec{0}.
  \]
  By letting 
  \[a=x+y-z,\quad b=y+z-x,\quad c=z+x-y\]
  we obtain that
  \begin{align*}
    a\vec{w}_1+b\vec{w}_2+c\vec{w}_3
    &=(x\vec{w}_1+y\vec{w}_1-z\vec{w}_1)
     +(y\vec{w}_2+z\vec{w}_2-x\vec{w}_2)\\
    &\qquad\qquad\qquad\qquad +(x\vec{w}_3+z\vec{w}_3-y\vec{w}_3)\\
    &=2x\vec{v}_1+2y\vec{v}_2+2z\vec{v}_3\\
    &=\vec{0}.
  \end{align*}
  Since $\{\vec{w}_1,\vec{w}_2,\vec{w}_3\}$
  are linearly independent, we must have $a=b=c=0$. Hence
  \[
    \begin{cases}
      x+y-z=0\\
      y+z-x=0\\
      z+x-y=0
    \end{cases}
  \]
  adding all the 3 eqations, $x+y+z=0$. Substituting back to the 
  system of eqations above we get
  \[x=y=z=0\]
  which contradicts to the fact that $\abs{x}+\abs{y}+\abs{z}>0$.
\end{proof}
% <==
% ==> ex7
\begin{exercise}
  Any finite independent system is a subset of some basis.
\end{exercise}
\begin{proof}
  Let $\{\vec{v}_1,\vec{v}_2,\dots,\vec{v}_n\}$ is linearly independent.
  If this system is generating, then it's a base and we're done. If not,
  from exercise 2.5, there exists $\vec{v}_{n+1}$ such that
  \[\{\vec{v}_1,\vec{v}_2,\dots,\vec{v}_n,\vec{v}_{n+1}\}\]
  is still linearly independent. Now if this new system is generating, then 
  we're done. If not, we keep continue this process a finite steps, 
  adding vectors $\vec{v}_{n+1},\vec{v}_{n+2},\dots,\vec{v}_{n+r}$, and 
  eventually the new system
  \[\{\vec{v}_1,\dots,\vec{v}_{n},\vec{v}_{n+1},\dots,\vec{v}_{n+r}\}\]
  is now a basis.
\end{proof}
% <==
% <==
% ==> linear transformation
\section{Linear Transformation}
% ==> hw1
\begin{homework}
  Prove that the transformation $T:\F^n\to\F^m$ if and only if 
  $T(\alpha\vec{x}+\beta\vec{y})=\alpha T(\vec{x})+\beta T(\vec{y})$
  for any scalars $\alpha,\beta$ and vectors 
  $\vec{x},\vec{y}\in\F$.
\end{homework}
\begin{proof}
  We need to prove this in two directions.
  \begin{itemize}
    \item[($\Rightarrow$)] Suppose $T$ is a linear transformation, 
      then 
      %$T(\vec{x}+\vec{y})=T(\vec{x})+T(\vec{y})$ and 
      %$T(\alpha\vec{x})=\alpha T(\vec{\alpha})$. Thus
      \[
        T(\alpha\vec{x}+\beta\vec{y})
        =T(\alpha\vec{x})+T(\beta\vec{y})
        =\alpha T(\vec{x})+\beta T(\vec{y})
      \]
      as needed.
    \item[$(\Leftarrow)$] For this direction, we first assume that
      $T$ has the property that 
      $T(\alpha\vec{x}+\beta\vec{y}) =\alpha T(\vec{x})+\beta T(\vec{y})$
      for all $\alpha,\beta,\vec{x},\vec{y}$. We need to show that
      $T$ has the property listed in the definition of the linear 
      transformation. Observe that
      \begin{itemize}
        \item take $\alpha=\beta=1$ then,
          $T(\vec{x}+\vec{y})=T(\vec{x})+T(\vec{y})$
        \item take $\beta=0$ then,
          $T(\alpha\vec{x})=\alpha T(\vec{x})$
      \end{itemize}
      Hence $T$ is a linear transformation, 
      and the proof is completed.
  \end{itemize}
\end{proof}
% <==
% ==> hw2
\begin{homework}
  Let $T:V\to W$ be a linear transformation. Prove that
  $T(\vec{0})=\vec{0}$ and 
  \[TV=\{T\vec{v}~:~\vec{v}\in V\}\]
  is a vector space.
\end{homework}
\begin{proof}
  Since $T$ is linear, and as proved before $0\cdot\vec{0}=\vec{0}$, 
  it's easy to see that 
  $$T(\vec{0})=T(0\cdot\vec{0})=0\cdot T(\vec{0})=\vec{0}.$$
  To prove that $TV$ is a vector space, we need to check that $TV$ satisfies
  all the eight conditions listed in the definition of vector space.

  We first need to prove that $TV$ is closed.
  Because $TV\subset W$, hence $TV$ is closed under scalar multiplication and
  vector addition. Let $\vec{x},\vec{y},\vec{z}\in V$. Observe that
  \begin{itemize}
    \item $T\vec{x}+T\vec{y}=T\vec{y}+T\vec{x}$
      \quad (commutitativity of $W$)
    \item $(T\vec{x}+T\vec{y})+T\vec{z}=T\vec{x}+(T\vec{y}+T\vec{z})$
      \quad (associativity of $W$)
    \item The vector $\vec{0}\in W$ is the indentity of $TV$ because 
      \[
        T\vec{x}+\vec{0}=T\vec{x}+T\vec{0}=T(\vec{x}+\vec{0})=T(\vec{x}),\quad 
        \forall \vec{x}\in V
      \]
    \item The vector $T(-\vec{x})$ is the additive inverse of $T\vec{x}$ because
      \[T\vec{x}+T(-\vec{x})=T(\vec{x}-\vec{x})=\vec{0}\]
    \item $1\cdot T\vec{v}=T\vec{v}$
      \quad (multiplicative iden. in $W$)
  \end{itemize}
  Let $\alpha, \beta$ be scalars.
  \begin{itemize}
    \item multiplicative associativity
      \begin{align*}
        (\alpha\beta)T\vec{x} 
        &= T((\alpha\beta)\vec{x})
        && (\text{linearity of $T$})\\
        &= T(\alpha (\beta\vec{x})) 
        && (\text{mult. asso. of $V$})\\
        &= \alpha T(\beta\vec{x})
        && (\text{linearity of $T$})\\
        &= \alpha\cdot\beta T\vec{x}
      \end{align*}
    \item scalar multiplication
      \begin{align*}
        \alpha(T\vec{x}+T\vec{y})
        &=\alpha T(\vec{x}+\vec{y}) && (\text{linearity of $T$})\\
        &=T(\alpha(\vec{x}+\vec{y})) \\
        &=T(\alpha\vec{x}+\alpha\vec{y}) && (\text{scalar mult. in $V$})\\
        &=T(\alpha\vec{x})+T(\alpha\vec{x}) && (\text{linearity of $T$})\\
        &=\alpha T\vec{x}+\alpha T\vec{y}
      \end{align*}
    \item scalar multiplication
      \begin{align*}
        (\alpha+\beta)T\vec{x}
        &=T((\alpha+\beta)\vec{x}) && (\text{linearity of $T$})\\
        &=T(\alpha\vec{x}+\beta\vec{x}) && (\text{scalar mult. of $V$})\\
        &=T(\alpha\vec{x})+T(\beta\vec{x}) \\
        &=\alpha T\vec{x}+\beta T\vec{x}
      \end{align*}
  \end{itemize}
  We see that $TV$ has all eight properties to be a vector space, and the proof
  is completed.


\end{proof}
% <==
% ==> hw3
\begin{homework}
  Let $V,W$ be vector spaces. Prove that $\mathcal{L}(V,W)$, the set of all 
  linear transformations $T:V\to W$, is also a vector space.
\end{homework}
\begin{proof}
  We first need to show that $\mathcal{L}(V,W)$ is closed.
  Let $T_1, T_2\in\mathcal{L}(V,W)$ and $a$ be a scalar.
  So we need to show the transformation $T_1+T_2$ and $a T_1$ 
  are both linear. 
  \begin{itemize}
    \item Let $\vec{x}, \vec{y}$ be arbitrary vectors in $V$ 
      and $\alpha,\beta$ be scalar. 
      Denote $T:=T_1+T_2$. Observe that 
      \begin{align*}
        T(\alpha\vec{x}+\beta\vec{y})
        =&(T_1+T_2)(\alpha\vec{x}+\beta\vec{y})\\
        &=T_1(\alpha\vec{x}+\beta\vec{y})+T_2(\alpha\vec{x}+\beta\vec{y}) && (\text{by def. of $T_1+T_2$})\\
        &=\alpha T_1\vec{x}+\beta T_1\vec{y}+\alpha T_2\vec{x}+\beta T_2\vec{y} && (\text{by lin. of $T_1$ and $T_2$})\\
        &=(\alpha T_1\vec{x}+\alpha T_2\vec{x}) + (\beta T_1\vec{y}+\beta T_2\vec{y})\\
        &=\alpha (T_1\vec{x}+T_2\vec{x}) + \beta (T_1\vec{y}+T_2\vec{y}) && (\text{by scalar mult. in $W$})\\
        &=\alpha (T_1+T_2)\vec{x}+\beta (T_1+T_2)\vec{y}\\
        &=\alpha T\vec{x}+\beta T\vec{y}
      \end{align*}
      This shows that $T_1+T_2$ is also a linear transformation, hence 
      $\mathcal{L}(V,W)$ is closed under addition.
    \item Similarly, we let $\vec{x},\vec{y}\in V$. For simplicity, we again 
      denote $T:=a T_1$. Hence for any scalars $\alpha, \beta$
      \begin{align*}
        T(\alpha\vec{x}+\beta\vec{y})
        &=(aT_1)(\alpha\vec{x}+\beta\vec{y})\\
        &=a\cdot T_1(\alpha\vec{x}+\beta\vec{y}) && (\text{by def. of $aT_1$})\\
        &=a\cdot (\alpha T_1\vec{x}+\beta T_1\vec{y}) && (\text{by lin. of $T_1$})\\
        &=\alpha aT_1\vec{x}+\beta aT_1\vec{y}\\
        &=\alpha (aT_1)\vec{x}+\beta (aT_1){\vec{y}}\\
        &=\alpha T\vec{x}+\beta T\vec{y}
      \end{align*}
      This suggests that $aT_1$ is also linear, hence $\mathcal{L}(V,W)$
      is closed under scalar multiplication.
      Ultimately, we've proved that $\mathcal{L}(V,W)$ is closed as needed.
  \end{itemize}
  We are now ready to prove that $\mathcal{L}(V,W)$ is a vector space.
  Let $T_1,T_2,T_3\in\mathcal{L}(V,W)$ we have
  \begin{itemize}
    \item $T_1+T_2=T_2+T_1$, because for any $\vec{x}\in V$
      \[ (T_1+T_2)\vec{x}=T_1\vec{x}+T_2\vec{x}=T_2\vec{x}+T_1\vec{x}=(T_2+T_1)\vec{x}. \]
    \item $T_1+(T_2+T_3)=(T_1+T_2)+T_3$, because for any $\vec{x}\in V$
      \begin{align*}
        (T_1+(T_2+T_3))\vec{x}
        &=T_1\vec{x}+(T_2+T_3)\vec{x}\\
        &=T_1\vec{x}+(T_2\vec{x}+T_3\vec{x})\\
        &=(T_1\vec{x}+T_2\vec{x})+T_3\vec{x} && (\text{by asso. of $W$})\\
        &=(T_1+T_2)\vec{x}+T_3\vec{x}\\
        &=((T_1+T_2)+T_3)\vec{x}
      \end{align*}
    \item Consider the transformation $0:V\to W$ such that
      $0(\vec{v})=\vec{0}$ for all $\vec{v}\in V$. We're going to prove that
      this $0$ is the indentity of $\mathcal{L}(V,W)$. But first, we need to 
      know if $0$ is  linear or not. For any $\vec{v_1},\vec{v_2}\in V$, we have
      \[
        0(\alpha\vec{v_1}+\beta\vec{v_2})=\vec{0}
        \quad\text{and}\quad
        \alpha 0\vec{v_1}+\beta 0\vec{v_2}=\alpha\vec{0}+\beta\vec{0}=\vec{0}.
      \]
      Hence $0(\alpha\vec{v_1}+\beta\vec{v_2})=\alpha 0\vec{v_1}+\beta 0\vec{v_2}$,
      thus the transformation $0$ is linear, i.e. $0\in\mathcal{L}(V,W)$.

      Observe that for any $\vec{x}\in V$
      \[(T_1+0)\vec{x}=T_1\vec{x}+0\vec{x}=T_1\vec{x}\]
      This implies that $T_1+0=T_1$ for any $T_1\in\mathcal{L}(V,W)$.
      We conclude that $0$ is the indentity of $\mathcal{L}(V,W)$.
    \item The transformation $-T_1:=(-1)T_1$ is the additive inverse of $T_1$
      because for any $\vec{x}\in V$
      \[T_1\vec{x}+(-T_1\vec{x})=T_1\vec{x}+T_1(-\vec{x})=T_1(\vec{x}-\vec{x})=\vec{0}=0(\vec{x}).\]
    \item $1\cdot T_1=T_1$ because  $(1\cdot T_1)\vec{x}=1\cdot T_1\vec{x}=T_1\vec{x}$
      for any $\vec{x}\in V$.
    \item $(\alpha\beta)T_1=\alpha (\beta T_1)$, because 
      \[ 
        [(\alpha \beta)T_1]\vec{x}=(\alpha\beta)T_1\vec{x}=T_1(\alpha\beta\vec{x})
        =\alpha T_1(\beta\vec{x})=\alpha (\beta T_1)\vec{x}
      \]
    \item $\alpha (T_1+T_2)=\alpha T_1+\alpha T_2$ because
      \[
        [\alpha(T_1+T_2)](\vec{x})=\alpha T_1\vec{x}+\alpha T_2\vec{x}=(\alpha T_1+\alpha T_2)(\vec{x})
      \]
  \end{itemize}
\end{proof}
% <==

% ==> ex1
\begin{exercise}
  Multiply
  \begin{enumerate}
    \item 
      $
      \begin{pmatrix} 1&2&3\\4&54&6 \end{pmatrix}
      \begin{pmatrix} 1\\3\\2 \end{pmatrix} =
      \begin{pmatrix} 1+6+6\\ 4+15+12 \end{pmatrix}=
      \begin{pmatrix} 13\\31 \end{pmatrix}
      $
    \item 
      $
      \begin{pmatrix} 1&2\\0&1\\2&0 \end{pmatrix}
      \begin{pmatrix} 1\\3 \end{pmatrix}=
      \begin{pmatrix} 1+6\\ 0+3\\ 2+0 \end{pmatrix}=
      \begin{pmatrix} 7\\3\\2 \end{pmatrix}
      $
    \item 
      $
      \begin{pmatrix} 1&2&0&0\\ 0&1&2&0\\ 0&0&1&2\\ 0&0&0&1 \end{pmatrix}
      \begin{pmatrix} 1\\2\\3\\4 \end{pmatrix}=
      \begin{pmatrix} 1+4+0+0\\ 0+2+6+0\\ 0+0+3+8\\ 0+0+0+4 \end{pmatrix}=
      \begin{pmatrix} 5\\8\\11\\4 \end{pmatrix}
      $
    \item 
      $ \begin{pmatrix} 1&2&0\\0&1&2\\ 0&0&1\\0&0&0 \end{pmatrix}
      \begin{pmatrix} 1\\2\\3\\4 \end{pmatrix} $\\
      can't be multiplied because the number of columns of the first matrix
      doesn't equal to the number of rows of the second matrix.
  \end{enumerate}
\end{exercise}
% <==
% ==> ex2
\begin{exercise}
  Let a linear transformation in $\R^2$ be the reflection in the line 
  $x_1=x_2$. Find its matrix.
\end{exercise}
\begin{proof}[Solution]
  Let $T:\R^2\to\R^2$ be this transformation. The basis of the domain is
  $\{\vec{e_1}, \vec{e_2}\}$ where 
  $\vec{e_1}=(1,0)\tran$ and $\vec{e_2}=(0,1)\tran$. Because $T$ reflect
  the line $x_1=x_2$ then 
  \[
    T\vec{e_1}= \begin{pmatrix} 0\\1 \end{pmatrix}
    \quad\text{and}\quad
    T\vec{e_2}= \begin{pmatrix} 1\\0 \end{pmatrix}.
  \]
  Therefore, the matrix of this transformation is
  $[T]= \begin{bmatrix} 0&1\\1&0 \end{bmatrix}$.

\end{proof}
% <==
% ==> ex3
\begin{exercise}
  For each linear transformation below, find its matrix
  \begin{enumerate}
    \item $T:\R^2\to\R^3$ defined by $T(x,y)\tran=(x+2y,2x-5y,7y)\tran$
    \item $T:\R^4\to\R^3$ defined by 
      $T(x_1,x_2,x_3,x_4)\tran=(x_1+x_2+x_3+x_4, x_2-x_4, x_1+3x_2+6x_4)\tran$
    \item $T:\P_n\to\P_n$ st $Tf(t)=f'(t)$ 
      (find the matrix with respect to the standard basis 
      $1,t,t^2,\dots, t^n$)
    \item $T:\P_n\to\P_n$ st $Tf(t)=2f(t)+3f'(t)-4f''(t)$.
  \end{enumerate}
\end{exercise}
\begin{proof}
  Find the matrix.
  \begin{enumerate}
    \item The standard basis in $\R^2$ is $\{\vec{e_1},\vec{e_2}\}$ where
      $\vec{e_1}=(1,0)\tran$ and $\vec{e_2}=(0,1)\tran$. We have
      \[
        T\vec{e_1}= \begin{pmatrix*}[r] 1\\2\\0 \end{pmatrix*}
        \quad\text{and}\quad
        T\vec{e_2}= \begin{pmatrix*}[r] 2\\-5\\7 \end{pmatrix*}
      \]
      %Observe also that
      %\[
        %\begin{pmatrix*}[r]
          %1&2\\
          %2&-5\\
          %0&7
        %\end{pmatrix*}
        %\begin{pmatrix}
          %x\\y
        %\end{pmatrix}=
        %\begin{pmatrix}
          %x+2y\\2x-5y\\7y
        %\end{pmatrix}
      %\]
      Hence
      $
        \begin{pmatrix*}[r]
          1&2\\
          2&-5\\
          0&7
        \end{pmatrix*}
      $ is its matrix.
    \item Let $\{\vec{e_1},\vec{e_2},\vec{e_3},\vec{e_4}\}$ be the standard 
      basis in $\R^4$. Hence
      \begin{align*}
        &T\vec{e_1}=T(1,0,0,0)\tran = \begin{pmatrix} 1\\0\\1 \end{pmatrix},
        &&T\vec{e_2}=T(0,1,0,0)\tran= \begin{pmatrix} 1\\1\\3 \end{pmatrix}\\
        &T\vec{e_3}=T(0,0,1,0)\tran= \begin{pmatrix} 1 \\0\\0 \end{pmatrix},
        &&T\vec{e_4}=T(0,0,0,1)\tran= \begin{pmatrix*}[r] 1\\-1\\6 \end{pmatrix*}
      \end{align*}
      Therefore, 
      $ \begin{pmatrix*}[r]
        1&1&1&1\\
        0&1&0&-1\\
        1&3&0&6
      \end{pmatrix*} $
      is its matrix.
    \item Let $E=\{t^n,t^{n-1},\dots,t,1\}$ be the standard basis
      and $f(t)=a_nt^n+a_{n-1}t^{n-1}+\cdots+a_1t+a_0\in\P_n$. We write
      \[
        f(t)=(a_n, a_{n-1},\dots,a_1,a_0)\tran
      \]
      is base $E$. Since
      \[
        T(t^n)=nt^{n-1},\quad~T(t^{n-1})=(n-1)t^{n-2}\quad,\dots,\quad T(t)=1,\quad T(1)=0
      \]
      Therefore its matrix is 
      \[
        \begin{pmatrix*}[l]
          0&0&0&\cdots&0&0\\
          n&0&0&\cdots&0&0\\
          0&n-1&0&\cdots&0&0\\
          0&0&n-2&\cdots&0&0\\
          \vdots & \vdots & \vdots & \ddots &\vdots & \vdots\\
          0&0&0&\cdots&1&0\\
          0&0&0&\cdots&0&0
        \end{pmatrix*}
      \]
      %\begin{align*}
        %Tf(t)=f'(t)&=na_nt^{n-1}+(n-1)a_{n-1}t^{n-2}+\cdots+a_1
      %\end{align*}
      %\begin{align*}
        %&T(t^n)=nt^{n-1}= \begin{pmatrix} 0\\n\\0\\\vdots\\0\\0 \end{pmatrix},
        %&&T(t^{n-1})=(n-1)t^{n-2}= \begin{pmatrix*} 0\\0\\n-1\\\vdots\\0\\0 \end{pmatrix*},\quad\\
        %&\vdots&&\vdots\\
        %&T(t)=1= \begin{pmatrix} 0\\0\\0\\\vdots\\1\\0 \end{pmatrix},
        %&&T(1)=0= \begin{pmatrix} 0\\0\\0\\\vdots\\0\\0 \end{pmatrix}
      %\end{align*}
      \newpage
    \item $Tf(t)=2f(t)+3f'(t)-4f''(t)$\\
      Again, the standard basis is $\{t^n,t^{n-1},\dots,t,1\}$. For
      each $i\in [0,n]$ we have
      \[T(t^i)=2t^i+3it^{i-1}-4i(i-1)t^{i-2}\]
      Hence the matrix is achieved by stacking 
      $[T(t^n),\dots,T(t^i),\dots,T(t),T(1)]$, therefore the matrix is
      \[ [T]=
        \begin{bmatrix*}
          2        &0      &\cdots &0   &0\\
          3n       &2      &\cdots &0   &0\\
          -4n(n-1) &3(n-1) &\cdots &0   &0\\
          \vdots \\
          0        &0      &\cdots &2   &0\\
          0        &0      &\cdots &3   &2
        \end{bmatrix*}
      \]
  \end{enumerate}
\end{proof}
% <==
% ==> ex4
\begin{exercise}
  Find $3\by3$ matrices representing the transformations of $\R^3$ which
  \begin{enumerate}
    \item project every vector onto $x$-$y$ plane;
    \item reflect every vector through $x$-$y$ plane;
    \item rotate the $x$-$y$ plane through $30^\circ$, leaving the
      $z$-axis alone.
  \end{enumerate}
\end{exercise}
\begin{proof}
  In space $\R^3$, we shall use its standard basis 
  $\{\vec{e_1},\vec{e_2},\vec{e_3}\}$ where 
  $\vec{e_1}=(1,0,0)\tran$,
  $\vec{e_2}=(0,1,0)\tran$ and 
  $\vec{e_3}=(0,0,1)\tran$.
  \begin{enumerate}
    \item Let $T$ be this transformation. This means 
      $T(x,y,z)\tran=(x,y,0)\tran$. We get
      \[
        T\vec{e_1}= \begin{pmatrix} 1\\0\\0 \end{pmatrix},\quad
        T\vec{e_2}= \begin{pmatrix} 0\\1\\0 \end{pmatrix},\quad
        T\vec{e_3}= \begin{pmatrix} 0\\0\\0 \end{pmatrix}.
      \]
      Therefore is matrix is 
      $ \begin{pmatrix} 1&0&0\\0&1&0\\0&0&0 \end{pmatrix}. $
    \item Let $R$ be this transformation. Since $R$ project every vector
      through $x$-$y$ plane, hence $R(x,y,z)\tran=(x,y,-z)\tran$. We get
      \[
        R\vec{e_1}= \begin{pmatrix} 1\\0\\0 \end{pmatrix},\quad
        R\vec{e_2}= \begin{pmatrix} 0\\1\\0 \end{pmatrix},\quad
        R\vec{e_3}= \begin{pmatrix*}[r] 0\\0\\-1 \end{pmatrix*}.
      \]
      Thus the matrix of $R$ is 
      $
      \begin{pmatrix*}[r]
        1&0&0\\
        0&1&0\\
        0&0&-1
      \end{pmatrix*}
      $
    \item Let $S$ be this transformation. $S$ moves the vectors
      $\vec{e_1},\vec{e_2}$ to the point $x',y'$ respectively.
      %Then we have 
      %$T(x,y,z)\tran = T(x',y',z)$ where $x'=\cos 30^\circ=\frac{\sqrt{3}}{2}$
      %and $y=\sin 30^\circ=\frac{1}{2}$.
      \begin{center}
        \begin{tikzpicture}[scale=2.5, rotate=-10]
          \coordinate (O) at (0,0);
          \coordinate (x) at (0,-1);
          \coordinate (y) at (1,0);
          \coordinate (xp) at (-60:1);
          \coordinate (yp) at (30:1);


          \draw[-open triangle 60, thick] (0,0)--(0,-1) node[below]{$\vec{e_1}$};
          \draw[-open triangle 60, thick] (0,0)--(1,0) node[right]{$\vec{e_2}$};

          \draw[-open triangle 60, gray] (0,0)--(-60:1);
          \draw[-open triangle 60, gray] (0,0)--(30:1);

          \draw[dashed] ($(0,-0.86)$)--(xp) (0.5,0)--(xp);
          %\draw

          \node at ($(-60:1)+(0.1,-0.05)$) {$S\vec{e_1}$};
          \node at ($(30:1)+(0.1,0.05)$) {$S\vec{e_2}$};

          \draw pic[draw, angle eccentricity=2] {angle=x--O--xp};
          \draw pic[draw, angle eccentricity=2,"$30^\circ$"] {angle=y--O--yp};
        \end{tikzpicture}
      \end{center}
      Since $\cos30^\circ=\frac{\sqrt{3}}{2}$ and $\sin30^\circ=\frac{1}{2}$,
      we conclude that
      \[
        S\vec{e_1}= \begin{pmatrix} \frac{\sqrt{3}}{2}\\[0.2cm]\frac{1}{2}\\[0.2cm]0 \end{pmatrix},\quad
        S\vec{e_2}= \begin{pmatrix} -\frac{1}{2}\\[0.2cm]\frac{\sqrt{3}}{2}\\[0.2cm]0 \end{pmatrix},\quad
        S\vec{e_3}= \begin{pmatrix*}[r] 0\\0\\1 \end{pmatrix*}.
      \]
      Therefore the matrix is
      $ \begin{pmatrix}
        \frac{\sqrt{3}}{2} & -\frac{1}{2} & 0\\[0.2cm]
        \frac{1}{2} & \frac{\sqrt{3}}{2} & 0\\[0.2cm]
        0 & 0 &1
      \end{pmatrix}. $
  \end{enumerate}
\end{proof}
% <==
% ==> ex5
\begin{exercise}
  Let $A$ be a linear transformation. If $\vec{z}$ is the center of the
  staight interval $[\vec{x},\vec{y}]$, show that $A\vec{z}$ is the
  center of the interval $[A\vec{x}, A\vec{y}]$.
\end{exercise}
\begin{proof}
  $\vec{z}$ is the center of $[\vec{x}, \vec{y}]$ iff 
  $\vec{z}=\frac{1}{2}\vec{x}+\frac{1}{2}\vec{y}$. Therefore,
  \[
    A\vec{z}=A\left(\frac{1}{2}\vec{x}+\frac{1}{2}\vec{y}\right)
    =\frac{1}{2}A\vec{x}+\frac{1}{2}A\vec{y}
  \]
  Thus, $A\vec{z}$ is the center of the interval $[A\vec{x}, A\vec{y}]$.
\end{proof}
% <==
% ==> ex6
\begin{exercise}
  The set $\C$ of complex numbers can be canonically identified 
  with the space $\R^2$ by treating each $z=x+iy\in\C$ as a column
  $(x,y)\tran\in\R^2$.
  \begin{enumerate}
    \item Treating $\C$ as a complex vector space, show that the multiplication
      by $\alpha=a+ib\in\C$ is a linear transformation in $\C$.
      What is its matrix.
    \item Treating $\C$ as the real vector space $\R^2$ show that the multiplication 
      by $\alpha=a+ib\in\C$ is a linear transformation there.
    \item Define $T(x+iy)=2x-y+i(x-3y)$. Show that this tran is not a 
      linear transformation in the complex vector space $\C$, but
      if we treat $\C$ as the real vector space $\R^2$ then it is a linear
      transformation there, then find its matrix.
  \end{enumerate}
\end{exercise}
\begin{proof}
  \text{}
  \begin{enumerate}
    \item Let $T$ be this transformation. For any $\vec{x}\in\C$, we have
      $T\vec{x}=\alpha \vec{x}\in\C$.
      Thus $T:\C\to\C$, and we'll prove that $T$ is a linear 
      transformation. Let $\vec{x},\vec{y}\in\C$ be two vectors, 
      and $z\in\C$ be a scalar (complex). Observe that
      \begin{itemize}
        \item $T(\vec{x}+\vec{y})=\alpha(\vec{x}+\vec{y})
          =\alpha\vec{x}+\alpha\vec{y}=T\vec{x}+T\vec{y}$
          \quad (distributivity of complex numbers)
        \item $T(z\vec{x})=\alpha(z\vec{x})=z(\alpha\vec{x})=zT\vec{x}$
      \end{itemize}
      This shows that this transformation $T$ is a linear one.
      To find its matrix, we only need to know the basis of $\C$.
      Since any vector $\vec{x}\in\C$ we be written as 
      \[\vec{x}=1\cdot \underbrace{\vec{x}}_{\text{scalar}}\]
      and because this representation is unique, we obtain that 
      $\{1\}\subset\C$ is a basis of $\C$. Thus the matrix is
      \[
        [T]=[T(1)]=[\alpha\cdot 1]=[\alpha].
      \]
    \item Because we treat $\C$ as $\R^2$, then any complex number 
      $\vec{x}=x+iy$ can be represented as 
      $\begin{pmatrix} x\\y \end{pmatrix}\in\R^2$.
      Let $T$ be this transformation. Thus $T$ would look like
      \begin{align*}
        T\begin{pmatrix} x\\y \end{pmatrix}
        &=T(\vec{x})=\alpha\vec{x}\\
        &=(a+ib)(x+iy)\\
        &=(ax-by)+i(ay+bx)\\
        &=\begin{pmatrix} ax-by\\ay+bx \end{pmatrix}\in\R^2
      \end{align*}
      Thus $T:\R^2\to\R^2$.
      We need to show that $T$ is in fact linear. Let 
      $\vec{x_1}= \begin{pmatrix} x_1\\y_1 \end{pmatrix} $ and 
      $\vec{x_2}= \begin{pmatrix} x_2\\y_2 \end{pmatrix} $ and 
      be two arbitrary vectors. We have
      \[
        T\vec{x_1}+T\vec{x_2}=
        \begin{pmatrix} ax_1-by_1\\ay_1+bx_1 \end{pmatrix}+
        \begin{pmatrix} ax_2-by_2\\ay_2+bx_2 \end{pmatrix}=
        \begin{pmatrix} a(x_1+x_2)-b(y_1+y_2)\\a(y_1+y_2)+b(x_1+x_2) \end{pmatrix}=
        T(\vec{x_1}+\vec{x_2}),
      \]
      and for any scalar $r\in\R$,
      \[
        rT\vec{x}=r\begin{pmatrix} ax-by\\ay+bx \end{pmatrix}=
        \begin{pmatrix} rax-rby\\ray+rbx \end{pmatrix}=
        T(r\vec{x}).
      \]
      This shows that $T$ is a linear transformation.
      To find the matrix, we first need to find a bisis in $\R^2$.
      Luckily, as we've proved earlier we could choose $\{\vec{e_1},\vec{e_2}\}$
      to be a basis where
      \[
        \vec{e_1} \begin{pmatrix} 1\\0 \end{pmatrix}
        \quad\text{and}\quad
        \vec{e_2} \begin{pmatrix} 0\\1 \end{pmatrix}
      \]
      Therefore
      \[
        T\vec{e_1}= \begin{pmatrix} a\\b \end{pmatrix}
        \quad\text{and}\quad
        T\vec{e_2}= \begin{pmatrix} -b\\a \end{pmatrix}
      \]
      Thus the matrix of this transformation is 
      $ \begin{pmatrix} a&-b\\b&a \end{pmatrix} $.
    \item Define $T(x+iy)=2x-y+i(x-3y)$
      \begin{itemize}
        \item We'll prove that $T$ is not linear in complex vector space.
          Observe that
          \[
            T(i)=T(0+i)=-1-3i
            \quad\text{and}\quad
            T(1)=T(1+0i)=2+i
          \]
          cleary $T(i)\neq iT(1)$, this implies that $T$ is 
          not a linear transformation in $\C$.
        \item 
          In $\R^2$ the transformation would look like
          \[
            T\begin{pmatrix}x\\y\end{pmatrix}=
            \begin{pmatrix}2x-y\\x-3y\end{pmatrix}.
          \]
          For any vectors
          $\vec{x_1}=\begin{pmatrix}x_1\\y_1\end{pmatrix}$ and
          $\vec{x_2}=\begin{pmatrix}x_2\\y_2\end{pmatrix}$, we have
          \[
            T\vec{x_1}+T\vec{x_2}=
            \begin{pmatrix}2x_1-y_1\\x_1-3y_1\end{pmatrix}+
            \begin{pmatrix}2x_2-y_2\\x_2-3y_2\end{pmatrix}=
            \begin{pmatrix}2(x_1+x_2)-(y_1+y_2)\\(x_1+x_2)-3(y_1+y_2)\end{pmatrix}=
            T(\vec{x_1}+\vec{x_2})
          \]
          and for any scalar $r\in\R$,
          \[
            rT\vec{x}=r\begin{pmatrix}2x-y\\x-3y\end{pmatrix}=
            \begin{pmatrix}2rx-ry\\rx-3ry\end{pmatrix}=
            T(r\vec{x})
          \]
          this shows that $T:\R^2\to\R^2$ is linear.
          Because $\begin{pmatrix} 1\\0 \end{pmatrix}$ and 
          $\begin{pmatrix} 0\\1 \end{pmatrix}$ is a basis in $\R^2$ and
          \[
            T\begin{pmatrix} 1\\0 \end{pmatrix}=\begin{pmatrix} 2\\1 \end{pmatrix}
            \quad\text{and}\quad
            T\begin{pmatrix} 0\\1 \end{pmatrix}=\begin{pmatrix} -1\\-3 \end{pmatrix},
          \]
          thus the matrix of this transformation is 
          $ \begin{pmatrix*}[r] 2 & -1\\1&-3 \end{pmatrix*} $.
      \end{itemize}
  \end{enumerate}
\end{proof}
% <==
% ==> ex7
\begin{exercise}
  Show that any linear transformation in $\C$ 
  (treated as a complex vector space) is a multiplication by 
  $\alpha\in\C$.
\end{exercise}
\begin{proof}
  Let $T:\C\to\C$ be this transformation. For any $\vec{x}\in\C$
  \[
    T\vec{x}=T(\vec{x}\cdot \vec{1})=\vec{x}\cdot \underbrace{T(\vec{1})}_{\text{scalar}}
  \]
  and the proof is completed.
\end{proof}
% <==
% <==
% ==> linear transformation as a vector
\section{Linear transformation as a Vector}
Let set $\mathcal{L}(V,W)$ is a vector space with addition and scalar 
multiplication (as proved above).
% <==
% ==> composition
\section{Composition}

% ==> hw1
\begin{homework}
  Let $A$ and $B$ be matrices of size $m\by n$ and $n\by m$ 
  respectively. Then \[\tr(AB)=\tr(BA).\]
\end{homework}
\begin{proof}
  % ==> intro
  We'd like to prove this theorem \emph{less} computationally.
  Let $X\in M_{n\by m}$. Consider the mapping
  $T,~T_1: M_{n\by m}\to\F$ defined by
  \[
    T(X)=\tr(AX)
    \quad\text{and}\quad
    T_1(X)=\tr(XA).
  \]
  To prove the theorem it is sufficient to show that 
  $T,T_1$ are linear and they are the same.
  so by substituting $X=B$ gives the theorem.
  % <==
  % ==> claim: linearity
  \begin{claim}
    The transformations $T,T_1$ defined above are linear.
  \end{claim}
  \begin{proof}
    For $X,Y\in M_{n\by m}$,
    \begin{itemize}
      \item From the properties of matrix, $A(X+Y)=AX+AY$. 
        Because $AX$ and $BX$ are both square matrices with 
        size $m\by m$, and since we add the matrices $AX+AY$
        entrywise, it follows that
        \begin{align*}
          T(X+Y)&=\tr(A(X+Y))=\tr(AX+AY)\\
                &=\tr(AX)+\tr(AY)\\
                &=T(X)+T(Y)
        \end{align*}
      \item Similarly for any scalar $\alpha\in\F$,
        \[
          T(\alpha X)=\tr(A\cdot\alpha X)
          =\tr(\alpha AX)=\alpha\tr(AX)=\alpha T(X)
        \]
    \end{itemize}
    This implies that $T$ is a linear transformation. With simply
    proof, we conclude that $T_1$ is also a linear transformation.
  \end{proof}
  % <==
  % ==> claim: T=T_1
  We choose $\vec{e_{11}},\vec{e_{21}},\dots,\vec{e_{nm}}$ to be 
  the standard basis of $M_{n\by m}$, meaning the vector
  \[
    \vec{e_{ij}}=
    \begin{pmatrix}
      0 & 0 & \cdots & 0 & \cdots & 0\\
      \vdots  & & \cdots & & \cdots & \vdots\\
      0 & 0 & \cdots & 1 & \cdots & 0\\
      0 & 0 & \cdots & 0 & \cdots & 0
    \end{pmatrix}
  \]
  is a matrix whose entries are zero, except at
  the entry at row $i$ and column $j$, which is $1$.
  Then we only need to show that $T\vec{e_{ij}}=T_1\vec{e_{ij}}$
  for all $i,j$. Let 
  \[
    A=
    \begin{pmatrix}
      a_{11} & a_{12} & \cdots & a_{1j} & \cdots  &a_{1m}\\
      \vdots &        & \vdots &        & \vdots  &\vdots\\
      a_{i1} & a_{i2} & \cdots & a_{ij} & \cdots  &a_{im}\\
      a_{nn} & a_{n2} & \cdots & a_{nj} & \cdots  &a_{nm}
    \end{pmatrix}
  \]
  Hence 
  \begin{align*}
    A\vec{e_{ij}}&=
    \begin{pmatrix}
      \cdot & \cdot & \cdot  & \cdot \\
      \cdot & \cdot & a_{ij} & \cdot \\
      \cdot & \cdot & \cdot  & \cdot
    \end{pmatrix}
    \begin{pmatrix}
      0 & 0 & 0 \\
      0 & 0 & 0 \\
      0 & 1 & 0 \\
      0 & 0 & 0
    \end{pmatrix}=
    \begin{pmatrix}
      0     &\cdot   &\cdot \\
      \cdot &a_{ij}  &\cdot \\
      \cdot &\cdot   &0
    \end{pmatrix}
  \intertext{and}
    \vec{e_{ij}}A&=
    \begin{pmatrix}
      0 & 0 & 0 \\
      0 & 0 & 0 \\
      0 & 1 & 0 \\
      0 & 0 & 0
    \end{pmatrix}
    \begin{pmatrix}
      \cdot & \cdot & \cdot  & \cdot \\
      \cdot & \cdot & a_{ij} & \cdot \\
      \cdot & \cdot & \cdot  & \cdot
    \end{pmatrix}=
    \begin{pmatrix}
      0      &  \cdot  &  \cdot  & \cdot \\
      \cdot  &  0      &  \cdot  & \cdot \\
      \cdot  &  \cdot  &  a_{ij} & \cdot \\
      \cdot  &  \cdot  &  \cdot  & 0
    \end{pmatrix}
  \end{align*}
  This implies that $T\vec{e_{ij}}=T_1\vec{e_{ij}}$ for all $i,j$,
  and hence $T=T_1$.
  % <==
\end{proof}
% <==

% <==





\end{document}
