
\usepackage{tikz}
\usetikzlibrary{arrows}
\usepackage{enumitem}
\usepackage{mathtools}
\usepackage{etoolbox}

\usepackage{amssymb}
\usepackage{amsmath}
\usepackage{framed}
\usepackage{pstricks}
\usepackage[framed]{ntheorem}


\mathtoolsset{centercolon} 
\DeclarePairedDelimiter\abs{\lvert}{\rvert}
\DeclarePairedDelimiter\ang{\langle}{\rangle}

%% my theorem styles
\newtheoremstyle{meng-margin}%
{\item[\hskip\labelsep {\bfseries ##1\ ##2}]}%
{\item[\llap{\itshape(##3)\quad\hskip\labelsep}  {\bfseries ##1\ ##2}]}%

\newtheoremstyle{meng-ex}%
{\item[\hskip\labelsep {\normalfont\bfseries ##1\ ##2}]}%
{\item[\hskip\labelsep {\normalfont\bfseries ##1\ ##2}\ {\itshape (##3)} ]}%

\newtheoremstyle{meng-thm}%
{\item[\hskip\labelsep {\normalfont\bfseries ##1\ ##2}]}%
{\item[\hskip\labelsep {\normalfont\bfseries ##1\ ##2}\ {\itshape (##3)} ]}%



%% exercise, example, definition
\theoremstyle{meng-ex}
\theorembodyfont{\normalfont\upshape}
\theoreminframepreskip{0pt}
\theoreminframepostskip{0pt}
\newtheorem{exercise}{Exercise}
\newtheorem{example}{Example}
\newframedtheorem{definition}{Definition}

\theoremstyle{meng-thm}
\theorembodyfont{\normalfont\itshape}
\theoremprework{\bigskip\hrule\leavevmode}
\theorempostwork{\hrule\leavevmode}
\newtheorem{theorem}{Theorem}
\newtheorem{lemma}{Lemma}
\newtheorem{corollary}{Corollary}



\DeclareMathOperator{\Aut}{Aut}
\DeclareMathOperator{\rot}{rot}
\def\IM{\text{Im}}
\def\inv{^{-1}}
\def\SL{\text{SL}_2(\mathbb R)}
\def\H{\mathbb{H}}
\def\D{\mathbb{D}}
\def\R{\mathbb{R}}
\def\C{\mathbb{C}}
\renewcommand{\vec}[1]{\mathbf{#1}}
\DeclarePairedDelimiterX\norm[1]\lVert\rVert{
	\ifblank{#1}{\:\cdot\:}{#1}
}



\usepackage{enumitem}
\def\labelenumi{\textbf{(\alph*)}}

\renewcommand{\theexercise}{\arabic{section}.\arabic{exercise}}

\begin{document}
\chapter{Basic Notions}
% ==> Vector Spaces
\section{Vector Spaces}
% ==> ex1
\begin{exercise}
  Let $\vec{x}=(1,2,3)\tran, ~\vec{y}=(y_1,y_2,y_3)\tran$ and 
  $\vec{z}=(4,2,1)\tran$. Compute 
  $2\vec{x},~3\vec{y},~\vec{x}+2\vec{y}-3\vec{z}$.
\end{exercise}
\begin{proof}
  Little calculation reveals that
  \[
    2\vec{x}=
    \begin{pmatrix}
      2\\4\\6
    \end{pmatrix},\quad
    3\vec{y}=
    \begin{pmatrix}
      3y_1\\3y_2\\3y_3
    \end{pmatrix},\quad
    \vec{x}+2\vec{y}-3\vec{z}=
    \begin{pmatrix}
      2y_1 -11\\
      2y_2-4\\
      2y_3
    \end{pmatrix}
  \]
\end{proof}
% <==
% ==> ex2
\begin{exercise}
  Which of the following sets (with natural addition and multiplication
  by a scalar) are vector spaces? Justify your answers.
  \begin{enumerate}
    \item The set of all continuous functions on the interval $[0,1]$;
    \item The set of all non-negative functions on the interval $[0,1]$;
    \item The set of all polynomials of degree \emph{exactly} $n$;
    \item The set of all symmetric $n\times n$ matrices, i.e. 
      the set of matrices $A=\{a_{j,k}\}_{j,k=1}^{n}$ such 
      that $A\tran=A$.
  \end{enumerate}
\end{exercise}
\begin{proof}
  \text{}
  \begin{enumerate}
    \item Let $\mathcal{C}[0,1]$ be the set of all continuous functions
      on $[0,1]$. For any $f,g\in\mathcal{C}[0,1]$ and $\alpha\in\R$, we define
      \[
        (f+g)(x):=f(x)+g(x)
        \quad\text{and}\quad
        (\alpha f)(x):=\alpha\cdot f(x)
      \]
      for each $x\in[0,1]$. Therefore, $(\mathcal{C}[0,1], +,\cdot)$
      is closed under addition and scalar multiplication.
      It's immediate to see that all the eight properties of vector space
      are all satisfied, that is
      \begin{multicols}{2}
        \begin{itemize}
          \item $f+g=g+f$
          \item $f+(g+h)=(f+g)+h$
          %\item the function $0\in\mathcal{C}[0,1]$ such that 
          \item $f+0=f$
          \item $f+(-f)=0$
          %
          \item $1f=f$
          \item $\alpha(\beta f)=(\alpha\beta)f$
          \item $(\alpha+\beta)f=\alpha f+\beta f$
          \item $\alpha(f+g)=\alpha f+\beta g$
        \end{itemize}
      \end{multicols}
      Note that the function $0\in\mathcal{C}[0,1]$ such that
      $0(x)=0$ for each $x\in[0,1]$.
    \item Let $\mathcal{B}$ is the set of all non-negative functions on $[0,1]$.
      Then $(\mathcal{B},+\cdot)$ is not a vector space because it's not closed 
      under scalar multiplication, i.e. if $f\in\mathcal{B}$, hence $f>0$ yet
      \[-f=(-1)\cdot f<0.\]
      (Even if we restrict the scalar be to positive real numbers, this set 
      still won't be a vector space, because it fails to have an inverse.)
    \item Let $\mathcal{P}$ be the set of all polynomials of degree exactly $n$,
      then $(\mathcal{P}, +, \cdot)$ is \emph{not} a vector space, 
      because the addtive indentity is the polynomial 
      $0$. However, $0\notin\mathcal{P}$.
    \item Let $\sym(n)$ be the set of all symmetric $n\times n$ matrices. 
      The addition and scalar multiplication are defined as an
      \emph{entrywise} operations. Hence, $\sym(n)$ is closed under
      $(+)$ and $(\cdot)$. The additive indentity is the matrix 
      \[ \vec{0}=
        \begin{pmatrix}
          0&0&\cdots&0\\
          0&0&\cdots&0\\
          \vdots&&\ddots&\\
          0&0&\cdots&0
        \end{pmatrix}\in\sym(n).
      \]
      We could easily see that all the eight properties of vector spaces 
      are all satisfied.
  \end{enumerate}
\end{proof}
% <==
% ==> ex3
\begin{exercise}
  True or false:
  \begin{enumerate}
    \item Every vector space contains a zero vector;
      (\textbf{True.})
    \item A vector space can have more than one zero vector;
      (\textbf{False.} The zero vector is unique.)
    \item An $m\times n$ matrix has $m$ rows and $n$ columns;
      (\textbf{True.})
    \item If $f$ and $g$ are polynomials of degree $n$, then $f+g$
      is also a polynomial of degree $n$.
      (\textbf{False.} consider $t^n$ and $t-t^n$.)
    \item If $f$ and $g$ are polynomials of degree atmost $n$, the $f+g$
      is also a polynomial of degree atmost $n$.
      (\textbf{True.})
  \end{enumerate}
\end{exercise}
% <==
% ==> ex4
\begin{exercise}
  Prove that a zero vector $\vec{0}$ of a vector space $V$ is
  unique.
\end{exercise}
\begin{proof}
  Suppose that $\vec{a}$ and $\vec{b}$ are the zero vectors of $V$.
  From the \emph{Axioms of Vector Space}, we obtain that
  \begin{align*}
    \vec{a}
      &=\vec{a}+\vec{b} && \text{($\vec{b}$ is the zero vector)}\\
      &=\vec{b}+\vec{a} && \text{(commutitativity)}\\
      &=\vec{b}         && \text{($\vec{a}$ is the zero vector)}
  \end{align*}
  Hence, a zero vector of a vector space is unique,
  and we usually denote it by $\vec{0}$.
\end{proof}
% <==
% ==> ex5
\begin{exercise}
  What is the zero matrix of the space $M_{2\times 3}$?
\end{exercise}
\begin{proof}[Answer]
  In the space $M_{2\times 3}$, the zero matrix is 
  \[
    \vec{0}=
    \begin{pmatrix}
      0&0&0\\
      0&0&0
    \end{pmatrix}.
  \]
\end{proof}
% <==
% ==> ex6
\begin{exercise}
  Prove that the additive inverse of a vector space is unique.
\end{exercise}
\begin{proof}
  Let $\vec{a}$ be an arbitrary vector. Assume the $\vec{a}$ has 
  two inverses, namely $\vec{x}$ and $\vec{y}$. Hence
  \begin{align*}
    \vec{x}
      &=\vec{x}+\vec{0}   && \\
      &=\vec{x}+(\vec{a}+\vec{y}) && \text{($\vec{y}$ is an inverse)}\\
      &=(\vec{x}+\vec{a})+\vec{y} && \text{(associativity)}\\
      &=\vec{0}+\vec{y}           && \text{($\vec{x}$ is an inverse)}\\
      &=\vec{y}.
  \end{align*}
  Therefore, the inverse of any vector $\vec{a}\in V$ is unique, and 
  is usually denoted by $-\vec{a}$.
\end{proof}
% <==
% ==> ex7
\begin{exercise}
  Prove that $0\vec{v}=\vec{0}$ for any vector $\vec{v}\in V$.
\end{exercise}
\begin{proof}
  Let $\vec{v}\in V$ and  $\vec{b}$ is an inverse of $0\vec{v}$. Therefore, 
  \begin{align*}
    0
      &=0\vec{v}+b        \\
      &=(0+0)\vec{v}+b    \\
      &=(0\vec{v}+0\vec{v})+b && \text{(distributivity)}\\
      &=0\vec{v}+(0\vec{v}+b) && \text{(associativity)}\\
      &=0\vec{v}+\vec{0}      && \text{($\vec{b}$ is an inverse of $0\vec{v}$)}\\
      &=0\vec{v}
  \end{align*}
  for any $\vec{v}\in V$.
\end{proof}
% <==
% ==> ex8
\begin{exercise}
  Prove that for any vector $\vec{v}$ its additive inverse $-\vec{v}$
  is given by $(-1)\vec{v}$.
\end{exercise}
\begin{proof}
  As proved in the above exercise for any $\vec{v}\in V$, 
  \[
    \vec{0}=0\vec{v}=(1-1)\vec{v}=\vec{v}+(-1)\vec{v}
  \]
  where the last equallity derives from the distributive property.
  Because $-\vec{v}$ is the inverse of $\vec{v}$, then
  \begin{align*}
    -\vec{v}
      &=-\vec{v}+\vec{0}\\
      &=-\vec{v}+\big[\vec{v}+(-1)\vec{v}\big]\\
      &=(\underbrace{-\vec{v}+\vec{v}}_{\vec{0}})+(-1)\vec{v}\\
      &=(-1)\vec{v}
  \end{align*}
  as desired.
\end{proof}
% <==

% <==

\end{document}
