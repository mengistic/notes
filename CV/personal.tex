\documentclass{article}



\usepackage[no-math]{fontspec}
\setsansfont{Khmer OS Siemreap}[
	BoldFont = Khmer OS Bokor,
%	ItalicFont = Khmer OS Metal Chrieng,
	AutoFakeSlant=0.15,
	BoldItalicFont = Khmer OS Muol,
	Script=Khmer,Scale=1
]
\usepackage[Khmer, Latin]{ucharclasses}
\usepackage{etoolbox}
\XeTeXlinebreaklocale "khm"
\XeTeXlinebreakskip = 0pt plus 1pt minus 1pt

\newfontfamily{\khmerfamily}
[
	BoldFont = Khmer OS Bokor,
	ItalicFont = Khmer OS Metal Chrieng,
	BoldItalicFont = Khmer OS Muol,
	SmallCapsFont = TeX Gyre Pagella,
	Script=Khmer,Scale=1
]{Khmer OS Siemreap}
\newfontfamily{\englishfamily}{Fira Sans}[Ligatures=TeX, Scale=1.1] 
%Simonetta, Zeyada, Caveat, 
\newfontfamily{\monofamily}[Scale=1.25]{Latin Modern Mono}
\newfontfamily{\latinfamily}[Scale=1.3]{Latin Modern Roman}
% \newfontfamily{\tacteingfamily}[Scale=3]{Tacteing}

% Define new font family (have to)   <-- BUG
\newrobustcmd{\englishfont}{\englishfamily\let\currentenglish\englishfamily }
\newrobustcmd{\khmerfont}{\khmerfamily\let\currentkhmer\khmerfamily}
\newrobustcmd{\mono}{\monofamily\let\currentenglish\monofamily }
\newrobustcmd{\en}{\latinfamily\let\currentenglish\latinfamily}
% \newrobustcmd{\tacteing}{\tacteingfamily\let\currentenglish\tacteingfamily}



%% initialize the font
\khmerfont\englishfont
\setTransitionsForLatin{\currentenglish}{\currentkhmer}

% Change typewriter font family
\renewcommand{\ttfamily}{\mono}


\usepackage{bbold}
%\usepackage{fourier}
\let\altmathbb\mathbb
\AtBeginDocument{\let\mathbb\altmathbb}

% change math fonts
\let\temp\rmdefault
\usepackage{mathpazo}
\let\rmdefault\temp



\newcommand{\kml}
{
	\fontspec[
		Script=Khmer, Scale=1,
		AutoFakeBold=1, AutoFakeSlant=0.25
	] {Khmer OS Muol Light}
	\selectfont
}

\newcommand{\km}
{
	\fontspec[
		Script=Khmer, Scale=1,
		AutoFakeBold=1, AutoFakeSlant=0.25
	] {Khmer OS Muol}
	\selectfont
}

\newcommand{\kpali}
{
	\fontspec[
		Script=Khmer, Scale=1,
		AutoFakeBold=1, AutoFakeSlant=0.25
	] {Khmer OS Muol Pali}
	\selectfont
}

%create khmer counter
%khmer number
\makeatletter
\def\khmer#1{\expandafter\@khmer\csname c@#1\endcsname}
\def\@khmer#1{\expandafter\@@khmer\number#1\@nil}
\def\@@khmer#1{%
	\ifx#1\@nil% terminate when encounter @nil
	\else%
	\ifcase#1 ០\or ១\or ២\or ៣\or ៤\or ៥\or ៦\or ៧\or ៨\or ៩\fi%
	\expandafter\@@khmer% recursively map the following characters
	 \fi}
	 
% khmer alphabet
\def\khmernumeral#1{\@@khmer#1\@nil}
\def\alpkh#1{\expandafter\@alpkh\csname c@#1\endcsname}
\def\@alpkh#1{%
	\ifcase#1% zero -> none
	\or ក\or ខ\or គ\or ឃ\or ង%
	\or ច\or ឆ\or ជ\or ឈ\or ញ%
	\or ដ\or ឋ\or ឌ\or ឍ\or ណ%
	\or ត\or ថ\or ទ\or ធ\or ន%
	\or ប\or ផ\or ព\or ភ\or ម%
	\or យ\or រ\or ល\or វ\or ស%
	\or ហ\or ឡ\or អ%
	\else%[most]
	\@ctrerr % otherwise, counter error!
	\fi}
\makeatother


%% Khmer & math
\def\KHstop{\quad\text{។}}
\def\KHand{\quad\text{និង}\quad}
\def\KHor{\quad\text{ឬ}\quad}

\def\chaptername{ជំពូកទី}

\pagestyle{empty}
\usepackage[a4paper, margin=2.5cm]{geometry}
\linespread{1.5}



\begin{document}


\begin{center}
  {\Large\bfseries Personal Statement}
\end{center}

My name is Sivmeng HUN. I was born in Prey Veng, a province
located about 90 km away from the capital of Cambodia.
Even though the education there was quite limited, my parents 
still value education and they sent me to a public primary school
near our village. They taught me how to read and write the Khmer alphabet,
and I was quite good at it. However, Mathematics is what I struggled
with the most. No matter how hard my parents and my teachers tried, 
I could hardly do a simple addition. When I was in grade 2, I remembered 
myself asking a friend of mine to do all the problems in my Maths test 
for me. It was very embarrassing, yet it was also my most remembered 
memory in my childhood.

As I grew older, my interest in my mathematics also grew stronger.
And the reason for which I was quite fond of mathematics is not that
I liked maths, but it's because I thought being good at maths gives me
better grades. This motivated me to set myself up and was ready to 
learn Maths for whatever it takes.
However, this time my parents couldn't help me because
they were not familiar with many lessons in the text books that was
constantly being updated. 
Thus, to answer the questions in the text books
I would ask the elder students in the highschool near my house,
and they would give me the answers.
This had developed me an important soft skill, which is bravery
and confidence. 

At the age of twelve, I entered highschool. There, I already had built
up a strong interest in mathematics. I was curious and I would want to
understand the lessons from the books. However, it gets more and more
challenging, yet this hadn't stopped me there. I kept asking the elders 
to explain me the points I don't understand, and I really appreciated
their help. Some problems took me weeks to understand the solutions,
and some took several months.
And as a result, not only that I got a better understanding of the 
subject, but I also learnt to be patient.

During the last year at highschool, I had to take an important 
National Exam (BACC II) in August. Back then, I had to prepared for 
other subjects that were given in the test as well. Unexpectedly, 
things had gone off differently. There was Covid-19 pandemic spread 
all over the world drastically. And the Cambodian government had made a 
decision to cancel BACC II exam. Not many students like this decision. 
It was here that I realised that getting good grades 
wasn't at all very important. The most important thing 
in my opinion, though, is accepting the truth, and be flexible at 
any circumstances.

2021 is when I entered college, and I decided to pursue Mathematics 
as my main major. There were many things that I hadn't expected to face.
Some courses were tough and challenging, not to mention all the classes
I took were all online classes. Despite all of these, I was lucky enough 
to join a class taught by a senior student, and he had guided me to
get through those obstacles. Some of his advises work, and some do not.
Yet I really appreciated his effort he put in the course. 
This has  taught me that to overcome obstacles in life, we need some help
from others, even it didn't help much but it's very meaningful to have
others stand by us. %And I also learn to practise empathy to others.

From my childhood experiences, I realised that the education system
especially for those provinces around the countryside aren't quite
sophisticated. Many students are having great difficulties especially
with mathematics. I also noticed that there aren't much math competitions
for highschool students as well. So, my main purpose is help building 
a solid background to students in Cambodia, especially those who are in 
their highschool year. Then I would make a collaboration
project for a formal Maths Olympiad competition in Cambodia.
To make this small dream into action, I plan to graduate and
continue to pursue my maths major as a master student or, if possible,
as a PHD student.
The journey I've been through so far had taught me not only the hard
skill in school, but also many other important soft skills that one 
should have. 
Based on my hard/soft skills and my future plan, I'm applying
for \textit{Mitsubishi UFJ Foundation}
because not only that it'll help my financial aid, but it's also 
motivates to keep me going through this journey.
I would like to thanks \textit{Mitsubishi UFJ Foundation} for
providing us this great opportunity.





\end{document}
