% ==> Preamble
% https://www.youtube.com/playlist?list=PL2SOU6wwxB0uwwH80KTQ6ht66KWxbzTIo

\usepackage{tikz}
\usetikzlibrary{arrows}
\usepackage{enumitem}
\usepackage{mathtools}
\usepackage{etoolbox}

\usepackage{amssymb}
\usepackage{amsmath}
\usepackage{framed}
\usepackage{pstricks}
\usepackage[framed]{ntheorem}


\mathtoolsset{centercolon} 
\DeclarePairedDelimiter\abs{\lvert}{\rvert}
\DeclarePairedDelimiter\ang{\langle}{\rangle}

%% my theorem styles
\newtheoremstyle{meng-margin}%
{\item[\hskip\labelsep {\bfseries ##1\ ##2}]}%
{\item[\llap{\itshape(##3)\quad\hskip\labelsep}  {\bfseries ##1\ ##2}]}%

\newtheoremstyle{meng-ex}%
{\item[\hskip\labelsep {\normalfont\bfseries ##1\ ##2}]}%
{\item[\hskip\labelsep {\normalfont\bfseries ##1\ ##2}\ {\itshape (##3)} ]}%

\newtheoremstyle{meng-thm}%
{\item[\hskip\labelsep {\normalfont\bfseries ##1\ ##2}]}%
{\item[\hskip\labelsep {\normalfont\bfseries ##1\ ##2}\ {\itshape (##3)} ]}%



%% exercise, example, definition
\theoremstyle{meng-ex}
\theorembodyfont{\normalfont\upshape}
\theoreminframepreskip{0pt}
\theoreminframepostskip{0pt}
\newtheorem{exercise}{Exercise}
\newtheorem{example}{Example}
\newframedtheorem{definition}{Definition}

\theoremstyle{meng-thm}
\theorembodyfont{\normalfont\itshape}
\theoremprework{\bigskip\hrule\leavevmode}
\theorempostwork{\hrule\leavevmode}
\newtheorem{theorem}{Theorem}
\newtheorem{lemma}{Lemma}
\newtheorem{corollary}{Corollary}



\DeclareMathOperator{\Aut}{Aut}
\DeclareMathOperator{\rot}{rot}
\def\IM{\text{Im}}
\def\inv{^{-1}}
\def\SL{\text{SL}_2(\mathbb R)}
\def\H{\mathbb{H}}
\def\D{\mathbb{D}}
\def\R{\mathbb{R}}
\def\C{\mathbb{C}}
\renewcommand{\vec}[1]{\mathbf{#1}}
\DeclarePairedDelimiterX\norm[1]\lVert\rVert{
	\ifblank{#1}{\:\cdot\:}{#1}
}



\usepackage{enumitem}
\def\labelenumi{\textbf{(\alph*)}}
\title{Notes on Statistics}
% <==

\begin{document}
\maketitle

\chapter{Propobility}
\section{Prerequisite}
Statistics applied to 
\begin{itemize}
  \item Goverment: IQSS
  \item life
  \item Finance
  \item Gambling: origin of probability, i.e Fermat and Pascal (1650)
\end{itemize}

\begin{definition}
  These are some definitions.
  \begin{itemize}
    \item A \emph{Sample Space} is the set of all possible outcomes of an experiment.
    \item An \emph{Event} is the subset of the sample space.
    \item The probability of event $A$, denoted as 
      $$P(A)=\frac{\text{\#want}}{\text{\#possibility}}$$
  \end{itemize}
\end{definition}
Here, we assume that all outcomes are equally likely, and finite
sample space. What's the probability that there is life in Neptune?

\subsection{Couting}
\begin{theorem}[Multiplication Rule]
  If there is an experiment with $n_1$ possible outcomes, and 
  the second experiment has $n_2$ possible outcomes, and so on.
  Then the overall outcomes are
  \[n_1n_2n_3\dots n_r.\]
\end{theorem}
\begin{theorem}
  The numbers of way to choose $r$ items from the total of 
  $n$ items is
  \[\binom{n}{r}:=\frac{n!}{(n-k)!k!}\]
\end{theorem}





\end{document}
