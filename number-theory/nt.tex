
\usepackage{tikz}
\usetikzlibrary{arrows}
\usepackage{enumitem}
\usepackage{mathtools}
\usepackage{etoolbox}

\usepackage{amssymb}
\usepackage{amsmath}
\usepackage{framed}
\usepackage{pstricks}
\usepackage[framed]{ntheorem}


\mathtoolsset{centercolon} 
\DeclarePairedDelimiter\abs{\lvert}{\rvert}
\DeclarePairedDelimiter\ang{\langle}{\rangle}

%% my theorem styles
\newtheoremstyle{meng-margin}%
{\item[\hskip\labelsep {\bfseries ##1\ ##2}]}%
{\item[\llap{\itshape(##3)\quad\hskip\labelsep}  {\bfseries ##1\ ##2}]}%

\newtheoremstyle{meng-ex}%
{\item[\hskip\labelsep {\normalfont\bfseries ##1\ ##2}]}%
{\item[\hskip\labelsep {\normalfont\bfseries ##1\ ##2}\ {\itshape (##3)} ]}%

\newtheoremstyle{meng-thm}%
{\item[\hskip\labelsep {\normalfont\bfseries ##1\ ##2}]}%
{\item[\hskip\labelsep {\normalfont\bfseries ##1\ ##2}\ {\itshape (##3)} ]}%



%% exercise, example, definition
\theoremstyle{meng-ex}
\theorembodyfont{\normalfont\upshape}
\theoreminframepreskip{0pt}
\theoreminframepostskip{0pt}
\newtheorem{exercise}{Exercise}
\newtheorem{example}{Example}
\newframedtheorem{definition}{Definition}

\theoremstyle{meng-thm}
\theorembodyfont{\normalfont\itshape}
\theoremprework{\bigskip\hrule\leavevmode}
\theorempostwork{\hrule\leavevmode}
\newtheorem{theorem}{Theorem}
\newtheorem{lemma}{Lemma}
\newtheorem{corollary}{Corollary}



\DeclareMathOperator{\Aut}{Aut}
\DeclareMathOperator{\rot}{rot}
\def\IM{\text{Im}}
\def\inv{^{-1}}
\def\SL{\text{SL}_2(\mathbb R)}
\def\H{\mathbb{H}}
\def\D{\mathbb{D}}
\def\R{\mathbb{R}}
\def\C{\mathbb{C}}
\renewcommand{\vec}[1]{\mathbf{#1}}
\DeclarePairedDelimiterX\norm[1]\lVert\rVert{
	\ifblank{#1}{\:\cdot\:}{#1}
}



\usepackage{enumitem}
\def\labelenumi{\textbf{(\alph*)}}



\title{Number Theory}
\author{Lectured by Meng T. Heang}

\renewcommand{\bar}[1]{\overline{#1}}

\begin{document}
\maketitle


\chapter{Quadratic Ring}

\section{Some definitions}
The motivation of this comes from the fact that solving 
equations like $x^n+y^n=z^n$ is rather hard to solve. For example, 
case $n=2$, the equation becomes
\[z^2=x^2+y^2=(x+\i y)(x-\i y)\]
so we need a language to say that $(x+\i y)$ is a divisor 
of $z^2$, I guess. And here comes the 

\begin{definition}
  For any square-free (not a perfect square) integer $d$,
  we define
  \[\Q[\sqrt{d}]:=\{x+y\sqrt{d}~:~ x,y\in\Q\}.\]
\end{definition}
In $\Q[\sqrt d]$, we define the usual addition and multiplication
as follow: for $a+b\sqrt d,~x+y\sqrt d\in\Q[\sqrt d]$, then
\[ (a+b\sqrt d)+(x+y\sqrt d)=(a+x)+(b+y)\sqrt d \]
and
\[ (a+b\sqrt d)(x+y\sqrt d)=(ax+byd)+(ay+bx)\sqrt d. \]
As we can see that $\Z[\sqrt d]$ is indeed a ring.
But it's more than that. In fact, it's a field. To see why,
let $a+b\sqrt d\neq 0\in\Z[\sqrt d]$, and we choose
$x=\frac{a}{a^2-db^2}$ and $y=\frac{-b}{a^2-db^2}$. We then have
\[(a+b\sqrt d)(x+y\sqrt d)=1.\]

\begin{example}
  For special case of $d$:
  \begin{itemize}
    \item When $d=2$, we have $\Q[\sqrt d]=\Q[\sqrt 2]=\{a+b\sqrt{2}~:~a,b\in\Q\}$.
    \item 
      When $d=-1$, we then have $\Q[\sqrt d]=\Q[\sqrt{-1}]$. In is field,
      a special subset is considered, the subring $\Z[\i]\subset \Q[\i]$,
      which is called \emph{the set of Gaussian Integers}.
  \end{itemize}
\end{example}

We can consider $\Q[\sqrt d]$ as a vector space over the field $\Q$.
Moreover we can setup an isomophism $\Q[\sqrt d]$ to $\Q^2$ by sending
$a+b\sqrt q$ to $(a,b)$.


\section{Norm and Trace}
Let $L=x+y\sqrt d\in\Q[\sqrt d]$. We define the
\begin{itemize}
  \item conjugate $\bar{L}:=x=y\sqrt d$
  \item trace $T(L)=\tr(L):=L+\bar{L}=2x$
  \item norm $N(L):=L\cdot\bar{L}=x^2-y^2d$
\end{itemize}
\begin{proposition}
  Let $L,~L_1,~L_2\in\Q[\sqrt d]$. The the following are true:
  \begin{itemize}
    \item $\bar{L_1+L_2}=\bar{L_1}+\bar{L_2}$
    \item $\bar{L_1L_2}=\bar{L_1}\bar{L_2}$
    \item $N(L_1L_2)=N(L_1)N(L_2)$
    \item $N(L)=0\iff L=0$.
  \end{itemize}
\end{proposition}
\begin{proof}
  The first three properties can be derived from direct computation
  of conjugate and norm. We only give the proof for the last one.

  If $L=0$, the clearly $N(L)=0$. Now we assume that $N(L)=0$, and
  try to prove that $L=0$. Let $L=x+y\sqrt d$. Thus
  \[0=N(L)=x^2-y^2\sqrt d\]
  (to be continued ... )
\end{proof}

\newpage
\section{Algebraic Integers}
\begin{definition}
  The number $L\in\Q[\sqrt d]$ is said to be an 
  \emph{algebraic integer} if $T(L)\in\Z$
  and $N(L)\in\Z$.
\end{definition}
What are the algebraic numbers in $\Q[\sqrt d]$ ?
The theorem below illustrates this
\begin{theorem}
  The set of all algebraic, aka \emph{quadratic ring}, of
  $\Q[\sqrt d]$ is
  \[
  \begin{cases}
    \Z[\sqrt d] &\text{if } d\equiv 2,3 \pmod{4}\\
    \Z\left[\frac{1+\sqrt d}{2}\right] &\text{if }d\equiv 1\pmod{4}
  \end{cases}
  \]
\end{theorem}

\begin{definition}
  An algebraic integer $L$ is said to be a \emph{unit}
  if its inverse $L\inv$ is also an algebraic integer.
\end{definition}

\begin{proposition}
  Let $L$ be an algebraic integer. Then 
  $L$ is a unit if and only if $N(1)=\pm 1$.
\end{proposition}

Yeah, I know it's boring to have ``theorem, definition'' style, 
but let's keep it that way.

\begin{definition}
  Let $R$ be a quadratic ring and let $a,b\in R$.
  \begin{itemize}
    \item We say $b$ \emph{divides} $a$ if $\exists r\in R$ such that $bx=a$.
    \item 
  \end{itemize}
\end{definition}








\end{document}
