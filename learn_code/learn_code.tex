
\usepackage{tikz}
\usetikzlibrary{arrows}
\usepackage{enumitem}
\usepackage{mathtools}
\usepackage{etoolbox}

\usepackage{amssymb}
\usepackage{amsmath}
\usepackage{framed}
\usepackage{pstricks}
\usepackage[framed]{ntheorem}


\mathtoolsset{centercolon} 
\DeclarePairedDelimiter\abs{\lvert}{\rvert}
\DeclarePairedDelimiter\ang{\langle}{\rangle}

%% my theorem styles
\newtheoremstyle{meng-margin}%
{\item[\hskip\labelsep {\bfseries ##1\ ##2}]}%
{\item[\llap{\itshape(##3)\quad\hskip\labelsep}  {\bfseries ##1\ ##2}]}%

\newtheoremstyle{meng-ex}%
{\item[\hskip\labelsep {\normalfont\bfseries ##1\ ##2}]}%
{\item[\hskip\labelsep {\normalfont\bfseries ##1\ ##2}\ {\itshape (##3)} ]}%

\newtheoremstyle{meng-thm}%
{\item[\hskip\labelsep {\normalfont\bfseries ##1\ ##2}]}%
{\item[\hskip\labelsep {\normalfont\bfseries ##1\ ##2}\ {\itshape (##3)} ]}%



%% exercise, example, definition
\theoremstyle{meng-ex}
\theorembodyfont{\normalfont\upshape}
\theoreminframepreskip{0pt}
\theoreminframepostskip{0pt}
\newtheorem{exercise}{Exercise}
\newtheorem{example}{Example}
\newframedtheorem{definition}{Definition}

\theoremstyle{meng-thm}
\theorembodyfont{\normalfont\itshape}
\theoremprework{\bigskip\hrule\leavevmode}
\theorempostwork{\hrule\leavevmode}
\newtheorem{theorem}{Theorem}
\newtheorem{lemma}{Lemma}
\newtheorem{corollary}{Corollary}



\DeclareMathOperator{\Aut}{Aut}
\DeclareMathOperator{\rot}{rot}
\def\IM{\text{Im}}
\def\inv{^{-1}}
\def\SL{\text{SL}_2(\mathbb R)}
\def\H{\mathbb{H}}
\def\D{\mathbb{D}}
\def\R{\mathbb{R}}
\def\C{\mathbb{C}}
\renewcommand{\vec}[1]{\mathbf{#1}}
\DeclarePairedDelimiterX\norm[1]\lVert\rVert{
	\ifblank{#1}{\:\cdot\:}{#1}
}



\usepackage{enumitem}
\def\labelenumi{\textbf{(\alph*)}}

\begin{document}

% ==> Getting Started
\chapter{C Program}
% ==> download
\section{How to download}
Just like other programming languages, in order to write a C program you
need to have a text editor and a compiler. For me, I use \verb|vim| 
(well, \verb|neovim| to be precise) as my main editor.
\footnote{I am also an \texttt{emacs} user.}
Installing a compiler is straight forwward. On Arch Linux, 
sudo pacman -S gcc
\begin{verbatim}
$ sudo pacman -S gcc neovim
\end{verbatim}
% <==
% ==> Data type
\section{Data types}
There are many data types in C programming.
\begin{center}
  \begin{tabular}{ll}
    \toprule
    data type      & wft is this\\
    \cmidrule(r){1-1} \cmidrule(r){2-2}
    \verb|int|     &   \verb|%d|\\
    \verb|float|   &   \verb|%f|\\
    \verb|char[]|  &   \verb|%s|\\
    \bottomrule
  \end{tabular}
\end{center}
% <==
\newpage
\section{Some examples}
\begin{lstlisting}[language=cmeng]
#include <stdio.h>

void whatEver(){
  float x, y;

  printf("Enter your first number: "); scanf("%f", &x);
  printf("Enter your second number: "); scanf("%f", &y);

  printf("\n------------------\n");
  printf("%f + %f=%f\n", x, y, x+y);
  printf("%f + %f=%f\n", x, y, x-y);
  printf("%f + %f=%f\n", x, y, x*y);
  printf("%f / %f=%f", x, y, x/y);
  printf("\n------------------\n");
}

int main(){

  whatEver();
  return 0;
}

\end{lstlisting}
% <==


\end{document}
