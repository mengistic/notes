
\documentclass[10pt]{memoir}
% ==> math
\usepackage{amsthm}
\usepackage{amsfonts}
\usepackage{amssymb}
\usepackage{amsmath}
\usepackage{mathtools}

\mathtoolsset{centercolon} % not work when using |mathpazo|
\DeclarePairedDelimiter\abs{\lvert}{\rvert}
\DeclarePairedDelimiterX\norm[1]\lVert\rVert{
	\ifblank{#1}{\:\cdot\:}{#1}
}
\def\set#1#2{\left\{#1 ~:~ #2\right\}}
\def\permil{\text{\hskip 0.3pt\englishfont\textperthousand}}


\DeclareMathOperator{\arccot}{cot}
\DeclareMathOperator{\arcsec}{arcsec}
\DeclareMathOperator{\arccsc}{arccsc}
\DeclareMathOperator{\lcm}{lcm}
\DeclareMathOperator{\ord}{ord}
\DeclareMathOperator{\sym}{sym}


\def\N{\mathbb{N}}
\def\Z{\mathbb{Z}}
\def\Q{\mathbb{Q}}
\def\R{\mathbb{R}}
\def\C{\mathbb{C}}
\def\labelitemi{$\circ$}
\def\inv{^{-1}}
\def\ang#1{\left\langle#1\right\rangle}
\def\tran{^\mathrm{T}}
\renewcommand{\vec}[1]{\mathbf{#1}}

\theoremstyle{definition}
\newtheorem{definition}{Definition}[chapter]
\newtheorem{axiom}[definition]{Axiom}
\newtheorem{exercise}{Exercise}[section]
\newtheorem{example}[exercise]{Example}

\theoremstyle{plain}
\newtheorem{theorem}{Theorem}
\newtheorem{proposition}{Proposition}

\theoremstyle{remark}
\newtheorem{corollary}{Corollary}
\newtheorem{claim}{Claim}


% <== end: math
% ==> language
\usepackage[no-math]{fontspec}
\usepackage{mathpazo}
\setmainfont{TeX Gyre Pagella}
% <== 
% ==> setup
%\usepackage{geometry}
%\geometry{a4paper,
%  left=2.5cm, right=2.5cm,
%  top=3cm, bottom=3cm
%}
% <== 
% ==> enumerate
\usepackage{enumitem}
\usepackage{multicol}
% <==
% ==> title
\author{Sivmeng}
% <==
% ==> listings
\usepackage{xcolor}
\usepackage{listings}
% ==> basic
\lstset{%
	basicstyle=\small\ttfamily,
	keywordstyle=\color{black},
	commentstyle=\color{gray},
	keywordstyle=[1]{\color{blue!90!black}},
	keywordstyle=[2]{\color{magenta!90!black}},
	keywordstyle=[3]{\color{red!60!orange}},
	keywordstyle=[4]{\color{teal}},
	commentstyle=\color{gray},
	stringstyle=\color{green!60!black},
	tabsize=2,
	%
	numbers=left,
	numberstyle=\tiny\color{blue!70!gray},
	stepnumber=1,
	%
	frame=Lt,
	breaklines=true,
	xleftmargin=0cm,
	rulecolor=\color{gray!50!black},
	aboveskip=0.5cm,
	belowskip=0.5cm
}
% <==
% ==> code c
\lstdefinelanguage{cmeng}{
  morekeywords={
    auto,break,case,char,const,continue,default,do,double,%
    else,enum,extern,float,for,goto,if,int,long,register,return,%
    short,signed,sizeof,static,struct,switch,typedef,union,unsigned,%
    void,volatile,while},%
  morekeywords=[2]{
    printf, scanf,  include
  },
  sensitive,%
	morecomment=[l]{//},
	morecomment=[s]{/*}{*/},
	morestring=[b]',
	morestring=[b]",
}
% <==
% ==> code python
\lstdefinelanguage{py}{
	morekeywords={
		access,and,as,break,class,continue,def,del,elif,else,
		except,exec,finally,for,from,global,if,import,in,is,lambda,
		not,or,pass,print,raise,return,try,while},
	% Built-ins
	morekeywords=[2]{
		abs,all,any,basestring,bin,bool,bytearray,
		callable,chr,classmethod,cmp,compile,complex,delattr,dict,dir,
		divmod,enumerate,eval,execfile,file,filter,float,format,
		frozenset,getattr,globals,hasattr,hash,help,hex,id,input,int,
		isinstance,issubclass,iter,len,list,locals,long,map,max,
		memoryview,min,next,object,oct,open,ord,pow,property,range,
		raw_input,reduce,reload,repr,reversed,round,set,setattr,slice,
		sorted,staticmethod,str,sum,super,tuple,type,unichr,unicode,
		vars,xrange,zip,apply,buffer,coerce,intern,True,False},
	%
	morecomment=[l]\#,%
	morestring=[b]',%
	morestring=[b]",%
	morecomment=[s]{'''}{'''},% used for documentation text
	%                         % (mulitiline strings)
	morecomment=[s]{"""}{"""},% added by Philipp Matthias Hahn
	morestring=[s]{r'}{'},% `raw' strings
	morestring=[s]{r"}{"},%
	morestring=[s]{r'''}{'''},%
	morestring=[s]{r"""}{"""},%
	morestring=[s]{u'}{'},% unicode strings
	morestring=[s]{u"}{"},%
	morestring=[s]{u'''}{'''},%
	morestring=[s]{u"""}{"""},%
	%
	sensitive=true,%
}
% <==
% ==> code asy
\lstdefinelanguage{asy}{ %% Added by Sivmeng HUN
	morekeywords=[1]{
		import, for, if, else,new, do,and, access,
		from, while, break, continue, unravel, 
		operator, include, return},
	morekeywords=[2]{
		struct,typedef,static,public,readable,private,explicit,
		void,bool,int,real,string,var,picture,
		pair, path, pair3, path3, triple, transform, guide, pen, frame
	},
	morekeywords=[3]{
		true,false,and,cycle,controls,tension,atleast,
		curl,null,nullframe,nullpath,
		currentpicture,currentpen,currentprojection,
		inch,inches,cm,mm,pt,bp,up,down,right,left,
		E,N,S,W,NE,NW,SE,SW,
		solid,dashed,dashdotted,longdashed,longdashdotted,
		squarecap,roundcap,extendcap,miterjoin,roundjoin,
		beveljoin,zerowinding,evenodd,invisible
	},
	morekeywords=[4]{
		size,unitsize,draw,dot,label,
		sqrt,sin,cos,tan,cot,Sin,Cos,Tan,Cot,
		graph,
	},
	%
	morecomment=[l]{//},
	morecomment=[s]{/*}{*/},
	morestring=[b]',
	morestring=[b]",
	%
}
% <==
% <==



% ==> playlist
% - Gross:
% https://www.youtube.com/playlist?list=PLelIK3uylPMGzHBuR3hLMHrYfMqWWsmx5
% <==


\title{Notes on Abstract Algebra}

\begin{document}
% ==> Gross
\maketitle
\chapter{Benedict Gross}

\section{Preliminaries}
\begin{itemize}
  \item We have group
    $GL_n(\R)$ set of all invertable $n\times n$ matrixes with
    entries in $\R$
  \item same thing with $GL_n(\C), GL_n(\Q)$
\end{itemize}
\begin{definition}[Group]
  All group $G$ is 
  \begin{itemize}
    \item  a set with a product structure, that is if 
      $a,b\in G$ then $a\cdot b\in G$
    \item associative: $a(bc)=(ab)c$
    \item exits indentiy $e$ such that $ae=ea=a$
    \item exits inverse $a\inv$  such that
      $aa\inv=a\inv a=e$
  \end{itemize}
\end{definition}

\section{Ur group}
The $\sym(T)$ set of all bijections $a:T\to T$. Deine 
$a\cdot b(t)=a(b(t))$
\begin{itemize}
  \item  bijection = ormomophism
  \item  $GL_n(\R)\subset \sym(\R^n)$
\end{itemize}
\begin{definition}
  sub group $H\subset G$ closed under $\cdot$ 
  contain indentiy, and closed.
\end{definition}
\begin{itemize}
  \item  Let $S_n=\sym\left\{1,2,3,...,n\right\}$ permution 
    group of $n$ letter. This is finite group of order
    $n$. So $\abs{S_n}=n!$
  \item $S_1(1)=\left\{e\right\}$
  \item $S_2=\left\{e, \tau\right\}$ where 
    $e\tau=\tau e=\tau$ and $\tau\tau=e$
  \item $S_3=\left\{e, \tau', \tau'', \sigma,
    \sigma'\right\}$
\end{itemize}
\begin{definition}
  If $ab=ba$ for all pairs, then 
  $G$ is Abelian group.
\end{definition}
\begin{corollary}
  The group $S_n$ is non-abelion for all 
  $n\geq 3$.
\end{corollary}
\begin{proof}
  Since $S_3\subset S_n$. Fixing the letters
  $\left\{4,5,6,...,n\right\}$. But $S_3$ 
  not non-abelion, then so is $S_n$.
\end{proof}
\begin{claim}
  For $k\leq n$ then $S_k\subset S_n$.
\end{claim}
What is a subgroup of $GL_2(\R)$ which 
stablized the line. The matrix
\[
  \begin{pmatrix}
    a&c\\
    0&d
  \end{pmatrix}
\]
where $ad\neq 0$.
$y=0$.

\begin{proposition}
  The subgroup of $(\Z,+)$ are precisely 
  given by $(b\Z,+)$. 
\end{proposition}
\begin{proof}
  First, there are all subgroups.
  Let $H\subset\Z$. If $H=\left\{0\right\}$ 
  then here $b=0$.

  Now suppose that $H\neq\{0\}$.  so contains $m\neq 0$.
  Let $b>0$ be the smallest positive 
  integer contained in $H$. Then $H\supset b\Z$.
  Suppose that $h\in H$ and write $h=mb+r$ with 
  $0\leq r<b$. We cliam that $r=0$. 
  \[h+(-m)b=r\]
  somehow $r=0$
\end{proof}

For any group $G$, $g\in G$, Denote $H=\ang{g}$
be cyclic subgruop generated by $G$
smallest subgroup contains $g$. If $g^m=e$ and 
$m$ is the smallest power, we say that $m$ is the order of
$g\in G$. If such $m$ not exits, we say
$g$ has infinite order.
% <==















\end{document}
