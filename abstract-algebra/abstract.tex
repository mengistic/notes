
\usepackage{tikz}
\usetikzlibrary{arrows}
\usepackage{enumitem}
\usepackage{mathtools}
\usepackage{etoolbox}

\usepackage{amssymb}
\usepackage{amsmath}
\usepackage{framed}
\usepackage{pstricks}
\usepackage[framed]{ntheorem}


\mathtoolsset{centercolon} 
\DeclarePairedDelimiter\abs{\lvert}{\rvert}
\DeclarePairedDelimiter\ang{\langle}{\rangle}

%% my theorem styles
\newtheoremstyle{meng-margin}%
{\item[\hskip\labelsep {\bfseries ##1\ ##2}]}%
{\item[\llap{\itshape(##3)\quad\hskip\labelsep}  {\bfseries ##1\ ##2}]}%

\newtheoremstyle{meng-ex}%
{\item[\hskip\labelsep {\normalfont\bfseries ##1\ ##2}]}%
{\item[\hskip\labelsep {\normalfont\bfseries ##1\ ##2}\ {\itshape (##3)} ]}%

\newtheoremstyle{meng-thm}%
{\item[\hskip\labelsep {\normalfont\bfseries ##1\ ##2}]}%
{\item[\hskip\labelsep {\normalfont\bfseries ##1\ ##2}\ {\itshape (##3)} ]}%



%% exercise, example, definition
\theoremstyle{meng-ex}
\theorembodyfont{\normalfont\upshape}
\theoreminframepreskip{0pt}
\theoreminframepostskip{0pt}
\newtheorem{exercise}{Exercise}
\newtheorem{example}{Example}
\newframedtheorem{definition}{Definition}

\theoremstyle{meng-thm}
\theorembodyfont{\normalfont\itshape}
\theoremprework{\bigskip\hrule\leavevmode}
\theorempostwork{\hrule\leavevmode}
\newtheorem{theorem}{Theorem}
\newtheorem{lemma}{Lemma}
\newtheorem{corollary}{Corollary}



\DeclareMathOperator{\Aut}{Aut}
\DeclareMathOperator{\rot}{rot}
\def\IM{\text{Im}}
\def\inv{^{-1}}
\def\SL{\text{SL}_2(\mathbb R)}
\def\H{\mathbb{H}}
\def\D{\mathbb{D}}
\def\R{\mathbb{R}}
\def\C{\mathbb{C}}
\renewcommand{\vec}[1]{\mathbf{#1}}
\DeclarePairedDelimiterX\norm[1]\lVert\rVert{
	\ifblank{#1}{\:\cdot\:}{#1}
}



\usepackage{enumitem}
\def\labelenumi{\textbf{(\alph*)}}

% ==> playlist
% - Gross:
% https://www.youtube.com/playlist?list=PLelIK3uylPMGzHBuR3hLMHrYfMqWWsmx5
% <==


\title{Notes on Abstract Algebra}

\begin{document}
% ==> Gross
\maketitle
\chapter{Benedict Gross}

\section{Preliminaries}
\begin{itemize}
  \item We have group
    $GL_n(\R)$ set of all invertable $n\times n$ matrixes with
    entries in $\R$
  \item same thing with $GL_n(\C), GL_n(\Q)$
\end{itemize}
\begin{definition}[Group]
  All group $G$ is 
  \begin{itemize}
    \item  a set with a product structure, that is if 
      $a,b\in G$ then $a\cdot b\in G$
    \item associative: $a(bc)=(ab)c$
    \item exits indentiy $e$ such that $ae=ea=a$
    \item exits inverse $a\inv$  such that
      $aa\inv=a\inv a=e$
  \end{itemize}
\end{definition}

\section{Ur group}
The $\sym(T)$ set of all bijections $a:T\to T$. Deine 
$a\cdot b(t)=a(b(t))$
\begin{itemize}
  \item  bijection = ormomophism
  \item  $GL_n(\R)\subset \sym(\R^n)$
\end{itemize}
\begin{definition}
  sub group $H\subset G$ closed under $\cdot$ 
  contain indentiy, and closed.
\end{definition}
\begin{itemize}
  \item  Let $S_n=\sym\left\{1,2,3,...,n\right\}$ permution 
    group of $n$ letter. This is finite group of order
    $n$. So $\abs{S_n}=n!$
  \item $S_1(1)=\left\{e\right\}$
  \item $S_2=\left\{e, \tau\right\}$ where 
    $e\tau=\tau e=\tau$ and $\tau\tau=e$
  \item $S_3=\left\{e, \tau', \tau'', \sigma,
    \sigma'\right\}$
\end{itemize}
\begin{definition}
  If $ab=ba$ for all pairs, then 
  $G$ is Abelian group.
\end{definition}
\begin{corollary}
  The group $S_n$ is non-abelion for all 
  $n\geq 3$.
\end{corollary}
\begin{proof}
  Since $S_3\subset S_n$. Fixing the letters
  $\left\{4,5,6,...,n\right\}$. But $S_3$ 
  not non-abelion, then so is $S_n$.
\end{proof}
\begin{claim}
  For $k\leq n$ then $S_k\subset S_n$.
\end{claim}
What is a subgroup of $GL_2(\R)$ which 
stablized the line. The matrix
\[
  \begin{pmatrix}
    a&c\\
    0&d
  \end{pmatrix}
\]
where $ad\neq 0$.
$y=0$.

\begin{proposition}
  The subgroup of $(\Z,+)$ are precisely 
  given by $(b\Z,+)$. 
\end{proposition}
\begin{proof}
  First, there are all subgroups.
  Let $H\subset\Z$. If $H=\left\{0\right\}$ 
  then here $b=0$.

  Now suppose that $H\neq\{0\}$.  so contains $m\neq 0$.
  Let $b>0$ be the smallest positive 
  integer contained in $H$. Then $H\supset b\Z$.
  Suppose that $h\in H$ and write $h=mb+r$ with 
  $0\leq r<b$. We cliam that $r=0$. 
  \[h+(-m)b=r\]
  somehow $r=0$
\end{proof}

For any group $G$, $g\in G$, Denote $H=\ang{g}$
be cyclic subgruop generated by $G$
smallest subgroup contains $g$. If $g^m=e$ and 
$m$ is the smallest power, we say that $m$ is the order of
$g\in G$. If such $m$ not exits, we say
$g$ has infinite order.
% <==















\end{document}
