
\documentclass[10pt]{memoir}
% ==> math
\usepackage{amsthm}
\usepackage{amsfonts}
\usepackage{amssymb}
\usepackage{amsmath}
\usepackage{mathtools}

\mathtoolsset{centercolon} % not work when using |mathpazo|
\DeclarePairedDelimiter\abs{\lvert}{\rvert}
\DeclarePairedDelimiterX\norm[1]\lVert\rVert{
	\ifblank{#1}{\:\cdot\:}{#1}
}
\def\set#1#2{\left\{#1 ~:~ #2\right\}}
\def\permil{\text{\hskip 0.3pt\englishfont\textperthousand}}


\DeclareMathOperator{\arccot}{cot}
\DeclareMathOperator{\arcsec}{arcsec}
\DeclareMathOperator{\arccsc}{arccsc}
\DeclareMathOperator{\lcm}{lcm}
\DeclareMathOperator{\ord}{ord}
\DeclareMathOperator{\sym}{sym}


\def\N{\mathbb{N}}
\def\Z{\mathbb{Z}}
\def\Q{\mathbb{Q}}
\def\R{\mathbb{R}}
\def\C{\mathbb{C}}
\def\labelitemi{$\circ$}
\def\inv{^{-1}}
\def\ang#1{\left\langle#1\right\rangle}
\def\tran{^\mathrm{T}}
\renewcommand{\vec}[1]{\mathbf{#1}}

\theoremstyle{definition}
\newtheorem{definition}{Definition}[chapter]
\newtheorem{axiom}[definition]{Axiom}
\newtheorem{exercise}{Exercise}[section]
\newtheorem{example}[exercise]{Example}

\theoremstyle{plain}
\newtheorem{theorem}{Theorem}
\newtheorem{proposition}{Proposition}

\theoremstyle{remark}
\newtheorem{corollary}{Corollary}
\newtheorem{claim}{Claim}


% <== end: math
% ==> language
\usepackage[no-math]{fontspec}
\usepackage{mathpazo}
\setmainfont{TeX Gyre Pagella}
% <== 
% ==> setup
%\usepackage{geometry}
%\geometry{a4paper,
%  left=2.5cm, right=2.5cm,
%  top=3cm, bottom=3cm
%}
% <== 
% ==> enumerate
\usepackage{enumitem}
\usepackage{multicol}
% <==
% ==> title
\author{Sivmeng}
% <==
% ==> listings
\usepackage{xcolor}
\usepackage{listings}
% ==> basic
\lstset{%
	basicstyle=\small\ttfamily,
	keywordstyle=\color{black},
	commentstyle=\color{gray},
	keywordstyle=[1]{\color{blue!90!black}},
	keywordstyle=[2]{\color{magenta!90!black}},
	keywordstyle=[3]{\color{red!60!orange}},
	keywordstyle=[4]{\color{teal}},
	commentstyle=\color{gray},
	stringstyle=\color{green!60!black},
	tabsize=2,
	%
	numbers=left,
	numberstyle=\tiny\color{blue!70!gray},
	stepnumber=1,
	%
	frame=Lt,
	breaklines=true,
	xleftmargin=0cm,
	rulecolor=\color{gray!50!black},
	aboveskip=0.5cm,
	belowskip=0.5cm
}
% <==
% ==> code c
\lstdefinelanguage{cmeng}{
  morekeywords={
    auto,break,case,char,const,continue,default,do,double,%
    else,enum,extern,float,for,goto,if,int,long,register,return,%
    short,signed,sizeof,static,struct,switch,typedef,union,unsigned,%
    void,volatile,while},%
  morekeywords=[2]{
    printf, scanf,  include
  },
  sensitive,%
	morecomment=[l]{//},
	morecomment=[s]{/*}{*/},
	morestring=[b]',
	morestring=[b]",
}
% <==
% ==> code python
\lstdefinelanguage{py}{
	morekeywords={
		access,and,as,break,class,continue,def,del,elif,else,
		except,exec,finally,for,from,global,if,import,in,is,lambda,
		not,or,pass,print,raise,return,try,while},
	% Built-ins
	morekeywords=[2]{
		abs,all,any,basestring,bin,bool,bytearray,
		callable,chr,classmethod,cmp,compile,complex,delattr,dict,dir,
		divmod,enumerate,eval,execfile,file,filter,float,format,
		frozenset,getattr,globals,hasattr,hash,help,hex,id,input,int,
		isinstance,issubclass,iter,len,list,locals,long,map,max,
		memoryview,min,next,object,oct,open,ord,pow,property,range,
		raw_input,reduce,reload,repr,reversed,round,set,setattr,slice,
		sorted,staticmethod,str,sum,super,tuple,type,unichr,unicode,
		vars,xrange,zip,apply,buffer,coerce,intern,True,False},
	%
	morecomment=[l]\#,%
	morestring=[b]',%
	morestring=[b]",%
	morecomment=[s]{'''}{'''},% used for documentation text
	%                         % (mulitiline strings)
	morecomment=[s]{"""}{"""},% added by Philipp Matthias Hahn
	morestring=[s]{r'}{'},% `raw' strings
	morestring=[s]{r"}{"},%
	morestring=[s]{r'''}{'''},%
	morestring=[s]{r"""}{"""},%
	morestring=[s]{u'}{'},% unicode strings
	morestring=[s]{u"}{"},%
	morestring=[s]{u'''}{'''},%
	morestring=[s]{u"""}{"""},%
	%
	sensitive=true,%
}
% <==
% ==> code asy
\lstdefinelanguage{asy}{ %% Added by Sivmeng HUN
	morekeywords=[1]{
		import, for, if, else,new, do,and, access,
		from, while, break, continue, unravel, 
		operator, include, return},
	morekeywords=[2]{
		struct,typedef,static,public,readable,private,explicit,
		void,bool,int,real,string,var,picture,
		pair, path, pair3, path3, triple, transform, guide, pen, frame
	},
	morekeywords=[3]{
		true,false,and,cycle,controls,tension,atleast,
		curl,null,nullframe,nullpath,
		currentpicture,currentpen,currentprojection,
		inch,inches,cm,mm,pt,bp,up,down,right,left,
		E,N,S,W,NE,NW,SE,SW,
		solid,dashed,dashdotted,longdashed,longdashdotted,
		squarecap,roundcap,extendcap,miterjoin,roundjoin,
		beveljoin,zerowinding,evenodd,invisible
	},
	morekeywords=[4]{
		size,unitsize,draw,dot,label,
		sqrt,sin,cos,tan,cot,Sin,Cos,Tan,Cot,
		graph,
	},
	%
	morecomment=[l]{//},
	morecomment=[s]{/*}{*/},
	morestring=[b]',
	morestring=[b]",
	%
}
% <==
% <==



\title{Vector Calculus}
\author{Mentoring Program}

\begin{document}

\chapter{Introduction}
\section{Complex Number}


\begin{theorem}
  The polynomial with complex-valued coefficient
  \[p(z)=z^n+a_{n-1}z^{n-1}+\cdots+a_1z+a_0\]
  has at root in $\C$.
\end{theorem}
\begin{proof}
  We break into two parts; firstly we show that $\abs{p(z)}$
  has a minimum, that is there exists a $z_0\in\C$ such that
  $\abs{p(z_0)}\leq\abs{p(z)}$ for any $z\in\C$.
  Next, we show that $z_0$ is indeed the root of $p$ by arguing
  that the case when $\abs{p(z_0)}\neq 0$ is not possible, this
  forces $\abs{p(z_0)}=0$ and hence $p(z_0)=0$.

  \vspace{0.3cm}
  \begin{claim}
    There is an $R>0$ such that $\abs{p(z)}\geq\abs{a_0}$ for any 
    $z\in\C\setminus D_R(0)$.
  \end{claim}
  \begin{proof}[Proof:]
    First, let's denote $A=\max\{\abs{a_{n-1}}, \dots, \abs{a_0}\}$,
    and we choose
    \[R:=\max\{1,~ A(n+1)\}.\]
    Thus for any $\abs{z}>R$, we have
    \begin{align*}
      \abs{p(z)}
      &=\abs{z^n+a_{n-1}z^{n-1}+\cdots+a_1z+a_0}\\
      &\geq \abs{z^n}-\abs{a_{n-1}z^{n-1}+\cdots+a_1z+a_0}
      &&(\text{Triangle inequality})\\
      &\geq \abs{z}^n-(\abs{a_{n-1}}\abs{z^{n-1}}+\cdots+\abs{a_1}z+\abs{a_0})\\
      &\geq \abs{z}^n-M(\abs{z}^{n-1}+\cdots+\abs{z}+1)\\
      &\geq R^n-M(R^{n-1}+\cdots+R+1)
      &&(\text{becuse }\abs{z}>R)\\
      &\geq R^n-M(\underbrace{R^{n-1}+\cdots+R^{n-1}+R^{n-1}}_{n})
      &&(\text{because }R>1)\\
      &=R^{n-1}\cdot(R-Mn)\\
      &\geq 1\cdot(M(n+1)-Mn)=M>\abs{a_0}
    \end{align*}
    Thus the claim is proved.
  \end{proof}
  This claim helps us to say that $\abs{p}$ has a minimum. To see why,
  observe that $D_R(0)$ is compact, and because $\abs{p}$ is 
  continues, then $\{\abs{p(z)}~:~ z\in D_R(0)\}$ is also compact. Thus
  it contains its minimum, say $z_0$,  that is 
  $\abs{p(z_0)}\leq\abs{p(z')}$ for any $z'\in D_R(0)$.
  But $0\in D_R(0)$, hence $\abs{p(z_0)}\leq\abs{p(0)}=\abs{a_0}$.
  Now for abitrary $z\notin D_R(0)$, that is $\abs{z}>R$, we have 
  $\abs{p(z)}\geq\abs{a_0}$ by the above claim. Hence
  $\abs{p(z_0)}\leq\abs{a_0}\leq\abs{p(z)}$.

  This shows that $\abs p$ has a global minimum value at $z_0\in\C$,
  i.e. $\abs{p(z_0)}\leq\abs{p(z)}$ for any $z\in\C$.

  \vspace{0.3cm}
  \begin{claim}
    For that minimum value $z_0$, $\abs{p(z_0)}=0$.
  \end{claim}
  \begin{proof}[Proof:]
    Assume on the contrary that $\abs{p(z_0)}\neq 0$.
    Let $z=z_0+u$ for some $u\in\C$, then
    \begin{align*}
      q(u)
      &:=p(z_0+u)\\
      &=(z_0+u)^n+a_{n-1}(z_0+u)^{n-1}+\cdots+a_1(z_0+u)+a_0\\
      &=u^n+b_{n-1}u^{n-1}+\cdots+b_1u+b_0
    \end{align*}
    where $b_0=z_0^n+a_{n-1}z_0^{n-1}+\cdots+a_1z_0+a_0=p(z_0)$.
    Let $j>0$ the the smallest index such that $b_j\neq 0$.
    Thus we can rewrite 
    \begin{align*}
      q(u)
      &=b_0+b_ju^j+(b_{j+1}u^{j+1}+\cdots+u^n)\\
      &=:b_0+b_ju^j+\mathcal O(u)
    \end{align*}
    Now let's suppose that $b_0=r_0\e^{\i\theta}$.
    Our goal is to find $u=r\e^{\i\phi}$ such that 
    $\abs{q(u)}<\abs{p(z_0)}$. If we could such such $u$,
    then we would derive to a contradiction.
    We want to choose $r$ and $\phi$ such that
    \[
      \begin{cases}
        b_ju^j \parallel b_0
        \abs{b_ju^j} < \abs{b_0};\\
        \abs{\mathcal O(u)}<\abs{b_ju^j}
      \end{cases}
    \]
    % ==> asy fig
    \begin{center}
    \begin{asy}
import geometry;
unitsize(1.5cm);
defaultpen(fontsize(10pt));


real theta=35;
real r0=5;
real rw=0.6*r0;

pair b0 = r0*dir(theta);
pair w = rw*dir(theta);

real rwq=0.55*(r0-rw);
pair q=w+rwq*dir(140);
pair q_=w+rwq*dir(theta);


distance(L=Label("$|b_ju^j|$", align=S+E),  w, b0, 18mm, gray, joinpen=dotted);
distance(L=Label("$|\mathcal O(n)|$", align=N+E), w,q, 2cm, gray, joinpen=dotted);

draw(arc(w,rwq, 25, 230), dotted+gray);
draw((0,0)--(b0));
draw(arc((0,0), 0.5, 0, theta), L=Label("$\theta$"));
draw(q--w--q_, StickIntervalMarker(2,2,p=gray, size=0.2cm, space=0.05cm));
draw((0,0)--q--q_, dashed);


draw( (-0.5, 0)--(5,0));
draw( (0,-0.1)--(0,3.5));

dot("$b_0$", b0, N+E);
dot("$b_0+b_ju^j$", w, E+S);
dot("$q(u)$", q, N+W);
dot("$Q$", q_, 1.2E+S);
dot("$O$", (0,0), W+2S);
    \end{asy}
    \end{center}
    % <==
    \begin{itemize}
      \item How to choose $r$?\\[0.3cm]
        Since we want $\abs{b_ju^j}<\abs{b_0}$, we can just choose
        $r<\abs*{\frac{b_0}{b_j}}^{\frac{1}{j}}$.
        Next we also want $\abs{\mathcal O(u}<\abs{b_j}r^j$. Observe that
        \[
          \abs{\mathcal O(u)}
          \leq \abs{b_{j+1}}r^{j+1}+\cdots+\abs{b_{n-1}}r^{n-1}+r^n
        \]
        So, if we choose $r<1$ and denote
        $B:=\max\{\abs{b_{j+1}},\dots,\abs{b_{n-1}},1\}$
        then we would obtain that
        \begin{align*}
          \abs{\mathcal O(u)}
          &<B(r^{j+1}+\cdots+r^{n-1}+r^n)\\
          &<B(\underbrace{r^{j+1}+\cdots+r^{j+1}+r^{j+1}}_{n-j})
          &&(\text{becase we choose $r<1$})\\
          &<B(n-j)r\cdot r^{j}
        \end{align*}
        Finally, if we choose $r<\frac{\abs{b_j}}{B(n-j)}$, then 
        we would have
        $\abs{\mathcal O(u)}<\abs{b_j}u^j$.
        Thus the choice of $r$ is 
        \[
          r:=\min\left\{
            \left(\frac{b_0}{b_j}\right)^{\frac{1}{j}},~
            \frac{\abs{b_j}}{B(n-j)},~
            1
          \right\}.
        \]

      \item How do we choose $\phi$?\\[0.3cm]
        Because $b_ju^j\parallel b_0$,
        then 
        \begin{align*}
          &\frac{b_0+b_ju^j}{\abs{b_0+b_ju^j}}=\frac{b_0}{\abs{b_0}}\\\implies\quad
          &\frac{b_0+b_ju^j}{\abs{b_0}-\abs{b_ju^j}}=\frac{b_0}{\abs{b_0}}\\\implies\quad
          &u^j=\frac{\frac{b_0}{\abs{b_0}}(\abs{b_0}-\abs{b_j}r^j)-b_0}{b_j}\\\implies\quad
          &\e^{\i\phi j}=\frac{\frac{b_0}{\abs{b_0}}(\abs{b_0}-\abs{b_j}r^j)-b_0}{b_jr^j}:=w\\\implies\quad
          &\phi=\frac{1}{j}\arg w
        \end{align*}
    \end{itemize}
    Thus we have found $u=r\e^{i\phi}$. From triangle inequality,
    \begin{align*}
      p(z)=\abs{q(u)}
      &\leq \abs{b_0+b_ju^j}+\abs{\mathcal O(u)}\\
      &=\abs{OQ}<\abs{b_0}=\abs{p(z_0)}
    \end{align*}
    Thus we have found $z$ such that $p(z)<\abs{p(z_0)}$. But this
    is a contradiction becase $z_0$ is supposed to be the minimum.
    Therefore, $\abs{p(z_0)}=0$.
  \end{proof}
  So far so good, now we're going to  finish this mess by using the last
  claim we proved. Because $\abs{p(z_0)}=0$, we obtain that $p(z_0)=0$
  which means that $z_0$ is a root of $p$. ;)
\end{proof}





\end{document}
