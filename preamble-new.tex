\PassOptionsToPackage{dvipsnames,table,usenames,svgnames}{xcolor}
\documentclass[a4paper, 10pt]{memoir}

% ==> language
\usepackage[no-math]{fontspec}
\usepackage{fontawesome}
\usepackage{mathpazo}
\setmainfont{TeX Gyre Pagella}
% <== 
% ==> figure, geometry
\usepackage[inline]{asymptote}
\def\asydir{asydir}
\usepackage{tikz}
\usetikzlibrary{arrows, calc, angles, quotes}
% <==
% ==> math
% ==> basic
\usepackage{amsthm}
\usepackage{amsfonts}
\usepackage{amssymb}
\usepackage{amsmath}
\usepackage{mathtools}
\usepackage{thmtools}
\usepackage[framemethod=TikZ]{mdframed}


\mathtoolsset{centercolon} % not work when using |mathpazo|
\DeclarePairedDelimiter\abs{\lvert}{\rvert}
\DeclarePairedDelimiterX\norm[1]\lVert\rVert{
	\ifblank{#1}{\:\cdot\:}{#1}
}
\def\set#1#2{\left\{#1 ~:~ #2\right\}}
\def\permil{\text{\hskip 0.3pt\englishfont\textperthousand}}


\DeclareMathOperator{\arccot}{cot}
\DeclareMathOperator{\arcsec}{arcsec}
\DeclareMathOperator{\arccsc}{arccsc}
\DeclareMathOperator{\lcm}{lcm}
\DeclareMathOperator{\ord}{ord}
\DeclareMathOperator{\sym}{sym}
\DeclareMathOperator{\tr}{trace}
\DeclareMathOperator{\spans}{span}
\DeclareMathOperator{\rank}{rank}
\DeclareMathOperator{\col}{col}
\DeclareMathOperator{\row}{row}
\DeclareMathOperator{\sign}{sign}
\DeclareMathOperator{\perm}{perm}
\DeclareMathOperator{\coef}{coef}
%\DeclareMathOperator{\proj}{proj}

\usepackage{bbold}
\let\altmathbb\mathbb
\AtBeginDocument{\let\mathbb\altmathbb}



\def\N{\mathbb N}
\def\Z{\mathbb Z}
\def\Q{\mathbb Q}
\def\R{\mathbb R}
\def\C{\mathbb C}
\def\F{\mathbb F}
\def\P{\mathbb P}

\def\L{\mathcal L}
\def\U{\mathcal U}
\def\i{\mathrm i}
\def\e{\mathrm e}

\def\labelitemi{$\circ$}
\def\inv{^{-1}}
\def\ang#1{\left\langle#1\right\rangle}
\def\tran{^\mathrm{T}}
\renewcommand{\vec}[1]{\mathbf{#1}}
\def\perc{^\perp}
\def\proj#1#2{\mathrm{proj}_{#2}{#1}}
%\renewcommand{\span}{\SPAN}

\newtheorem{homework}{Homework}
\renewcommand{\proofname}{\textcolor{blue}{\bfseries Proof}}
% <==
% ==> mdframe theorem
\usepackage{thmtools}
\usepackage[framemethod=TikZ]{mdframed}

\def\axiomName{{\large\faThumbTack}~~Axiom}
\def\definitionName{{\large\faThumbTack}~~Definition}
\def\theoremName{{\large\faFlask}~~Theorem}
\def\lemmaName{{\large\faFlask}~~Lemma}
\def\propositionName{{\large\faFlask}~~Proposition}
\def\corollaryName{{\large\faFlask}~~Corollary}
\def\exampleName{Example}
\def\exerciseName{Exercise}
\def\proofName{Proof}
\def\solutionName{Solution}
\def\remarkName{{\large\faExclamationTriangle}~~Remark}

% ==> Axiom
\mdfdefinestyle{mdAxiomStyle}{
	roundcorner = 5pt,
	linewidth=1.125pt,
	skipabove=12pt,
	innerbottommargin=9pt,
	skipbelow=2pt,
	nobreak=false,
	linecolor=blue,
	backgroundcolor=TealBlue!5,
}
\declaretheoremstyle[
	headfont=\large\bfseries\color{MidnightBlue},
	bodyfont=\itshape,
	mdframed={style=mdAxiomStyle},
	headpunct={\\[3pt]},
	postheadspace={0pt},
	notefont=\normalfont\small\bfseries\color{gray!50!blue},
	notebraces={~(}{)},
]{mdAxiom}
% <==
% ==> Definition
\mdfdefinestyle{mdDefinitionStyle}{
	roundcorner = 6pt,
	linewidth=1.125pt,
	skipabove=12pt,
	innerbottommargin=9pt,
	skipbelow=2pt,
	nobreak=false,
	linecolor=red!80,
	backgroundcolor=orange!10,
}
\declaretheoremstyle[
	headfont=\large\bfseries\color{red!80},
	bodyfont=\normalfont,
	mdframed={style=mdDefinitionStyle},
	headpunct={\\[3pt]},
	postheadspace={0pt},
	notefont=\normalfont\small\bfseries\color{red!80},
	notebraces={~(}{)},
]{mdDefinition}
% <==
% ==> Theorem
\mdfdefinestyle{mdTheoremStyle}{%
	linewidth=1pt,
	skipabove=12pt,
	frametitleaboveskip=5pt,
	frametitlebelowskip=0pt,
	skipbelow=2pt,
	innertopmargin=5pt,
	innerbottommargin=8pt,
	nobreak=false,
	linecolor=magenta,
	backgroundcolor=Goldenrod!10,
}
\declaretheoremstyle[
	headfont=\large\bfseries\color{magenta},
  headindent=0pt,
	bodyfont=\itshape,
	mdframed={style=mdTheoremStyle},
	notefont=\normalfont\itshape\small\color{magenta},
	notebraces={~$\big\langle$}{$\big\rangle$},
	headpunct={\\[3pt]},
	postheadspace={0pt},
]{mdTheorem}
% <==
% ==> Remark
\mdfdefinestyle{mdRemarkStyle}{%
	rightline=false,
	leftline=true,
	topline=false,
	bottomline=false,
	linewidth=1pt,
	skipabove=12pt,
	frametitleaboveskip=5pt,
	frametitlebelowskip=0pt,
	skipbelow=2pt,
	innertopmargin=5pt,
	innerbottommargin=8pt,
	nobreak=false,
	linecolor=magenta,
	backgroundcolor=Goldenrod!10,
}
\declaretheoremstyle[
	headfont=\large\bfseries\color{magenta},
  headindent=0pt,
	bodyfont=\itshape,
	mdframed={style=mdRemarkStyle},
	notefont=\normalfont\itshape\small\color{magenta},
	notebraces={~$\big\langle$}{$\big\rangle$},
	headpunct={\\[3pt]},
	postheadspace={0pt},
]{mdRemark}
% <==
% ==> Example
\mdfdefinestyle{mdExampleStyle}{%
	skipabove=5pt,
	skipbelow=5pt,
	rightline=false,
	leftline=false,
	topline=false,
	bottomline=false,
	backgroundcolor=white
}
\declaretheoremstyle[
	headfont=\bfseries\color{blue},
	bodyfont=\normalfont,
	headindent=0em,
	spaceabove=2pt,
	spacebelow=5pt,
	mdframed={style=mdExampleStyle},
	notefont=\normalfont\small\itshape\color{gray!50!blue},
	notebraces={~(}{)},
	headpunct={ },
	qed={\color{blue}$\blacktriangleleft$}
]{mdExample}
% <==
% ==> Exercise
\mdfdefinestyle{mdExerciseStyle}{%
	skipabove=5pt,
	skipbelow=5pt,
	linewidth=1pt,
	rightline=true,
	leftline=true,
	topline=true,
	bottomline=true,
	linecolor=blue,
	backgroundcolor=white
}
\declaretheoremstyle[
	headfont=\bfseries\color{blue},
	bodyfont=\normalfont,
	headindent=0pt,
	spaceabove=2pt,
	spacebelow=8pt,
	mdframed={style=mdExerciseStyle},
	notefont=\normalfont\small\itshape\color{DarkGreen},
	notebraces={~(}{)},
	headpunct={ },
]{mdExercise}
% <==
% ==> Proof
\declaretheoremstyle[
	spaceabove=6pt,
	spacebelow=6pt,
	headindent=0pt,
	headfont=\normalfont\color{magenta},
	bodyfont = \normalfont,
	postheadspace=1em,
	qed=$\color{magenta}\blacksquare$,
	headpunct={ },
]{mdProof}

\newenvironment{prf}{\begin{proof}}{\end{proof}}

% <==
% ==> Declaring the theorems
\declaretheorem{axiom}[style=mdDefinition, name={\axiomName}, numberwithin=chapter]
\declaretheorem[style=mdDefinition, name={\definitionName}, sibling=axiom]{definition}

\declaretheorem{theorem}[style=mdTheorem, name={\theoremName},numberwithin=chapter]
\declaretheorem{lemma}[style=mdTheorem, name={\lemmaName},sibling=theorem]
\declaretheorem{proposition}[style=mdTheorem, name={\propositionName}, sibling=theorem]
\declaretheorem{corollary}[style=mdTheorem, sibling=theorem, name={\corollaryName}]
\declaretheorem{remark}[style=mdRemark, sibling=theorem, numbered=no, name={\remarkName}]

\declaretheorem{example}[name={\exampleName},style=mdExercise, numberwithin=chapter]
\declaretheorem{exercise}[name={\exerciseName},style=mdExercise, sibling=example]
\declaretheorem{problem}[name={\exerciseName},style=mdExercise, numberwithin=chapter]

%% Proof
\declaretheorem[name={\bfseries\proofName},numbered=no,style=mdProof]{Proof}
\renewenvironment{proof}{\begin{Proof}}{\end{Proof}}
\declaretheorem[name={\itshape\solutionName},numbered=no,style=mdProof]{solution}
% <==
% <==




% <== end: math
% ==> enumerate
\usepackage{enumitem}
\usepackage{multicol}
\def\labelitemi{{\faCaretRight}}
% <==
% ==> listings
\usepackage{xcolor}
\usepackage{listings}
% ==> basic
\lstset{%
	basicstyle=\small\ttfamily,
	keywordstyle=\color{black},
	commentstyle=\color{gray},
	keywordstyle=[1]{\color{blue!90!black}},
	keywordstyle=[2]{\color{magenta!90!black}},
	keywordstyle=[3]{\color{red!60!orange}},
	keywordstyle=[4]{\color{teal}},
  keywordstyle=[5]{\color{magenta!90!black}}, % <-- otherkeywords, BUG
	commentstyle=\color{gray},
	stringstyle=\color{green!60!black},
	tabsize=2,
	%
	numbers=left,
	numberstyle=\tiny\color{blue!70!gray},
	stepnumber=1,
	%
	frame=Lt,
	breaklines=true,
	xleftmargin=0cm,
	rulecolor=\color{gray!50!black},
	aboveskip=0.5cm,
	belowskip=0.5cm
}
% <==
% ==> code c
\lstdefinelanguage{cmeng}{
  morekeywords={
    auto,break,case,char,const,continue,default,do,double,%
    else,enum,extern,float,for,goto,if,int,long,register,return,%
    short,signed,sizeof,static,struct,switch,typedef,union,unsigned,%
    void,volatile,while},%
  morekeywords=[2]{
    printf, scanf,  include
  },
  otherkeywords={\#,<,>,\&},
  morekeywords=[5]{\#,<,>,\&},
  sensitive,%
	morecomment=[l]{//},
	morecomment=[s]{/*}{*/},
	morestring=[b]',
	morestring=[b]",
}
% <==
% ==> code python
\lstdefinelanguage{py}{
	morekeywords={
		access,and,as,break,class,continue,def,del,elif,else,
		except,exec,finally,for,from,global,if,import,in,is,lambda,
		not,or,pass,print,raise,return,try,while},
	% Built-ins
	morekeywords=[2]{
		abs,all,any,basestring,bin,bool,bytearray,
		callable,chr,classmethod,cmp,compile,complex,delattr,dict,dir,
		divmod,enumerate,eval,execfile,file,filter,float,format,
		frozenset,getattr,globals,hasattr,hash,help,hex,id,input,int,
		isinstance,issubclass,iter,len,list,locals,long,map,max,
		memoryview,min,next,object,oct,open,ord,pow,property,range,
		raw_input,reduce,reload,repr,reversed,round,set,setattr,slice,
		sorted,staticmethod,str,sum,super,tuple,type,unichr,unicode,
		vars,xrange,zip,apply,buffer,coerce,intern,True,False},
	%
	morecomment=[l]\#,%
	morestring=[b]',%
	morestring=[b]",%
	morecomment=[s]{'''}{'''},% used for documentation text
	%                         % (mulitiline strings)
	morecomment=[s]{"""}{"""},% added by Philipp Matthias Hahn
	morestring=[s]{r'}{'},% `raw' strings
	morestring=[s]{r"}{"},%
	morestring=[s]{r'''}{'''},%
	morestring=[s]{r"""}{"""},%
	morestring=[s]{u'}{'},% unicode strings
	morestring=[s]{u"}{"},%
	morestring=[s]{u'''}{'''},%
	morestring=[s]{u"""}{"""},%
	%
	sensitive=true,%
}
% <==
% ==> code asy
\lstdefinelanguage{asy}{ %% Added by Sivmeng HUN
	morekeywords=[1]{
		import, for, if, else,new, do,and, access,
		from, while, break, continue, unravel, 
		operator, include, return},
	morekeywords=[2]{
		struct,typedef,static,public,readable,private,explicit,
		void,bool,int,real,string,var,picture,
		pair, path, pair3, path3, triple, transform, guide, pen, frame
	},
	morekeywords=[3]{
		true,false,and,cycle,controls,tension,atleast,
		curl,null,nullframe,nullpath,
		currentpicture,currentpen,currentprojection,
		inch,inches,cm,mm,pt,bp,up,down,right,left,
		E,N,S,W,NE,NW,SE,SW,
		solid,dashed,dashdotted,longdashed,longdashdotted,
		squarecap,roundcap,extendcap,miterjoin,roundjoin,
		beveljoin,zerowinding,evenodd,invisible
	},
	morekeywords=[4]{
		size,unitsize,draw,dot,label,
		sqrt,sin,cos,tan,cot,Sin,Cos,Tan,Cot,
		graph,
	},
	%
	morecomment=[l]{//},
	morecomment=[s]{/*}{*/},
	morestring=[b]',
	morestring=[b]",
	%
}
% <==
% <==
% ==> tweaking memoir
\makepagestyle{myruled}
\makeheadrule {myruled}{\textwidth}{\normalrulethickness}
\makefootrule {myruled}{\textwidth}{\normalrulethickness}{\footruleskip}
\makeevenhead {myruled}{}{\small\itshape\leftmark} {}
\makeoddhead  {myruled}{}{\small\itshape\rightmark}{}
\makeevenfoot {myruled}{}{\small \thepage} {}
\makeoddfoot  {myruled}{}{\small \thepage} {}

\makeatletter % because of \@chapapp
\makepsmarks{myruled}{
  \nouppercaseheads
  \createmark{chapter}{both}{shownumber}{\@chapapp\ }{. \ }
  \createmark{section}{right}{shownumber}{}{. \ }
  \createmark{subsection}{right}{shownumber}{}{. \ }
  \createmark{subsubsection}{right}{shownumber}{}{. \ }
  \createplainmark {toc} {both} {\contentsname} 
  \createplainmark {lof} {both} {\listfigurename}
  \createplainmark {lot} {both} {\listtablename} 
  \createplainmark {bib} {both} {\bibname} 
  \createplainmark {index} {both} {\indexname} 
  \createplainmark {glossary} {both} {\glossaryname}
}
\makeatother
\setsecnumdepth{subsubsection}
\pagestyle{myruled}
% <==
% ==> page setup
\author{Sivmeng}
% <==
